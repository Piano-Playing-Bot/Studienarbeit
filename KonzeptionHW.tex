% !TEX root =  master.tex
\chapter{Konzeption - Hardware}\label{Hardware - Konzeption}

\nocite{*}

Für die Umsetzung eines selbstspielenden Klaviers wurden verschiedene technische Überlegungen angestellt.
Die Grundfragen, denen wir bei der Konstruktion begegneten, betrafen die Auswahl und Implementierung der
Aktuatoren, die Methode des Anspielens der Klaviertasten sowie die Signalübertragung des Arduinos.
Im Hinblick auf die Aktuatoren fiel unsere Wahl auf Hubmagneten, die eine präzise Steuerung der Tasten
ermöglichen. Durch ihre Verwendung können die Tasten des Klaviers mit der erforderlichen Geschwindigkeit und
Genauigkeit angespielt werden. \newline
Die Aktuatoren wurden mit den Klaviertasten mittels Seilen verbunden, um die mechanische Integrität des Klaviers
zu erhalten und gleichzeitig eine präzise Steuerung zu gewährleisten.\newline
Die Signalübertragung erfolgte über einen Arduino-Mikrocontroller. Da der Arduino nicht über ausreichend
viele Ausgänge verfügt, um alle Tasten anzusteuern, wurde das Signal mithilfe von Schieberegistern erweitert.
Diese Erweiterung ermöglichte es, alle 88 Tasten des Klaviers anzusteuern, ohne zusätzliche Hardware einsetzen
zu müssen.

\section{Elektronik}\label{Vorgehen - Hardware}


\subsection{Arduino R3}\label{Ansteuerung}
Zur Signalsteuerung wird ein Arduino R3 genutzt.
Es handelt sich dabei um eine Open-Source-Plattform für die Entwicklung von elektronischen Prototypen. Er besteht aus
einem Mikrocontroller-Board, das mit verschiedenen Sensoren, Aktuatoren und anderen elektronischen Komponenten
verbunden werden kann. Der Arduino verfügt über digitale und analoge Ein- und Ausgangspins, die für die Interaktion mit
externen Geräten verwendet werden können. \newline
Der Arduino Uno R3 eignet sich hervorragend zur Ansteuerung der Aktuatoren eines selbstspielenden Klaviers aus
mehreren Gründen. Erstens bietet der Arduino eine benutzerfreundliche Plattform für die Entwicklung und Implementierung
von Steuerungslogik. Die einfache Programmierung mittels der Arduino-IDE ermöglicht es, komplexe Steuerungsalgorithmen
für die Aktuatoren zu erstellen. \newline
Zweitens bietet er PWM (Pulsweitenmodulation) Pins, die im Rahmen des Projekts genutzt wurden.
//TODO: Speicher \newline
//TODO: Vergleich mit anderen Microcontrollern 

\subsection{Pulsweitenmodulation (PWM)}\label{PWM}
PWM ist einfach ausgedrückt eine Technik, bei der die Zeitdauer eines digitalen Signals variiert wird, um einen
durchschnittlichen Wert zu erzeugen. Dies ermöglicht es, die Aktuatoren mit verschiedenen Stärken anzusteuern, indem die
Pulsweite des Signals
entsprechend angepasst wird. Dadurch können die Klaviertasten mit unterschiedlichen Intensitäten angespielt werden,
was zu einer realistischeren und dynamischeren Wiedergabe führt.\newline

Eine Digitale Steuerung wird verwendet, um eine Rechteckwelle - ein Signal das zwischen Ein und Aus umgeschaltet wird
- zu erzeugen. Dieses Ein-Aus-Muster kann Spannungen zwischen der vollen Versorgungsspannung (5V) und Aus (0 Volt)
simulieren.
Dabei wird der Anteil der Zeit geändert, den das Signal eingeschaltet ist im Vergleich zur Zeit, die das Signal
ausgeschaltet ist. Die Dauer der Einschaltzeit wird als Pulsweite bezeichnet. Um unterschiedliche analoge Werte zu
erhalten, wird die Pulsweite geändert bzw. moduliert. Wenn dieses Ein-Aus-Muster schnell genug mit einer LED
wiederholt wird, resultiert daraus eine konstante Spannung zwischen 0 und Vcc, die die Helligkeit
der LED steuert. \newline

Auf den Arduino bezogen sieht die Umsetzung eines PWM Signals wie folgt aus:
Ein Taktsignal gelangt in die entsprechende Clock.
Die Clock stellt den entsprechenden PWM-Modus ein. Dabei werden zwei wichtige Werte gesetzt:
Der erste bestimmt, wann das Signal von HIGH auf LOW
umschaltet, während der zweite bestimmt, wann es zurückkommt. Das Verhältnis zwischen HIGH und LOW wird als
Tastverhältnis bezeichnet und bestimmt die Helligkeit unserer LED bzw. die Stärke, mit welcher der Hubmagnet anschlägt.
Je länger die Ausgabe im HIGH-Zustand bleibt, desto schneller erfolgt der Tastenanschlag. \newline

Neben dem Tastverhältnis, also das Verhältnis der Einschaltzeit zur
Periodendauer, welches oft in Prozent ausgedrückt wird, ist auch die Resolution (de: Auflösung) ein variierbarer Parameter.
Die Resolution bezieht sich auf die Anzahl der möglichen diskreten Werte, die das Signal annehmen kann.



\subsection{Vermehrung der Ausgänge}
Es werden 88 Tasten angesteuert. Da ein Arduino keine 88 PWM-Ports besitzt, müssen die Signale über
eine Erweiterung der Ausgänge an die Motoren weitergegeben werden. Dafür gibt es mehrere Möglichkeiten:

\begin{enumerate}
	\item Schieberegister
	\item Motor-Matrix
	\item Demultiplexer
\end{enumerate}

 \paragraph{Schieberegister}
Ein Schieberegister ist ein integrierter Schaltkreis, der hier verwendet wird, um die Anzahl der verfügbaren Ausgänge des
Mikrocontrollers zu erweitern. Das gängigste Schieberegister ist das 8-Bit-Schieberegister. \newline

Um 88 Ausgänge mit Hilfe von Schieberegistern von nur 3 PWM-Pins des Arduinos anzusteuern, werden insgesamt 11
Schieberegister benötigt.\newline

Zuerst werden die Datenleitungen (SERIAL IN) aller Schieberegister in Reihe geschaltet. Dies bedeutet, dass das
Ausgangssignal des ersten Schieberegisters mit dem Eingang (SERIAL IN) des nächsten verbunden wird, und so weiter, bis
zum letzten Schieberegister. \newline
Der Arduino sendet Daten seriell über die verbundenen PWM-Pin. Diese Daten werden über ein Schieberegister an
alle Schieberegister weitergegeben. Durch diesen Vorgang werden die Daten parallel über alle Schieberegister gesendet.
\newline
Die Ausgänge der Schieberegister (Q0 bis Q7) werden dann an die entsprechenden Aktuatoren angeschlossen.
Das Signal wird zu einer definierten Clock durch die Schieberegister durchgeschoben. So kann durch Steuerung der Daten,
die an die Schieberegister gesendet werden, kann der Arduino jeden einzelnen Ausgang manipulieren, um die Aktuatoren
entsprechend anzusteuern. \newline
Auf diese Weise kann der Arduino mit Hilfe von Schieberegistern eine größere Anzahl von Ausgängen kontrollieren, als er
direkt verfügbar hat.


\paragraph{Motor-Matrix}
Die Motor-Matrix ist einer LED-Matrix nachgeahmt.
In einer Matrix, werden zwei Reihen an Ports angeschlossen, nach folgendem Muster: \newline
(00)(01)(02)(03)(04)(05)(06)(07) \newline
(10)(11)(12)(13)(14)(15)(16)(17) \newline
(20)(21)(22)(23)(24)(25)(26)(27) \newline
(30)(31)(32)(33)(34)(35)(36)(37) \newline
(40)(41)(42)(43)(44)(45)(46)(47) \newline
(50)(51)(52)(53)(54)(55)(56)(57) \newline
(60)(61)(62)(63)(64)(65)(66)(67) \newline
(70)(71)(72)(73)(74)(75)(76)(77) \newline

Eine LED-Matrix besteht aus einer Anordnung von LEDs in Zeilen und Spalten. Jede LED kann unabhängig von den anderen
ein- oder ausgeschaltet werden. Die Steuerung erfolgt über Multiplexing, bei dem jede Zeile der Matrix nacheinander
aktiviert wird, während die entsprechenden LEDs in den Spalten gleichzeitig eingeschaltet werden. Durch schnelles
Wechseln zwischen den Zeilen (PWM) erscheint es den BetrachterInnen, als ob alle LEDs gleichzeitig leuchten würden,
obwohl sie tatsächlich nacheinander aktiviert werden. \newline

Um das Prinzip einer LED-Matrix auf Aktuatoren zu übertragen, müssen lediglich die LEDs durch die Aktuatoren ersetzt werden.
Durch Steuerung der Zeilen und Spalten dieses Rasters können verschiedene Motoren selektiv
aktiviert werden. Dies ermöglicht eine effiziente Steuerung mehrerer Motoren mit einem begrenzten Satz von
Steuersignalen. \newline

//TODO: Nachteile Matrix


\paragraph{Demultiplexer}
//TODO

\subsection{Transistor}
\subsection{Akkumulator}
\subsection{Schaltplan}

\section{Mechanik}\label{vorgehenHW}

\subsection{Auswahl des Klaviers}
\subsection{Ansteuerungskonzept}
\subsection{Verbindung Tasten und Hubmagnete}
\subsection{Weitere Ideen und Limitationen}
