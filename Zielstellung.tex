% !TEX root =  master.tex
\chapter{Zielstellung} \label{Zielstellung}

\nocite{*}

Die Idee der Arbeit besteht ist die Entwicklung eines Klaviers, sowohl von Personen, als auch automatisiert von einem Computer bespielt werden kann.
Hierfür wird ein kleiner Computer eingesetzt, um Hardware anzusteuern, die in der Lage ist, die Klaviertasten zu betätigen.
Die Steuerung dieses selbstspielenden Klaviers erfolgt über eine
Desktop-Anwendung, mittels derer Nutzer:innen zwischen Musikstücken wählen können.
Darüber hinaus soll die Möglichkeit bestehen, dem Katalog weitere Musikstücke hinzuzufügen.
Des Weiteren sollen Nutzer:innen die Wiedergabe-Geschwindigkeit -und Lautstärke verändern können.
Inspiriert von gängigen Musikplayern, soll das Spielen pausiert und zu beliebigen Stellen des Stücks gesprungen werden können.


\section{Vorgehensweise} \label{sec:zielstellung-vorgehen}
Die Vorgehensweise des Projekts gliedert sich in mehrere Schlüsselschritte.
Zunächst erfolgt die Konzeptualisierung des selbstspielenden Klaviers, wobei die spezifischen Anforderungen und Funktionalitäten
definiert werden.//TODO: noch An die Struktur anpassen!!!!
Hierbei liegt ein besonderer Fokus auf der Identifikation der Schnittstellen zwischen der
Hardware und Software, wobei die Arduino-Plattform als zentrales Steuerungselement berücksichtigt wird. \newline

Im Anschluss erfolgt die Hardware-Implementierung mithilfe des Arduinos. Dies beinhaltet die sorgfältige
Auswahl geeigneter Hardwarekomponenten, die in der Lage sind, die Klaviertasten präzise anzusteuern.
Die Programmierung des Arduinos erfolgt mit dem Ziel, eine nahtlose Integration in das Gesamtsystem zu
gewährleisten. \newline

Die nachfolgende Etappe konzentriert sich auf die Entwicklung einer
Desktop-Anwendung, die als Schnittstelle für die Steuerung des selbstspielenden Klaviers dient. Hierbei wird
besonderes Augenmerk auf die Benutzerfreundlichkeit gelegt, und die Anwendung ermöglicht Nutzer:innen die
Auswahl und Wiedergabe von Musikstücken aus einem vordefinierten Katalog. \newline

Ein weiterer Schwerpunkt liegt auf der Integration von MIDI-Dateien, um dem Katalog kontinuierlich weitere
Musikstücke hinzufügen zu können.
Dieser Prozess beinhaltet die Entwicklung einer Schnittstelle, die eine
unkomplizierte Integration neuer Musikstücke in die bestehende Datenbank ermöglicht.