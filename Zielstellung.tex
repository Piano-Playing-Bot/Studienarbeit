%%%%%%%%%%%%%%%%%%%%%%%%%%%%%%%%%%%%%%%%%%%%%%%%%%%%%%%%%%
%   Autoren des Abschnitts:
%   Olivier Stenzel
%%%%%%%%%%%%%%%%%%%%%%%%%%%%%%%%%%%%%%%%%%%%%%%%%%%%%%%%%%

% !TEX root =  master.tex
\chapter{Zielstellung} \label{Zielstellung}

\nocite{*}


Das Kernanliegen dieser Arbeit ist die Entwicklung eines multifunktionalen Klaviers,
das sowohl manuell von Musiker:innen als auch automatisiert durch einen Computer bespielt werden kann.
Zu diesem Zweck wird ein kompakter Mikrocontroller genutzt, der die Aufgabe hat, spezifische Hardwarekomponenten zu steuern.
Diese Hardware ist dazu in der Lage, eigenständig die Tasten des Klaviers zu bedienen und somit die Saiten zum Schwingen zu bringen.
% @Note(Val): Warum starten wir mit einer Beschreibung der Umsetzung statt einer Beschreibung des Ziels hier?
% @Note(Val): In diesem Kapitel würde ich immer "soll" verwenden, um klarzustellen, dass zum Stand dieses Kapitels noch keine Implementierung existiert

Die Interaktion mit dem selbstspielenden Klavier erfolgt über eine Desktop-Anwendung, die es Nutzerinnen und Nutzern ermöglicht,
aus einer Sammlung von Musikstücken zu wählen.
Ein wesentlicher Aspekt des Projekts ist zudem die Flexibilität des Systems: Nutzende sollen in der Lage sein,
den bestehenden Musik-Katalog durch eigene Stücke zu erweitern.
Neben der Auswahl von Musikstücken bietet die Anwendung auch die Möglichkeit,
die Wiedergabegeschwindigkeit und die Lautstärke individuell anzupassen.

% @Note(Val): Dieser Absatz und generell dieser ganze Einstieg liest sich eher nach einer Werbung statt einer wissenschaftlichen Zielstellung
In Anlehnung an die Funktionsweise populärer Musikplayer wird angestrebt,
dass die Wiedergabe nicht nur pausiert, sondern auch innerhalb eines Stücks frei navigiert werden kann.
Dieses Merkmal unterstreicht nochmals den interaktiven Charakter des Instruments und wie intuitiv dessen Bedienung ist.

\section{Anforderungen} \label{sec:zielstellung-anforderungen}

Um das Ziel eines selbstspielenden Klaviers messbar zu machen, wurde Tabelle \ref{table:anforderungen} erstellt.
Darin befinden sich die allgemeinen Anforderungen, die an dieses Projekt gestellt werden.

% @TODO(Val): Hier müsste noch Text kommen. Kruse mag es ja nicht, wenn wir Tabellen/Diagramme direkt unter Überschriften haben
\begin{table}[ht]
    \centering
    \begin{tabular}{ | m{2cm} | m{10cm}| m{2cm} | }
        \hline
        \textbf{Kategorie} & \textbf{Anforderung} & \textbf{Priorität} \\
        \hline
        Allgemein & Flexibilität: Das Klavier muss sowohl manuell als auch automatisch bespielbar sein. & Hoch \\
        \cline{2-3}
        & Benutzerinterface: Entwicklung eines intuitiven Interfaces. & Hoch \\
        \cline{2-3}
        & Responsivität: Die Anwendung muss auf verschiedenen Computern mit Windows 10 und 11 laufen. & Mittel \\
        \cline{2-3}
        \hline
        Hardware & Tastenbetätigung: Hardware muss in der Lage sein, Klaviertasten autonom zu bedienen. & Hoch \\
        \cline{2-3}
        & Anpassbarkeit: Hardware-Komponenten müssen auf unterschiedliche Klaviertypen anpassbar sein. & Mittel \\
        \hline
        Software & Musikstück-Auswahl: Nutzer:innen wählen Musikstücke über eine Desktop-Anwendung. & Hoch \\
        \cline{2-3}
        & Wiedergabe-Kontrolle: Anpassung der Wiedergabegeschwindigkeit und Lautstärke. & Mittel \\
        \cline{2-3}
        & Navigation: Möglichkeit, die Wiedergabe zu pausieren und innerhalb des Stücks zu navigieren. & Mittel \\
        \cline{2-3}
        & MIDI-Integration: System muss MIDI-Dateien für die Erweiterung des Katalogs unterstützen. & Mittel \\
        \hline
    \end{tabular}
    \caption{Anforderungen an das selbstspielende Klavier}
    \label{table:anforderungen}
\end{table}


\section{Vorgehensweise} \label{sec:zielstellung-vorgehen}

In diesem Kapitel wird der geplante Ablauf des Projektes zur Entwicklung eines selbstspielenden Klaviers vorgestellt.
Die Vorgehensweise ist in mehrere Phasen unterteilt, die das parallele Arbeiten vereinfachen.

\subsubsection{Konzeptualisierung und Planung}\label{Vorgehensweise - Konzeptualisierung und Planung}

Zu Beginn des Projektes steht die Konzeptualisierung, in der die grundlegenden Anforderungen an das System festgelegt werden.
Besondere Aufmerksamkeit gilt dabei der Schnittstelle zwischen der Hardware, die durch einen kompakten Mikrocontroller gesteuert wird, und der Software der Desktop-Anwendung.
Die Planungsphase umfasst:

% @Note(Val): Sollten diese Stichpunkte stattdessen im Fließtext eingegliedert werden?
% @Note(Val): Scheint auch als würden hier viele Punkte noch fehlen, oder?
\begin{itemize}
    \item Definition der funktionalen Anforderungen an das Klavier. (siehe Kapitel \ref{sec:zielstellung-anforderungen})
    \item Auswahl der Ansteuerung-Hardware. (siehe Kapitel \ref{vorgehenHW})
    \item Bau der Ansteuerungs-Hardware. (siehe Kapitel \ref{umsetzungHW})
    \item Entwurf des Konzepts für die Desktop-Anwendung. (siehe Kapitel \ref{vorgehenSW})
    \item Entwicklung der Software. (siehe Kapitel \ref{umsetzungSW})
    \item Testen der Kombination aus Hard-und Software. (siehe Kapitel \ref{tests})
    \item Fazit. (siehe Kapitel \ref{fazit})
\end{itemize}

\subsubsection{Hardware-Implementierung}\label{Vorgehensweise - Hardware-Implementierung}

Nach der Konzeptualisierung folgt die Implementierungsphase der Hardware:

\begin{itemize}
    \item Auswahl und Testung von Komponenten, die in der Lage sind, die Klaviertasten autonom zu betätigen.
    \item Entwicklung eines Prototyps zur Überprüfung der Ansteuerung und Funktionalität der Tasten.
    \item Integration des Mikrocontrollers in das Gesamtsystem und Programmierung der Steuerungslogik.
\end{itemize}

\subsubsection{Software-Entwicklung}\label{Vorgehensweise - Software-Entwicklung}

Parallel zur Hardware-Entwicklung wird die Software konzipiert und programmiert:

\begin{itemize}
    \item Entwicklung eines Kommunikationsprotokolls zwischen Anwendungsprogramm und Mikrocontroller.
    \item Gestaltung der Benutzeroberfläche der Desktop-Anwendung, welche die Auswahl und Wiedergabe der Musikstücke ermöglicht.
    \item Implementierung von Funktionen zur Anpassung der Wiedergabegeschwindigkeit und Lautstärke.
    \item Entwicklung eines Systems zur Verwaltung und Erweiterung des Musik-Katalogs, einschließlich der Integration und des Parsings von MIDI-Dateien.
\end{itemize}

\subsubsection{Integration und Tests}\label{Vorgehensweise - Integration und Tests}

In der abschließenden Phase werden die Hardware und Software integriert und umfassend getestet:

\begin{itemize}
    \item Zusammenführung der Hardware- und Softwarekomponenten zu einem einheitlichen System.
    \item Durchführung von Tests zur Überprüfung der Funktionalität und Benutzerfreundlichkeit.
    \item Fehlerbehebung und Feinabstimmung des Systems basierend auf Testergebnissen.
\end{itemize}

Das Ergebnis dieser Vorgehensweise soll ein benutzerfreundliches, selbstspielendes Klavier sein,
das den in Kapitel \ref{sec:zielstellung-anforderungen} genannten Anforderungen entspricht und einen Mehrwert sowohl für Musiker:innen als auch für Technikaffine bietet.
