%%%%%%%%%%%%%%%%%%%%%%%%%%%%%%%%%%%%%%%%%%%%%%%%%%%%%%%%%%
%   Autoren des Abschnitts:
%   Olivier Stenzel
%%%%%%%%%%%%%%%%%%%%%%%%%%%%%%%%%%%%%%%%%%%%%%%%%%%%%%%%%%
% !TEX root =  master.tex
\chapter{Ziele und Anforderungen} \label{Zielstellung}
\chapterauthor{Olivier Stenzel}


Das Kernanliegen dieser Arbeit ist die Entwicklung eines multifunktionalen Klaviers,
das sowohl manuell von Musiker:innen als auch automatisiert durch einen Computer bespielt werden kann.

Da es wünschenswert ist, dass Nutzer:innen auf eine breite Palette fertiger Stücke zugreifen und diese Palette selbstständig erweitern können,
wäre eine Steuerung über den PC sinnvoll.
Neben der Auswahl von Musikstücken soll die Anwendung außerdem die Möglichkeit bieten,
die Wiedergabegeschwindigkeit und Lautstärke individuell anzupassen.

In Anlehnung an die Funktionsweise populärer Musikplayer wird angestrebt,
dass die Wiedergabe nicht nur pausiert, sondern auch innerhalb eines Stücks frei navigiert werden kann.

In Tabelle \ref{table:anforderungen} befinden sich die allgemeinen Anforderungen, die an dieses Projekt gestellt wurden.
Basierend auf den Anforderungen und deren Priorisierungen kann die Arbeit in kleinere Arbeitspakete aufgeteilt werden.
Zuerst sollen die Anforderungen mit hoher Priorität umgesetzt werden und im Anschluss jene mit einer niedrigeren.
Im weiteren Verlauf dieser Arbeit wird nicht weiter auf die Priorisierung eingegangen, da nur Anforderungen mit niedriger Priorität nicht umgesetzt wurden.

Wie an den genannten Anforderungen erkenntlich wird, lässt sich diese Arbeit entlang der Software-/Hardware-Grenze teilen.
Das ist insoweit sinnvoll, dass die Planung und Umsetzung der beiden Teile dann parallel getan werden kann.

\begin{table}[ht]
    \centering
    \begin{tabular}{ | m{1cm} | m{2cm} | m{8cm} | m{2cm} | }
        \theadstart{ID} & \theadcol{Kategorie} & \theadcol{Anforderung} & \theadcol{Priorität} \\
        \hline
        A1 & \multirow{3}{2cm}{Allgemein} & \textbf{Flexibilität:} Das Klavier muss sowohl manuell als auch automatisch bespielbar sein & Hoch \\
        \cline{1-1} \cline{3-4}
        A2 & & \textbf{Benutzerinterface:} Entwicklung eines intuitiven Interfaces & Hoch \\
        \cline{1-1} \cline{3-4}
        A3 & & \textbf{Budget:} Gesamtkosten < 2000\euro{} & Mittel \\
        \cline{1-1} \cline{3-4}
        A4 & & \textbf{Portabilität:} Die Anwendung muss auf verschiedenen Computern mit Windows 10 und 11 laufen & Mittel \\
        \cline{1-1} \cline{3-4}
        A5 & & \textbf{Performanz:} Die Performanz soll mindestens die von professionellen Pianist:innen erreichen,
        Genauer bedeutet dies, dass im Vollbetrieb maximal 70ms zwischen zwei Tastenanschlägen liegen sollen\footnote{siehe: \cite*[vgl.]{AnschlagGeschwindigkeit} für Herleitung der Anforderung}
        & Niedrig \\
        \cline{1-1} \cline{3-4}
        A6 & & \textbf{Spielbarkeit:} Der Aufbau sollte dazu in der Lage sein, mindestens 10 Tasten gleichzeitig anspielen zu können & Mittel \\
        \hline
        A7 & \multirow{2}{2cm}{Hardware} & \textbf{Tastenbetätigung:} Die Hardware muss in der Lage sein, Klaviertasten autonom zu bedienen & Hoch \\
        \cline{1-1} \cline{3-4}
        A8 & & \textbf{Anpassbarkeit:} Hardware- \& Software-Komponenten müssen auf unterschiedliche Klaviertypen anpassbar sein & Mittel \\
        \hline
        A9 & \multirow{4}{2cm}{Software} & \textbf{Musikstück-Auswahl:} Musikstücke sind über eine Desktop-Anwendung wählbar & Hoch \\
        \cline{1-1} \cline{3-4}
        A10 & & \textbf{Wiedergabe-Kontrolle:} Anpassung der Wiedergabegeschwindigkeit und Lautstärke & Mittel \\
        \cline{1-1} \cline{3-4}
        A11 & & \textbf{Navigation:} Möglichkeit, die Wiedergabe zu pausieren und innerhalb des Stücks zu navigieren & Mittel \\
        \cline{1-1} \cline{3-4}
        A12 & & \textbf{\ac{MIDI}-Integration:} System muss \ac{MIDI}-Dateien für die Erweiterung des Katalogs unterstützen & Mittel \\
        \cline{1-1} \cline{3-4}
        A13 & & \textbf{Musikverwaltung:} Musikstücke im Katalog müssen jederzeit umbenannt und entfernt werden können & Niedrig \\
        \hline
    \end{tabular}
    \caption{Anforderungen an das selbstspielende Klavier}
    \label{table:anforderungen}
\end{table}

