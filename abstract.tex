%%%%%%%%%%%%%%%%%%%%%%%%%%%%%%%%%%%%%%%%%%%%%%%%%%%%%%%%%%
%   Autoren des Abschnitts:
%   ???
%%%%%%%%%%%%%%%%%%%%%%%%%%%%%%%%%%%%%%%%%%%%%%%%%%%%%%%%%%

% !TEX root =  master.tex
\chapter*{Kurzfassung (Abstract)}
\addcontentsline{toc}{chapter}{Kurzfassung (Abstract)}
\chapterauthor{Jakob Kautz}

\paragraph*{Deutsch:}

Diese Studienarbeit beschäftigt sich mit der Konzeption eines selbstspielenden Klaviers, das über einen Arduino gesteuert
wird. Das Hauptziel besteht darin, eine Benutzeroberfläche zu entwickeln, über die Benutzerinnen und Benutzer Lieder
aus einer Bibliothek auswählen oder eigene MIDI-Dateien einspielen können. Die MIDI-Dateien werden angepasst und in
Signale umgewandelt, die dem Arduino und somit den Aktuatoren, die das Klavier betätigen, übermittelt werden. Der
praktische Teil der Arbeit umfasst die Umsetzung von mindestens einer Oktave - 12 Tönen. Die Arbeit konzentriert sich
hauptsächlich auf die erste Konzeption. Themen wie mechanische Belastbarkeit oder erweiterter Brandschutz werden nicht
behandelt.


\paragraph*{Englisch:}
This study focuses on the conception of a self-playing piano controlled via an Arduino. The main objective is to develop
a user interface through which users can select songs from a library or input their own MIDI files. The MIDI files are
adapted and converted into signals that are transmitted to the Arduino and thus to the actuators that operate the piano.
The practical part of the work involves implementing at least one octave - 12 notes. The focus of the work is primarily
on the initial conception. Topics such as mechanical robustness or extended fire protection are not considered in this
work.

