% !TEX root =  master.tex
\chapter{Anhang: Quellcode} \label{appndix-code}

Der vollständige Quellcode kann über Github heruntergeladen werden.
Die Software wurde dabei in drei Repositories aufgeteilt.

% @TODO(Val): Actually create these releases on github
\begin{enumerate}
	\item Code der Desktop-Anwendung \ac{SAM}: \url{https://github.com/Piano-Playing-Bot/SAM/releases/tag/v1.0}
	\item Code für den \ac{MC}: \url{https://github.com/Piano-Playing-Bot/Arduino/releases/tag/v1.0}
	\item Code der von \ac{SAM} und \ac{MC} verwendet wird: \url{https://github.com/Piano-Playing-Bot/common/releases/tag/v1.0}
\end{enumerate}

Instruktionen zum Kompilieren und Ausführen des Codes finden sich dabei jeweils in der \enquote{README}-Datei, die in jedem der Repositories vorhanden ist.

Im Repository der Anwendung \ac{SAM} ist außerdem bereits eine Zip-Datei mit ausführbarem, 64-bit, Windows Executable vorhanden.