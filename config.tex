% !TEX root =  master.tex

%%%%%%%%%%%%%%%%%%%%%%%%%%%%%%%%%%%%%%%%%%%%%%%%%%%%%%%%%%%%%%%%%%
%	ANLEITUNG:
% Passen Sie gegebenenfalls alle Stellen im Dokument an, die mit
% @stud
% markiert sind.
%%%%%%%%%%%%%%%%%%%%%%%%%%%%%%%%%%%%%%%%%%%%%%%%%%%%%%%%%%%%%%%%%%

%%
%% @stud
%%
%% LANGUAGE SETTINGS
%% Schriftarten- und Zeichenpakete
\usepackage[ngerman]{babel}				% german language
\usepackage[T1]{fontenc}				% For font encodings
\usepackage[utf8]{inputenc}				% For input encodings
\usepackage[german=quotes]{csquotes}	% correct quoting using \enquote{}

\usepackage{makeidx}					% allows index generation
\usepackage{listings}					% Format Listings properly
\usepackage{lipsum}						% Blindtext
\usepackage{graphicx}					% use various graphics formats
\usepackage[german]{varioref}			% nicer references \vref
\usepackage[format=plain]{caption}		% better Captions (format=plain to avoid hanging indentation)
\usepackage{booktabs}					% nicer Tabs
\usepackage[hidelinks=true]{hyperref}	% keine roten Markierungen bei Links
\usepackage{fnpct}						% Correct superscripts
\usepackage{calc}						% Used for extra space below footsepline, in particular
\usepackage{array}						% Better Array & Tabular environments
\usepackage{acronym}					% Allows using acronyms
\usepackage{algorithm}					% provides block command \algorithm
\usepackage{algpseudocode}				% Typesetting pseudocode
\usepackage{setspace}					% Allows setting line-spacing
\usepackage{tocloft}					% better table of contents, lists of figures/tables
\usepackage{tikz}						% Used for drawing directly in LaTeX
\usepackage{amsmath}					% Advanced Maths
\usepackage{multirow}

%% Table-Header
%% taken from: https://tex.stackexchange.com/a/102970
\newcommand*{\theadstart}[1]{\hline \multicolumn{1}{|c|}{\bfseries #1}}
\newcommand*{\theadcol}[1]{\multicolumn{1}{c|}{\bfseries #1}}

%%
%% Tikz Style Configuration
%%
\usetikzlibrary{shapes.geometric, arrows} % Use the tikz arrows library
\tikzstyle{rect} = [rectangle, rounded corners, text centered, draw=black, minimum width=2cm, minimum height=1cm]
% \tikzstyle{startstop} = [rectangle, rounded corners, minimum width=3cm, minimum height=1cm,text centered, draw=black, fill=red!30]
% \tikzstyle{io} = [trapezium, trapezium left angle=70, trapezium right angle=110, minimum width=3cm, minimum height=1cm, text centered, draw=black, fill=blue!30]
% \tikzstyle{process} = [rectangle, minimum width=3cm, minimum height=1cm, text centered, draw=black, fill=orange!30]
% \tikzstyle{decision} = [diamond, minimum width=3cm, minimum height=1cm, text centered, draw=black, fill=green!30]
\tikzstyle{arrow} = [thick,->,>=stealth]

%%
%% @stud
%%
%%	FONT SELECTION: Schriftarten und Schriftfamilie
%%%%%%%%%%%%%
%% SCHRIFTART
%%%%%%%%%%%%%
% 0) without decomment: normal font families
% ...
% 1) Latin Modern
%\usepackage{lmodern}
% 2) Times
%\usepackage{mathptmx}
% 3) Helvetica
%\usepackage[scaled=.92]{helvet}
%%%%%%%%%%%%%%%%%%
%%	SCHRIFTFAMILIE
%%%%%%%%%%%%%%%%%%
% ohne Serifen
\renewcommand*{\familydefault}{\sfdefault}
\addtokomafont{disposition}{\sffamily}
%
% mit Serifen
%\renewcommand*{\familydefault}{\rmdefault}
%\addtokomafont{disposition}{\rmfamily}
%
% Typewriter
%\renewcommand*{\familydefault}{\ttdefault}
%\addtokomafont{disposition}{\ttfamily}

%%
%% @stud
%%
%% Uncomment the following lines to support hard URL breaks in bibliography
%\apptocmd{\UrlBreaks}{\do\f\do\m}{}{}
%\setcounter{biburllcpenalty}{9000}% Kleinbuchstaben
%\setcounter{biburlucpenalty}{9000}% Großbuchstaben

%%
%% @stud
%%
%% FOOTNOTES: Count footnotes over chapters
%% \counterwithout{footnote}{chapter}

%	ACRONYMS
\makeatletter
\@ifpackagelater{acronym}{2015/03/20}
{\renewcommand*{\aclabelfont}[1]{\textbf{{\acsfont{#1}}}}}{}
\makeatother

%	LISTINGS
% @stud: ggf. Namen/Text anpassen (englisch)
\renewcommand{\lstlistingname}{Quelltext}
\renewcommand{\lstlistlistingname}{Quelltextverzeichnis}
\lstset{numbers=left,
	numberstyle=\tiny,
	captionpos=b,
	basicstyle=\ttfamily\small}

\definecolor{cGreen}{rgb}{0,0.6,0}
\definecolor{cGray}{rgb}{0.5,0.5,0.5}
\definecolor{cPurple}{rgb}{0.58,0,0.82}
\definecolor{cBackgroundColour}{rgb}{0.95,0.95,0.92}

\lstdefinestyle{CStyle}{
    backgroundcolor=\color{cBackgroundColour},
    commentstyle=\color{cGreen},
    keywordstyle=\color{magenta},
    numberstyle=\tiny\color{cGray},
    stringstyle=\color{cPurple},
    basicstyle=\footnotesize,
    breakatwhitespace=false,
    breaklines=true,
    captionpos=b,
    keepspaces=true,
    numbers=left, % @Note(Val): Ist es schöner mit Line-Numbers links vom Code oder sollten wir die ausstellen mit `numbers=none`?
    numbersep=5pt,
    showspaces=false,
    showstringspaces=false,
    showtabs=false,
    tabsize=2,
    language=C
}

% Code-Listing, that doesn't spill over page-breaks (see: https://tex.stackexchange.com/a/22889)
\lstnewenvironment{UnbrokenCodePage}[1][]%
{
   \noindent
   \minipage[htbp]{\linewidth}
   \vspace{0.5\baselineskip}
   \lstset{#1}}
{\endminipage}

%	ALGORITHMS
% @stud: ggf. Namen/Text anpassen (englisch)
\renewcommand{\listalgorithmname}{Algorithmenverzeichnis}
\floatname{algorithm}{Algorithmus}

%	PAGE HEADER / FOOTER
%	Warning: There are some redefinitions throughout the master.tex-file!  DON'T CHANGE THESE REDEFINITIONS!
\RequirePackage[automark]{scrlayer-scrpage}
%alternatively with separation lines: \RequirePackage[automark,headsepline,footsepline]{scrlayer-scrpage}

\renewcommand{\chaptermarkformat}{}
\RedeclareSectionCommand[beforeskip=0pt]{chapter}
\clearpairofpagestyles

%\ifoot[\rule{0pt}{\ht\strutbox+\dp\strutbox}DHBW Mannheim]{\rule{0pt}{\ht\strutbox+\dp\strutbox}DHBW Mannheim}
\ofoot[\rule{0pt}{\ht\strutbox+\dp\strutbox}\pagemark]{\rule{0pt}{\ht\strutbox+\dp\strutbox}\pagemark}
\ohead{\headmark}

\newcommand{\TitelDerArbeit}[1]{\def\DerTitelDerArbeit{#1}\hypersetup{pdftitle={#1}}}
\newcommand{\AutorDerArbeit}[1]{\def\DerAutorDerArbeit{#1}\hypersetup{pdfauthor={#1}}}
\newcommand{\Kurs}[1]{\def\DieKursbezeichnung{#1}}
\newcommand{\Studiengangsleiter}[1]{\def\DerStudiengangsleiter{#1}}
\newcommand{\WissBetreuer}[1]{\def\DerWissBetreuer{#1}}
\newcommand{\Bearbeitungszeitraum}[1]{\def\DerBearbeitungszeitraum{#1}}
\newcommand{\Abgabedatum}[1]{\def\DasAbgabedatum{#1}}
\newcommand{\Matrikelnummer}[1]{\def\DieMatrikelnummer{#1}}
\newcommand{\Studienrichtung}[1]{\def\DieStudienrichtung{#1}}
\newcommand{\ArtDerArbeit}[1]{\def\DieArtDerArbeit{#1}}
\newcommand{\Literaturverzeichnis}{Literaturverzeichnis}

\newcommand{\settingBibFootnoteCite}{
	\setlength{\bibparsep}{\parskip}		  % Add some space between biblatex entries in the bibliography
	\addbibresource{bibliography.bib}	    % Add file bibliography.bib as biblatex resource
	\DefineBibliographyStrings{ngerman}{andothers = {{et\,al\adddot}},}
}

\newcommand{\setTitlepage}{
	\input{titlepage}
	\pagenumbering{roman} % Römische Seitennummerierung
	\normalfont
}

\newcommand{\initializeText}{
	\clearpage
	\ihead{\chaptername~\thechapter} % Neue Header-Definition
	\pagenumbering{arabic}           % Arabische Seitenzahlen
}

\newcommand{\initializeBibliography}{
	\ihead{}
	\printbibliography[title=\Literaturverzeichnis]
	\cleardoublepage
}

\newcommand{\initializeAppendix}{
	\appendix
  \ihead{}
  \cftaddtitleline{toc}{chapter}{Anhang}{}
}

\makeatletter
\newcommand{\chapterauthor}[1]{%
  {\parindent0pt\vspace*{-25pt}%
  \linespread{1.1}#1%
  \par\nobreak\vspace*{30pt}}
  \@afterheading%
}
\makeatother
