%%%%%%%%%%%%%%%%%%%%%%%%%%%%%%%%%%%%%%%%%%%%%%%%%%%%%%%%%%
%   Autoren des Abschnitts:
%   Olivier Stenzel
%%%%%%%%%%%%%%%%%%%%%%%%%%%%%%%%%%%%%%%%%%%%%%%%%%%%%%%%%%

% !TEX root =  master.tex
\chapter{Einleitung} \label{einleitung}

\nocite{*}

Im Rahmen dieser Studienarbeit wird ein selbstspielendes Klavier konzipiert und gebaut.
Klaviermusik findet in zahlreichen Umgebungen ihren Einsatz: von der Untermalung in Hotel-Lobbys durch Bar-Pianist:innen bis hin zu privaten Haushalten,
wo sie zur Atmosphäre und Unterhaltung beiträgt.
Vor diesem Hintergrund könnte ein selbstspielendes Klavier, speziell in Bereichen, wo traditionellerweise Live-Musiker:innen engagiert werden,
einen signifikanten Mehrwert darstellen.

Obwohl die Technologie der selbstspielenden Klaviere bereits etabliert ist, bleiben die Kosten für solche Instrumente mit Preisen,
die leicht 10.000 Euro überschreiten können, eine beträchtliche Hürde \cite*{YamahaU1}.
Diese finanzielle Barriere limitiert den Zugang zu dieser Technologie für eine breite Nutzerbasis erheblich.
Angesichts dieser Preisgestaltung ist eine wesentliche Motivation dieser Arbeit die Entwicklung einer deutlich preisgünsigeren Alternative.
Das Ziel ist es, ein Instrument zu konzipieren, das die wesentlichen Funktionalitäten seines professionellen Pendants beibehält,
jedoch zu einem Bruchteil der Kosten realisiert werden kann.
Durch dieses Vorhaben sollen selbstspielende Klaviere einem breiteren Publikum zugänglich gemacht werden.



