%%%%%%%%%%%%%%%%%%%%%%%%%%%%%%%%%%%%%%%%%%%%%%%%%%%%%%%%%%
%   Autoren des Abschnitts:
%   Olivier Stenzel
%%%%%%%%%%%%%%%%%%%%%%%%%%%%%%%%%%%%%%%%%%%%%%%%%%%%%%%%%%

% !TEX root =  master.tex
\chapter{Einleitung} \label{einleitung}
\chapterauthor{Olivier Stenzel}

\nocite{*}

Im Rahmen dieser Studienarbeit wird ein selbstspielendes Klavier konzipiert und gebaut.
Klaviermusik findet in zahlreichen Umgebungen ihren Einsatz: von der Untermalung in Hotel-Lobbys durch Bar-Pianist:innen bis hin zu privaten Haushalten,
wo sie zur Atmosphäre und Unterhaltung beiträgt.
Vor diesem Hintergrund könnte ein selbstspielendes Klavier, speziell in Bereichen, wo traditionellerweise Live-Musiker:innen engagiert werden,
einen signifikanten Mehrwert darstellen.

Obwohl die Technologie der selbstspielenden Klaviere bereits etabliert ist, bleiben die Kosten für solche Instrumente bei Preisen,
die leicht 10.000 Euro überschreiten können. \cite{YamahaU1}.


Angesichts dieser Preisgestaltung ist eine wesentliche Motivation dieser Arbeit die Entwicklung einer Alternative.
Das Ziel ist es, ein Instrument zu konzipieren, das die wesentlichen die Funktionalitäten seines professionellen Pendants beibehält,
jedoch zu einem Bruchteil der Kosten und des Know-Hows großer Hersteller realisiert werden kann.


