% !TEX root =  master.tex
\chapter{Einleitung} \label{einleitung}

\nocite{*}

Die vorliegende Studienarbeit befasst sich mit der Konstruktion eines selbstspielenden Klaviers.
Klaviermusik wird in vielen Settings gespielt.
In Szenarien wie Bar-Pianisten in Hotel-Lobbys könnte ein selbstspielendes
Klavier als innovatives Instrument daher einen Mehrwert bieten. \newline
Selbstspielende Klaviere gibt es schon seit längerer Zeit. Diese sind allerdings in der Preisklasse um die 7000
Euro, womit viele sich dieses Instrument nicht leisten können.
Eine Motivation dieser Arbeit liegt daher darin, eine kostengünstigere Alternative zu entwickeln, die die
Funktionalitäten der Professionellen Instrumente beibehält.


\section{Ziel} \label{einleitung-ziel}

Zielsetzung der Arbeit besteht darin, ein selbstspieldendes Klavier zu entwickeln, welches gleichzeitig
noch normal bespielt werden kann. Hierbei wird ein Arduino eingesetzt, um Hardware anzusteuern, die in der
Lage ist, die Klaviertasten zu betätigen. Die Steuerung dieses selbstspielenden Klaviers erfolgt über eine
Desktop-Anwendung, mittels derer Nutzer:innen die Auswahl von Musikstücken vornehmen können. Darüber hinaus
soll die Möglichkeit bestehen, dem Katalog weitere Musikstücke in Form von MIDI-Dateien hinzuzufügen. Ferner
sollen potenzielle Erweiterungen in Betracht gezogen werden, wie beispielsweise die Steuerung des
selbstspielenden Klaviers durch Bilder von Notenblättern.


\section{Vorgehensweise} \label{einleitung-vorgehen}
Die Vorgehensweise des Projekts gliedert sich in mehrere Schlüsselschritte. Zunächst erfolgt die
Konzeptualisierung des selbstspielenden Klaviers, wobei die spezifischen Anforderungen und Funktionalitäten
definiert werden. Hierbei liegt ein besonderer Fokus auf der Identifikation der Schnittstellen zwischen der
Hardware und Software, wobei die Arduino-Plattform als zentrales Steuerungselement berücksichtigt wird. \newline

Im Anschluss erfolgt die Hardware-Implementierung mithilfe des Arduinos. Dies beinhaltet die sorgfältige
Auswahl geeigneter Hardwarekomponenten, die in der Lage sind, die Klaviertasten präzise anzusteuern.
Die Programmierung des Arduinos erfolgt mit dem Ziel, eine nahtlose Integration in das Gesamtsystem zu
gewährleisten. \newline

Die nachfolgende Etappe konzentriert sich auf die Entwicklung einer
Desktop-Anwendung, die als Schnittstelle für die Steuerung des selbstspielenden Klaviers dient. Hierbei wird
besonderes Augenmerk auf die Benutzerfreundlichkeit gelegt, und die Anwendung ermöglicht Nutzer:innen die
Auswahl und Wiedergabe von Musikstücken aus einem vordefinierten Katalog. \newline

Ein weiterer Schwerpunkt liegt auf der Integration von MIDI-Dateien, um dem Katalog kontinuierlich weitere
Musikstücke hinzufügen zu können. Dieser Prozess beinhaltet die Entwicklung einer Schnittstelle, die eine
unkomplizierte Integration neuer Musikstücke in die bestehende Datenbank ermöglicht.

