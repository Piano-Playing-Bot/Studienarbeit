%%%%%%%%%%%%%%%%%%%%%%%%%%%%%%%%%%%%%%%%%%%%%%%%%%%%%%%%%%
%   Autoren des Abschnitts:
%   Olivier Stenzel
%%%%%%%%%%%%%%%%%%%%%%%%%%%%%%%%%%%%%%%%%%%%%%%%%%%%%%%%%%

% !TEX root =  master.tex
\chapter{Einleitung} \label{einleitung}
\chapterauthor{Olivier Stenzel, Val Richter}

Diese Studienarbeit befasst sich mit der Frage, welchen Technik- und Kostenaufwand die Eigenentwicklung eines selbstspielenden Klaviers mit sich bringt.
Professionell hergestellte, selbstspielende Klaviere existieren zwar bereits,
jedoch nur mit Preisen, die leicht über 10.000 Euro gehen können \cite{YamahaU1}, was für Privatpersonen oft nicht bezahlbar ist.
Hier soll dagegen versucht werden, mit einem deutlich geringerem Budget von maximal 2000\euro{}, ein Klavier zu entwicklen, welches die selben Funktionen bereitstellt, wie das der großen Hersteller.


Der Fokus dieser Arbeit liegt dabei in der Konzeption dieses Instruments.
Über die Konzeption hinaus soll jedoch auch ein Prototyp umgesetzt werden.
Der Prototyp soll hierbei nicht nur über die Technik zum automatischen Anspielen des Klaviers verfügen, sondern auch über eine Computer-Anwendung, von der aus das Piano angesteuert werden kann.
Damit soll erkundet werden, wie funktional die Konzeption in der Praxis ist und mit wie viel Aufwand die Implementierung verbunden ist.
Die erstellte Software des Prototypen soll dabei möglichst portabel sein, um für Eigenentwicklungen Anderer mit möglichst wenigen Änderungen verwendbar zu sein.

Ähnlich wie die professionellen, selbstspielenden Klaviere, soll der Prototyp dieser Arbeit sowohl manuell von Musiker:innen als auch automatisiert durch einen Computer bespielt werden können.
Das automatisierte Anspielen des Pianos sollte über die Eingabe breit verwendeter Dateien möglich sein.
Logische Kandidaten wären hier \enquote{\ac{MIDI}}-Dateien, aber auch generelle Audio-Dateien oder eingescannte bzw. digtal vorliegende Notensheets wären hier naheliegend.

Im folgenden Kapitel werden die Anforderungen an den zu entwickelnden Prototypen genauer herausgearbeitet. \newline
Die darauf folgenden Kapitel \enquote{\nameref{konzeptionHW}} und \enquote{\nameref{umsetzungHW}} behandeln die Hardware, bevor dann in den Kapiteln \enquote{\nameref{vorgehenSW}} und \enquote{\nameref{umsetzungSW}} der Software-Teil der Arbeit betrachtet wird.
Dabei wird jeweils zuerst die Planung durchgeführt und dann auf die Umsetzung des Prototypen eingegangen.

In Kapitel \ref{ergebnisse} wird dann der erstellte Prototyp dann zusammengefasst.
Außerdem werden Tests der Hardware und Software des Prototypen durcheführt und Limitationen des Projekts aufgezeigt. \newline
Abschließend zieht Kapitel \ref{fazit} dann ein Fazit über diese Arbeit und fasst zusammen, welche der gestellten Anforungen erfüllt werden konnten.
