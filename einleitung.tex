% !TEX root =  master.tex
\chapter{Einleitung} \label{einleitung}

\nocite{*}

In dieser Studienarbeit wird ein selbstspielendes Klavier konzipiert und gebaut.
Klaviermusik findet in zahlreichen Umgebungen ihren Einsatz, von der Untermalung in Hotel-Lobbys durch Bar-Pianist:innen bis hin zu privaten Haushalten,
wo sie zur Atmosphäre und Unterhaltung beiträgt.
Vor diesem Hintergrund könnte ein selbstspielendes Klavier, speziell in Bereichen, wo traditionellerweise Live-Musiker:innen engagiert werden,
als innovatives Element einen signifikanten Mehrwert darstellen.
% @Note(Val): Wollen wir es wirklich ein "innovatives" Element nennen, wenn andere sowas schon gemacht haben?

Obwohl die Technologie der selbstspielenden Klaviere bereits etabliert ist, bleiben die Kosten für solche Instrumente mit Preisen,
die leicht 10.000 Euro überschreiten können, eine beträchtliche Hürde \cite*{YamahaU1}.
Diese finanzielle Barriere limitiert den Zugang zu dieser Technologie für eine breite Nutzerbasis erheblich.
Angesichts dieser Preisgestaltung ist eine wesentliche Motivation dieser Arbeit die Entwicklung einer erschwinglicheren Alternative. % @Note(Val): "erschwinglicheren" klingt sehr umgangssprachlich
Das Ziel ist es, ein Instrument zu konzipieren, das die wesentlichen Funktionalitäten seines professionellen Pendants beibehält,
jedoch zu einem Bruchteil der Kosten realisiert werden kann.
Durch dieses Vorhaben soll die faszinierende Welt der selbstspielenden Klaviere einem breiteren Publikum zugänglich gemacht und die musikalische Landschaft bereichert werden.
% @Note(Val): v.a. der letzte Satz hier ist vielleicht ein wenig zu breit aufgretragen



