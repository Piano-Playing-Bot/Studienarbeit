

\chapter{Ergebnisse} \label{ergebnisse}
\chapterauthor{Jakob Kautz, Olivier Stenzel, Val Richter}
\nocite{*}

Nachdem sowohl Hardware als auch Software konzipiert und zu großen Teilen umgestzt wurden, soll nun der entstandene Prototyp abgeschlossen werden.
Dafür wird zuerst in Abschnitt \ref{ergebnisse-prototyp} ein Überblick über den gesamten Prototypen gegeben.
In Abschnitt \ref{ergebnisse-tests} werden dann mehrere Tests und Messungen am Prototypen ausgeführt.
Die Ergebnisse der Testungen zeigen einerseits die Fähigkeiten und Limitationen des Prototypen auf, können zu Teilen aber auch verwendet werden, um bestehende Probleme der Umsetzung zu lösen.
Zuletzt wird in Abschnitt \ref{ergebnisse-limitationen} noch ein Überblick darüber gegeben, was beim Prototypen noch fehlt, um ihn zu einem vollständigen Produkt zu bringen. % @Note(Val): Der Satz klingt komisch, bitte überarbeiten
Hier werden nicht nur Limitationen der Umsetzung, sondern auch der Konzeption betrachtet, welche in Fortführungen oder Replikationen dieser Arbeit beachtet werden sollten.



\section{Prototyp} \label{ergebnisse-prototyp}

% @TODO(Val): Prototyp vollständig beschreiben, also auch Hardware
- Was wurde umgesetzt \newline
- UX und UI Design (kurz) \newline
- Messungen: \newline
	- Kommunikationsgeschwindigkeit \newline
	- FPS beim Arduino \newline
	- FPS bei UI \newline
	- FPS bei Comm-Thread der UI \newline
	- Speicherverbrauch der UI \newline
	- Code-Komplexität (ist das interessant?) \newline
	- Lines of Code (i guess?) \newline
- Limitationen \newline
	- nur MIDI, kein Image→/Ton→PIDI \newline
	- UI Design ist verbesserungsfähig \newline
	- SPPP garantiert keine Datenkonsistenz \newline
	- SPPP ist nicht auf mehrere Nutzer:innen ausgelegt \newline
	- Mögliche Fehler in SPPP (siehe Github issue) \newline
	- siehe Github issues \newline
	- Es wurde nicht getestet, ob Flash-Speicher (der oftmals ausreichend groß ist) zum Speichern eines gesamten Songs ausgereicht hätte und welche Performanz-Probleme das mit sich gebracht hätte. Dann wäre Streaming unnötig. Wurde aber nicht getestet, ob das überhaupt eine viable Alternative wäre \newline



\section{Tests} \label{ergebnisse-tests}
\chapterauthor{Olivier Stenzel}

% @Note(Val): Der zweite Satz hat nichts mit dem ersten zu tun, auch wenn es so aussieht als würden sie es tun.
% Wenn ich das richtig verstehe, sagst du, dass wir wir die maximale Geschwindigkeit bestimmen müssen, um die Software nicht schneller werden zu lassen.
% Aber wenn die Software zu schnell ist, ist das ja gar kein Problem.
% Es gibt der Software nur das Wissen, dass sie langsamer sein darf.
% Stattdessen ist das Wissen über die Maximalgeschwindigkeit in sich selbst einfach interessant, da es zeigt wie gut unser kosten-effizienter Prototyp ist
% @Note(Val): In kurz: Diese Einleitung muss neu geschrieben werden
Um eine Überbelastung der Komponenten und eine korrekte Funktionalität zu gewährleisten, müssen Software-seitig Begrenzungen eingeführt werden.
Um zum Beispiel die maximale Geschwindigkeit bestimmen zu können, mit der ein Stück gespielt werden kann, muss die maximale Frequenz eines physischen Tastendrucks berüclsichtigt werden.
Es wurden folgende Tests durchgeführt:

\subsubsection{T1 Max. Frequenz eines Tastendrucks:}

% @Note(Val): Dieser Test gehört eher in die Software-Umsetzung
% Alternativ, falls wir den Test hier lassen wollen, müsste ich hier noch ein-zwei Paragraphen dazu schreiben, wie das Problem in der Software gelöst wird
Erklärung: Software-seitig kann sehr genau angegeben werden, wann welche Taste gespielt wird (siehe Kapitel \enquote{\nameref{vorgehenSW-PIDI}}).
Hierbei kann auch die Frequenz, mit der dieselbe Taste hintereinander angespielt, wird bis auf mehrere hundert Hertz erhöht werden.
Da der Hubmagnet, bevor er einen erneuten Tastendruck initiieren kann, wieder in seinen Ausgangszustand zurückkehren muss, ist dieser der limitierende Faktor.
Im Endprodukt soll sichergestellt werden, dass alle Töne des Stücks auch wirklich erklingen. % @Note(Val): Dieser letzte Satz erklärt, warum es hier ein Problem geben kann. Dieser Satz sollte eher an den Anfang, um das Problem aufzuzeigen. Sonst argumentieren wir wieder falsch herum

Ablauf: Schrittweise wird software-seitig die Frequenz des Anspielens einer Taste erhöht.
Es werden mehrere Tasten für diesen Test verwendet, wobei der Fokus, auf Grund des höchsten Mechanik-Gewichts, auf der A0 Taste liegt. % @Note(Val): Was wird mit "Mechanik-Gewicht" gemeint? Vorher wird nie erwähnt, dass manche Tasten ein höheres Gewicht haben. Das wäre vielleicht wichtig vorher schon zu erwähnen
Während des Spielens wird eine Slow-Motion-Aufnahme des Hammers und der Klaviersaite angefertigt ud im Anschluss analysiert.
Wird die Saite angeschlagen, kann die Frequenz im nächsten Schritt erhöht werden.
Sollte sie nicht angeschlagen werden, ist die Limitation gefunden und kann in den Code aufgenommen werden.

Ergebnis: Ein Tasten-Magnet-Zyklus dauert durchschnittlich ca. 30ms und maximal ca. 40ms.
% @TODO(Val): Hier sollte vermutlich die Lösung im Code kurz erklärt werden

\subsubsection{T2 Min. Lautstärke:}

Erklärung: Um den am leisesten spielbaren Ton zu ermitteln, muss die minimale Spannung ermittelt werden, mit der eine Taste betätigt werden kann.
Wie in Kapitel \enquote{\nameref{subsec:aktuator}} erklärt, benötigt ein Hubmagnet eine entsprechende Ansprechspannung, um den Anker zu bewegen.
Da diese Spannung im Datenblatt des Hubmagneten nicht ersichtlich ist und das Tastengewicht sowieso mit einbezogen werden muss, muss diese Mindestspannung beim Klavier experimentell ermittelt werden.

Ablauf: Software-seitig wird der Wert, der die Intensität des Anschlags beschreibt (siehe Kapitel \enquote{\nameref{vorgehenSW-MC}}) schrittweise soweit reduziert, bis die Taste keinen Ton mehr spielt.
Dies wird an allen (mit Fokus auf den tieferen) Tasten durchgeführt, da sich die einzelnen Tasten in ihrem Ansprechverhalten teils deutlich unterscheiden. % @Note(Val): Erneut - warum wird erst jetzt erwähnt, dass sich die Tasten so stark unterscheiden? Wäre das bei der Konzeption oder dem Anbringen der Aktuatoren an das Piano nicht interessant zu wissen?
Anschließend wird der Größte der einzelnen Minima in den Code mit aufgenommen.

Ergebnis: Das leisteste Anspielen der Tasten ist mit einer Spannung von 17,3V möglich.
Dieses Ergebnis fließt in die Lautstärkeregelung der Software ein.
% @TODO(Val): Hier sollte vermutlich die Lösung im Code kurz erklärt werden


\paragraph{T3 Min. Akkuratheit:}











\section{Limitationen \& Erweiterungsmöglichkeiten} \label{ergebnisse-limitationen}
\chapterauthor{Jakob Kautz, Olivier Stenzel}

Im Laufe der Konzeption traten mehrere Herausforderungen auf, welche aus Zeit- und Konsten- Gründen nicht weiter
behandelt wurden.
\paragraph{Anzahl der Aktuatoren}
Wie bereits erklärt, wurden insgesamt 88 Aktuatoren verbaut, womit jede Taste angespielt werden kann. Trotz der Anzahl an
Hubmagneten, werden maximal 10 Tasten glechzeitig angespielt. Dies liegt an der Stromversorgung. Ein Hubmagnet benötigt
eine Stromversorgung von etwa 0.4 Ampere, für 10 Aktuatoren sind das also 4 Ampere. Das Netzteil welches wir verwenden, ist auf
6(?) Ampere ausgelegt, wobei wir mit einem Netzteil getestet haben, welches 2,8 Ampere unterstützt. Es gibt Netzteile die
einen höheren Stromföuss ermöglichen, allerdings sind diese um weiten teurer als das Netzteil, für welches wir uns entschieden haben.
Es wäre auch möglich, ein zweites Netzteil parallel zu Schalten, wodurch die Kosten nicht dramatisch gestiegen wären.
Wir haben hier allerdings keinen Mehrwert mehr gesehen. Die Logik und Ansteuerung des Klaviers bleibt die gleiche, weswegen
wir bei einem Netzteil mit einem maximalen Stromfluss von 6 Ampere (und 24V Leistung) verblieben sind.

\paragraph{Pedalansteuerung}
Ein Klavier hat normalerweise zwei oder drei Pedale, welche die Dynamik und den Klang des Klaviers beeinflussen.
Diese sind schwerer anzuspielen als die Tasten und benötigen somit Leistungsfähigere Aktuatoren als wir für die Tasten nutzen.
Es gab eine Überlegung diese Aktuatoren zu besorgen, da die Pedale klangtechnisch Mehrwert bringen.
Außerdem hätten wir uns noch Gedanken bezüglich der Schaltug und des Signals machen können.
Einerseits hätte ein weiteres Schieberegister genutzt werden können, wobei das Signal angepasst wird das die letzten 6 Ausgänge nie ein Signal bekommen.
ANdererseits hätten die Pedale mit weiteren \ac{PWM} Ports des Arduinos verbunden werden können und die Signale unabhängig von den
Schieberegistern erhalten können.
Im Prinzip wäre die Idee hier allerdings wieder dieselbe wie beim Rest des Projektes gewesen,
weswegen wir Kosten an den Aktuatoren und der extra Stromversorgung für die Pedale gespart haben und diese nicht ansteuern.

\paragraph{Wärmeabfuhr}
Da die Hitzeentwicklung der Hubmagnete mit dem Dauer der Nutzung stark ansteigt, ist die Holzplatte bezogen auf die Wärmeabfuhr nicht geeignet.
Besser würde sich hier eine Metallplatte anbieten, auf der die Aktuatoren verschraubt werden.