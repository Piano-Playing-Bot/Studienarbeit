%%%%%%%%%%%%%%%%%%%%%%%%%%%%%%%%%%%%%%%%%%%%%%%%%%%%%%%%%%
%   Autoren des Abschnitts:
%   Val Richter
%%%%%%%%%%%%%%%%%%%%%%%%%%%%%%%%%%%%%%%%%%%%%%%%%%%%%%%%%%

% !TEX root =  master.tex
\chapter{Ergebnisse Software} \label{ergebnisseSW}
\chapterauthor{Val Richter}

\nocite{*}

- Was wurde umgesetzt \newline
- UX und UI Design (kurz) \newline
- Messungen: \newline
	- Kommunikationsgeschwindigkeit \newline
	- FPS beim Arduino \newline
	- FPS bei UI \newline
	- FPS bei Comm-Thread der UI \newline
	- Speicherverbrauch der UI \newline
	- Code-Komplexität (ist das interessant?) \newline
	- Lines of Code (i guess?) \newline
- Limitationen \newline
	- nur MIDI, kein Image→/Ton→PIDI \newline
	- UI Design ist verbesserungsfähig \newline
	- SPPP garantiert keine Datenkonsistenz \newline
	- SPPP ist nicht auf mehrere Nutzer:innen ausgelegt \newline
	- Mögliche Fehler in SPPP (siehe Github issue) \newline
	- siehe Github issues \newline
	- Es wurde nicht getestet, ob Flash-Speicher (der oftmals ausreichend groß ist) zum Speichern eines gesamten Songs ausgereicht hätte und welche Performanz-Probleme das mit sich gebracht hätte. Dann wäre Streaming unnötig. Wurde aber nicht getestet, ob das überhaupt eine viable Alternative wäre \newline