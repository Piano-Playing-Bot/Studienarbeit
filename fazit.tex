%%%%%%%%%%%%%%%%%%%%%%%%%%%%%%%%%%%%%%%%%%%%%%%%%%%%%%%%%%
%   Autoren des Abschnitts:
%   Jakob Kautz
%%%%%%%%%%%%%%%%%%%%%%%%%%%%%%%%%%%%%%%%%%%%%%%%%%%%%%%%%%

% !TEX root =  master.tex
\chapter{Zusammenfassung} \label{fazit}
\chapterauthor{Jakob Kautz}

\section{Fazit}
Die Arbeit umfasst eine erfolgreiche Konzeption und Entwicklung des selbstspielenden Pianos, wobei ein Gr0ßteil der in der Zielstellung
definierten Anforderungen mit mittlerer odre höherer Priorität erfüllt werden konnten. In diesem Kapitel soll Rückblickend der Erfolg
des Projektes insbesondere in Bezug auf die Ausgangsfrage evaluiert werden.
\newline
Vorab die Betrachtung der in Kapitel \ref{Zielstellung} definierten Anforderungen:
\begin{table}[htbp]
    \centering
    \begin{tabular}{|m{1cm}|m{4cm}|m{8cm}|}
        \theadstart{ID} & \theadcol{Name} & \theadcol{Status}  \\
        \hline
        A1 & Flexibilität & Erfüllt: Die Aktuatoren werden von manuellen Tastendrücken nicht beeinflusst \\
        \hline
        A2 & Benutzerinterface & Erfüllt \\
        \hline
        A3 & Budget & Erfüllt, liegt insgesamt bei 1753,78\euro{} \\
        \hline
        A4 & Responsivität & Erfüllt \\
        \hline
        A5 & Performanz & Erfüllt: liegt durchschnittlich bei 30ms \\
        \hline
        A6 & Spielbarkeit & Teils Erfüllt, da nicht alle Hubmagnete angeschlossen wurden. \\
        \hline
        A7 & Tastenbetätigung & Teilweise erfüllt: Die verbundenen Tasten können erfolgreich betätigt werden, wobei die Anzahl der Tasten auf
        8 reduziert werden musste \\
        \hline
        A8 & Anpassbarkeit & Erfüllt: Die Elektronik und Mechanik sind nicht Klavier-abhängig, wobei der Fußraum zum Befästigen der Aktuatoren
        gegeben sein muss oder das Ansteuerungskonzept geändert werden müsste\\
        \hline
        A9 & Musikstück-Auswahl & Erfüllt \\
        \hline
        A10 & Wiedergabe-Kontrolle & Erfüllt \\
        \hline
        A11 & Navigation & Erfüllt \\
        \hline
        A12 & MIDI-Integration & Erfüllt \\
        \hline
        A13 & Musikverwaltung & Nicht Erfüllt \\
    \end{tabular}
    \caption{Ergebnisse der Anforderungen}
    \label{table:anorderungen-ergebnis}
\end{table}

Nun zur Betrachtung der Ausgangsfrage: Kann ein selbstgebautes \enquote{Player-Piano} mit den
auf dem Markt erhältlichen mithalten und bietet der Selbstbau somit eine Kostengünstigeren aber dennoch Funktionsfähige
Alternative? \newline
Hier sind mehrere Punkte von Bedeutung.
\begin{enumerate}
    \item Funktionalität: dieser Punkt wurde teils schon in der Tabelle \ref{table:anorderungen-ergebnis} beantwortet.
    Letztendlich ist es möglich, ein Funktionelles selbst-spielendes Klavier selber zu Entwickeln. Dieses Projekt  kann im Bezug
    auf Qualität und Ästhetik allerdings noch nicht mit professionellen Klavieren mithalten.
    Was allerdings die Funktionalitäten, wie zum Beispiel die kostenfreie Erweiterung des Musikkatalogs, betrifft, die das Klavier bietet,
    sind diese beim Eigenbau deutlich flexibler erweiterbar.
    Dafür ist allerdings, zumindest bei diesem Prototypen, die dauerhafte Funktionsfähigkeit noch nicht getestet und somit nicht
    garantiert.
    \item Kostenaufwand: Eine Anforderung für die erfolgreiche Beantwortung der Fragestellung ist, dass der Eigenbau Kostengünstiger
    sein soll als die professionell Hergestellten \enquote{Player-Pianos}. Dies ist der Fall. Die Kosten dieses Projektes
    beliefen sich wie in Tabelle \ref{table:kosten} aufgeführt auf 1753,78\euro{}. Diese Kosten können noch stark reduziert
    werden, wenn die Komponenten über andere Anbieter bestellt werden, was in diesem Projekt aufgrund von Firmenregulationen
    nicht möglich war. Trotz dessen liegen die Kosten für die Entwicklung unter 2000\euro{}, was ein großer Unterschied
    zu den auf dem Markt erhältlichen Modellen ist (vgl. Kapitel \ref{einleitung}).
    \item Zeitaufwand: Es stellt sich natürlich noch die Frage, ob sich der Aufwand für das \enquote{Player-Piano} tatsächlich
    lohnt. Wie in Kapitel \ref{ErgebnissePrototyp} aufgeführt ist, beläuft sich der Zeitaufwand - rein für die Montage und Schaltung
    nach der Planung - bei der hälfte der Tasten auf Rund 44 Stunden. Das sind etwa 80 Stunden reine Fleißarbeit.
    Dazu kommt der Zeitaufwand für Reparaturen und Tests. Außerdem sind diese 80 Stunden nur ein Bruchteil dessen, was
    für die Konzeption des Klaviers und der Entwicklung der Software gebraucht wurde. Die Evaluation ob der Zeitaufwand
    sich lohnt, hängt daher sehr stark davon ab, wie viel Freude diese Arbeiten den Personen die dieses Projekt
    umsetzen bringen. Es steht fest, dass das Projekt, ein Player-Piano selber zu bauen, besonders für jene sinnvoll ist,
    die gerne ihr Wissen und ihre Erfahrungen sowohl im Informatik -als auch im Elekronik-Bereich erweitern wollen.
\end{enumerate}

Die Ergebnisse zeigen, dass selbstgebaute Varianten im Vergleich zu käuflichen Produkten erhebliche Kosteneinsparungen
ermöglichen, jedoch mit einem hohen Zeitaufwand und komplexeren Anpassungsmöglichkeiten einhergehen.

Für diese spezifische Studienarbeit gilt insgesamt, dass im Laufe des Projekts viele Erkenntnisse gewonnen werden konnten.
Im Bereich der Hardware umfassen diese ein vertieftes Verständnis für Elektronik, zum Beispiel der Bedeutung der Kabeldurchmesser
und dessen Bedeutung für den maximal fließbaren Strom, der Stromversorgungsanforderungen.
Positiv zu vermerken ist die erfolgreiche Funktionalität des Schaltplans nach einigen Anpassungen sowie die hilfreiche
Nutzung von Tinkercad (Tinkercad.com) als Simulationswerkzeug.
Herausforderungen ergaben sich vor allem beim Testen der Soft -und Hardware.
Da in diesem Projekt das Testen eines Teils sehr stark vom Funktionieren des anderen abhängig war,
behinderten sie sich anfangs stark.
Zusätzlich gab es anfangs ungeplante Verzögerungen, durch die hohen bürokratischen Hürden bezüglich der Teilebestellungen.
Die Erfahrung aus dem Projekt bietet wertvolle Erkenntnisse im Projektmanagement, in der Recherche und im Fachwissen.
Es ist jedoch zu beachten, dass der Bau eines selbstspielenden Klaviers mit erheblichem Arbeitsaufwand und nicht zu
vernachlässigen Kosten verbunden ist.

\section{Ausblick}

Für zukünftige Entwicklungen bieten sich zahlreiche Möglichkeiten sowohl im Software- als auch im Hardwarebereich des Projekts an.
Auf der Hardwareseite könnten ein verbessertes Sicherheitskonzept für Elektronik, die Fertigstellung der Schaltungen für die
restlichen Aktuatoren und die Untersuchung von Schalldämmungstechniken in Betracht gezogen werden.
In Bezug auf Softwareverbesserungen könnte die Unterstützung weiterer Dateiformate wie PDF neben MIDI erwogen werden.
Es ist zu beachten, dass Projekte dieser Art selten abgeschlossen sind, da ständig neue Funktionen hinzugefügt und
Verbesserungen vorgenommen werden können.

