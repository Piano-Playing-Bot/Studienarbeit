%%%%%%%%%%%%%%%%%%%%%%%%%%%%%%%%%%%%%%%%%%%%%%%%%%%%%%%%%%
%   Autoren des Abschnitts:
%   ???
%%%%%%%%%%%%%%%%%%%%%%%%%%%%%%%%%%%%%%%%%%%%%%%%%%%%%%%%%%

% !TEX root =  master.tex
\chapter{Zusammenfassung} \label{fazit}
\chapterauthor{Jakob Kautz}
\nocite{*}
\section{Fazit}
Die Arbeit umfasst eine erfolgreiche Konzeption und Entwicklung des selbstspielenden Pianos, wobei alle in der Zielstellung
definierten Features erfüllt werden konnten: \newline

\begin{tabular}{| m{4cm} | m{8cm} |}
    \hline
    \textbf{Anforderung} &  \textbf{Status}  \\
    \hline
    Flexibilität & Erfüllt: Die Aktuatoren werden von manuellen Tastendrücken nicht beeinflusst \\
    \hline
    Benutzerinterface & Erfüllt \\
    \hline
    Responsivität & Erfüllt \\
    \hline
    Tastenbetätigung & Erfüllt: Die verbundenen Tasten können erfolgreich betätigt werden, wobei die Anzahl der Tasten auf
    40 reduziert wurde\\
    \hline
    Anpassbarkeit & Erfüllt: Die Elektronik und Mechanik sind nicht Klavier-abhängig, wobei der Fußraum zum befästigen der Aktuatoren
    gegeben sein muss oder das Ansteuerungskonzept geändert werden müsste\\
    \hline
    Musikstück-Auswahl & Erfüllt \\
    \hline
    Wiedergabe-Kontrolle & Erfüllt \\
    \hline
    Navigation & Erfüllt \\
    \hline
    MIDI-Integration & Erfüllt \\
    \hline
\end{tabular} \newline
%Note(Jay) help this sounds
Insgesamt konnten im laufe des Projektes viele Erkenntnisse getroffen werden.
Im Bereich der Hardware umfassen diese ein vertieftes Verständnis für Elektronik, zum Beispiel der Bedeutung der
Kabeldicke, der Stromversorgungsanforderungen und der Notwendigkeit einer gründlichen Vorplanung.
Letztendlich gab es viele Dinge die gut liefen, wie auch welche, die Nachträglich anders angegangen werden sollten.
Positiv zu vermerken ist die erfolgreiche Funktionalität des Schaltplans nach einigen Anpassungen sowie die hilfreiche
Nutzung von Tinkercad als Simulationswerkzeug.
Herausforderungen ergaben sich aus dem Zeitmanagement, dem erstmaligen Testen der Schaltung und
der Sicherheit im Umgang mit Elektronik, insbesondere aufgrund des begrenzten Vorwissens der Teammitglieder. Die
Projektgröße und die höheren als erwarteten Kosten, insbesondere durch Preisunterschiede bei den Aktuatoren zwischen
verschiedenen Bezugsquellen, stellten zusätzliche Herausforderungen dar.
Die Erfahrung aus dem Projekt bietet wertvolle Erkenntnisse im Projektmanagement, in der Recherche und im
Fachwissen. Es ist jedoch zu beachten, dass der Bau eines selbstspielenden Klaviers mit erheblichem Arbeitsaufwand und
unvorhersehbaren Kosten verbunden ist.

Im größeren Kontext betrachtet, wurde das Projekt initiiert, um die eigene Umsetzung selbstspielender Klaviere
im Gegensatz zu fertig gekauften Player-Pianos zu untersuchen.
Die Ergebnisse zeigen, dass selbstgebaute Varianten im Vergleich zu käuflichen Produkten erhebliche Kosteneinsparungen
ermöglichen, jedoch mit einem erheblichen Zeitaufwand und komplexeren Anpassungsmöglichkeiten einhergehen. Gekaufte Versionen
bieten möglicherweise eine höhere Qualität und eine elegantere Ästhetik, erfordern jedoch erhebliche finanzielle
Investitionen.
Was Features betrifft, sind diese beim selbst bauen leichter anzupassen.
Dafür ist allerdings die Funktionsfähigkeit und  Mindesthaltbarkeit nicht garantiert. \newline
Insgesamt steht fest, dass das Projekt ein Player-Piano selber zu bauen für die Menschen sinnvoll ist, die
gerne ausprobieren und viel Zeit (rund 40h nur für das Löten der hälfte der Schaltung) für solche Projekte aufbringen.
Es kann auf jeden Fall sehr viel Erfahrung im Bereich Projektmanagement, Recherche und natürlich Fachbezogenes Wissen
mitgenommen werden. Vorab sollte allerdings klar sein, dass das Projekt einen hohen Zeitaufwand und - je nach Anbieter der
Hardware - unerwartet Hohe Kosten mit sich bringt.


\section{Ausblick}

Für zukünftige Entwicklungen bieten sich zahlreiche Möglichkeiten sowohl im Software- als auch im Hardwarebereich des
Projekts. Auf der Hardwareseite könnten ein verbessertes Sicherheitskonzept für Elektronik, die Fertigstellung der
Schaltungen für die restlichen Aktuatoren und die Untersuchung von Schalldämmungstechniken in Betracht gezogen werden.
In Bezug auf Softwareverbesserungen könnte die Unterstützung weiterer Dateiformate wie PDF neben MIDI erwogen werden.
Es ist zu beachten, dass Projekte dieser Art selten abgeschlossen sind, da ständig neue Funktionen hinzugefügt und
Verbesserungen vorgenommen werden können.

