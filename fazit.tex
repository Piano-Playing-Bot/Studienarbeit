%%%%%%%%%%%%%%%%%%%%%%%%%%%%%%%%%%%%%%%%%%%%%%%%%%%%%%%%%%
%   Autoren des Abschnitts:
%   Jakob Kautz
%%%%%%%%%%%%%%%%%%%%%%%%%%%%%%%%%%%%%%%%%%%%%%%%%%%%%%%%%%

% !TEX root =  master.tex
\chapter{Zusammenfassung} \label{fazit}
\chapterauthor{Jakob Kautz}

\section{Fazit}
Die Arbeit umfasst eine erfolgreiche Konzeption und Entwicklung des selbstspielenden Pianos, wobei alle in der Zielstellung
definierten Anforderungen mit mittlerer odre höherer Priorität erfüllt werden konnten: \newline

\begin{table}[htbp]
    \centering
    \begin{tabular}{|m{1cm}|m{4cm}|m{8cm}|}
        \theadstart{ID} & \theadcol{Name} & \theadcol{Status}  \\
        \hline
        A1 & Flexibilität & Erfüllt: Die Aktuatoren werden von manuellen Tastendrücken nicht beeinflusst \\
        \hline
        A2 & Benutzerinterface & Erfüllt \\
        \hline
        A3 & Budget & Erfüllt \\
        \hline
        A4 & Responsivität & Erfüllt \\
        \hline
        A5 & Performanz & Erfüllt: liegt durchschnittlich bei 30ms \\
        \hline
        A6 & Spielbarkeit & Teils Erfüllt, da nicht alle Hubmagnete angeschlossen wurden. \\
        \hline
        A7 & Tastenbetätigung & Teilweise erfüllt: Die verbundenen Tasten können erfolgreich betätigt werden, wobei die Anzahl der Tasten auf
        8 reduziert werden musste \\
        \hline
        A8 & Anpassbarkeit & Erfüllt: Die Elektronik und Mechanik sind nicht Klavier-abhängig, wobei der Fußraum zum Befästigen der Aktuatoren
        gegeben sein muss oder das Ansteuerungskonzept geändert werden müsste\\
        \hline
        A9 & Musikstück-Auswahl & Erfüllt \\
        \hline
        A10 & Wiedergabe-Kontrolle & Erfüllt \\
        \hline
        A11 & Navigation & Erfüllt \\
        \hline
        A12 & MIDI-Integration & Erfüllt \\
        \hline
        A13 & Musikverwaltung & Nicht Erfüllt \\
    \end{tabular}
    \caption{Ergebnisse der Anforderungen}
    \label{table:anorderungen-ergebnis}
\end{table}

Insgesamt konnten im Laufe des Projekts viele Erkenntnisse gewonnen werden.
Im Bereich der Hardware umfassen diese ein vertieftes Verständnis für Elektronik, zum Beispiel der Bedeutung der Kabeldurchmesser und dessen Bedeutung für den maximal fließbaren Strom, der Stromversorgungsanforderungen.
Positiv zu vermerken ist die erfolgreiche Funktionalität des Schaltplans nach einigen Anpassungen sowie die hilfreiche Nutzung von Tinkercad (Tinkercad.com) als Simulationswerkzeug.
Herausforderungen ergaben sich vor allem beim Testen der Soft -und Hardware.
Da in diesem Projekt das Testen eines Teils sehr stark vom Funktionieren des anderen abhängig war,
behinderten sie sich anfangs stark.
Zusätzlich gab es anfangs ungeplante Verzögerungen, durch die hohen bürokratischen Hürden bezüglich der Teilebestellungen.
Die Erfahrung aus dem Projekt bietet wertvolle Erkenntnisse im Projektmanagement, in der Recherche und im Fachwissen.
Es ist jedoch zu beachten, dass der Bau eines selbstspielenden Klaviers mit erheblichem Arbeitsaufwand und nicht zu vernachlässigen Kosten verbunden ist.

Im größeren Kontext betrachtet, wurde das Projekt initiiert, um den benötigten Aufwand für die Eigenentwicklung selbstspielender Klaviere im Gegensatz zu fertig gekauften zu untersuchen.
Die Ergebnisse zeigen, dass selbstgebaute Varianten im Vergleich zu käuflichen Produkten erhebliche Kosteneinsparungen ermöglichen, jedoch mit einem hohen Zeitaufwand und komplexeren Anpassungsmöglichkeiten einhergehen.
Auch in Bezug auf Qualität und Ästhetik kann dieses Projekt noch nicht mit professionellen Klavieren mithalten.
Was die Funktionalitäten, wie zum Beispiel die kostenfreie Erweiterung des Musikkatalogs, betrifft, die das Klavier bietet, sind diese beim Eigenbau deutlich flexibler erweiterbar.
Dafür ist allerdings, zumindest bei diesem Prototypen, die dauerhafte Funktionsfähigkeit noch nicht getestet und somit nicht garantiert.

Insgesamt steht fest, dass das Projekt, ein Player-Piano selber zu bauen, besonders für jene sinnvoll ist,
die gerne ihr Wissen und ihre Erfahrungen sowohl im Informatik -als auch im Elekronik-Bereich erweitern wollen.


\section{Ausblick}

Für zukünftige Entwicklungen bieten sich zahlreiche Möglichkeiten sowohl im Software- als auch im Hardwarebereich des Projekts an.
Auf der Hardwareseite könnten ein verbessertes Sicherheitskonzept für Elektronik, die Fertigstellung der Schaltungen für die restlichen Aktuatoren und die Untersuchung von Schalldämmungstechniken in Betracht gezogen werden.
In Bezug auf Softwareverbesserungen könnte die Unterstützung weiterer Dateiformate wie PDF neben MIDI erwogen werden.
Es ist zu beachten, dass Projekte dieser Art selten abgeschlossen sind, da ständig neue Funktionen hinzugefügt und Verbesserungen vorgenommen werden können.

