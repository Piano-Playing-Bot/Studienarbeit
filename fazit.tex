%%%%%%%%%%%%%%%%%%%%%%%%%%%%%%%%%%%%%%%%%%%%%%%%%%%%%%%%%%
%   Autoren des Abschnitts:
%   ???
%%%%%%%%%%%%%%%%%%%%%%%%%%%%%%%%%%%%%%%%%%%%%%%%%%%%%%%%%%

% !TEX root =  master.tex
\chapter{Zusammenfassung} \label{fazit}
\chapterauthor{Jakob Kautz}
\nocite{*}
\section{Fazit}
Die Arbeit umfasst eine erfolgreiche Konzeption und Entwicklung des selbstspielenden Pianos, wobei alle in der Zielstellung
definierten Features erfüllt werden konnten: \newline

% @TODO(Val): Remember to update this when updating the Anforderungen-table in chapter 2
\begin{table}[htbp]
    \centering
    \begin{tabular}{|m{4cm}|m{8cm}|}
        \hline
        \textbf{Anforderung} &  \textbf{Status}  \\
        \hline
        Flexibilität & Erfüllt: Die Aktuatoren werden von manuellen Tastendrücken nicht beeinflusst \\
        \hline
        Benutzerinterface & Erfüllt \\
        \hline
        Responsivität & Erfüllt \\
        \hline
        Spielbarkeit & Teils Erfüllt, da nicht alle Hubmagnete angeschlossen wurden. \\
        \hline
        Tastenbetätigung & Teilweise erfüllt: Die verbundenen Tasten können erfolgreich betätigt werden, wobei die Anzahl der Tasten auf
        8 reduziert werden musste \\
        \hline
        Anpassbarkeit & Erfüllt: Die Elektronik und Mechanik sind nicht Klavier-abhängig, wobei der Fußraum zum Befästigen der Aktuatoren
        gegeben sein muss oder das Ansteuerungskonzept geändert werden müsste\\
        \hline
        Musikstück-Auswahl & Erfüllt \\
        \hline
        Wiedergabe-Kontrolle & Erfüllt \\
        \hline
        Navigation & Erfüllt \\
        \hline
        MIDI-Integration & Erfüllt \\
        \hline
    \end{tabular}
    \caption{Ergebnisse der Anforderungen}
    \label{table:anorderungen-ergebnis}
\end{table}

% @Note(Jay) help this sounds
Insgesamt konnten im Laufe des Projekts viele Erkenntnisse getroffen werden.
Im Bereich der Hardware umfassen diese ein vertieftes Verständnis für Elektronik, zum Beispiel der Bedeutung der Kabeldicke, der Stromversorgungsanforderungen und der Notwendigkeit einer gründlichen Vorplanung. % @Note(Val): Mit Außnahme der Stromversorgungsanforderungen wird im Text davor nicht wirklich erklärt, was wir daran hätten lernen können. Kabeldicke kam nie vor und so wie wir es geschrieben haben, haben wir ja gründlich vorher geplant, also gibt es da eigentlich nichts zu lernen von, oder?
Letztendlich gab es viele Dinge die gut liefen, wie auch welche, die rückblickend anders getan hätten sollten.
Positiv zu vermerken ist die erfolgreiche Funktionalität des Schaltplans nach einigen Anpassungen sowie die hilfreiche Nutzung von Tinkercad als Simulationswerkzeug. % @Note(Val): Tinkercad wurde vorher in einem einzelnen Nebensatz erwähnt. Wenn wir es im Fazit haben wollen, sollte es bei der Konzeption des Schaltplans vllt. nochmal einen eigenen Satz bekommen oder so
Herausforderungen ergaben sich vor allem im Zeitmanagement, dem erstmaligen Testen der Schaltung und der Sicherheit im Umgang mit Elektronik, insbesondere aufgrund des begrenzten Vorwissens der Teammitglieder. % @Note(Val): Erneut, es ist unklar worauf sich das bezieht. Hat sich jemand verletzt? Wo ist die Lehre, die auch eine Leser:in daraus ziehen kann?
Die Projektgröße und die höheren als erwarteten Kosten, insbesondere durch Preisunterschiede bei den Aktuatoren zwischen verschiedenen Bezugsquellen, stellten zusätzliche Herausforderungen dar. % @Note(Val): siehe obige Anmerkung
Die Erfahrung aus dem Projekt bietet wertvolle Erkenntnisse im Projektmanagement, in der Recherche und im Fachwissen.
Es ist jedoch zu beachten, dass der Bau eines selbstspielenden Klaviers mit erheblichem Arbeitsaufwand und unvorhersehbaren Kosten verbunden ist. % @Note(Val): Fazit also, dass Menschen es nicht machen sollten?

Im größeren Kontext betrachtet, wurde das Projekt initiiert, um den benötigten Aufwand für die Eigenentwicklung selbstspielender Klaviere im Gegensatz zu fertig gekauften Player-Pianos zu untersuchen. % @Note(Val): "Player-Piano" kam so als Begriff nie auf und ist womöglich verwirrend
Die Ergebnisse zeigen, dass selbstgebaute Varianten im Vergleich zu käuflichen Produkten erhebliche Kosteneinsparungen ermöglichen, jedoch mit einem erheblichen Zeitaufwand und komplexeren Anpassungsmöglichkeiten einhergehen.
Gekaufte Versionen bieten möglicherweise eine höhere Qualität und eine elegantere Ästhetik, erfordern jedoch erhebliche finanzielle Investitionen.
Was Features betrifft, sind diese beim selbst bauen leichter anzupassen.
Dafür ist allerdings die Funktionsfähigkeit und  Mindesthaltbarkeit nicht garantiert.

Insgesamt steht fest, dass das Projekt, ein Player-Piano selber zu bauen, für jene Menschen sinnvoll ist, die gerne ausprobieren und viel Zeit (rund 40h nur für das Löten der Hälfte der Schaltung) für solche Projekte aufbringen. % @Note(Val): Grobe Schätzung zum benötigten Zeitaufwand, um alles aufzubauen wäre hier sehr interessant. Wenn Schaltplan bereits steht und alle benötigte Hardware da ist, wie lange braucht es dann die Hardware aufzubauen? Weil wenn das nur 80h oder so sind, dann wäre das mMn voll machbar und gut zu empfehlen. Falls diese Schätzungen auch noch auf die einzelnen Aufgaben aufgeteilt werden könnte (Zeit fürs Löten, Zeit für Befestigung der Aktuatoren, etc.), dann wäre das sehr sinnvoll.
% @Note(Val): Wenn wir hier auch die Kosten noch nennen könnten, die für die Einzelteile benötigt werden (vielleicht auch im Vergleich unserer Quelle und Alibaba oder so), dann hätten wir unseere Frage vom Start der Arbeit komplett beantwortet.
Es kann auf jeden Fall sehr viel Erfahrung im Bereich Projektmanagement, Recherche und natürlich Fachbezogenes Wissen mitgenommen werden. % @Note(Val): Wer kann das mitnehmen? Wir? Die Menschen, die unseren Beispiel folgend auch ein selbstspielendes Klavier bauen?
Vorab sollte allerdings klar sein, dass das Projekt einen hohen Zeitaufwand und - je nach Anbieter der Hardware - unerwartet Hohe Kosten mit sich bringt.
% @Note(Val): Inwieweit sind die Kosten unerwartet? Die stehen doch auf der Webseite und können schnell berechnet werden. Bei uns waren sie nur unerwartet, weil wir einen anderen Anbieter verwenden mussten als erwartet. Ansonsten sind die Kosten aber sehr erwartbar.


\section{Ausblick}

Für zukünftige Entwicklungen bieten sich zahlreiche Möglichkeiten sowohl im Software- als auch im Hardwarebereich des Projekts an.
Auf der Hardwareseite könnten ein verbessertes Sicherheitskonzept für Elektronik, die Fertigstellung der Schaltungen für die restlichen Aktuatoren und die Untersuchung von Schalldämmungstechniken in Betracht gezogen werden.
In Bezug auf Softwareverbesserungen könnte die Unterstützung weiterer Dateiformate wie PDF neben MIDI erwogen werden.
Es ist zu beachten, dass Projekte dieser Art selten abgeschlossen sind, da ständig neue Funktionen hinzugefügt und Verbesserungen vorgenommen werden können.

