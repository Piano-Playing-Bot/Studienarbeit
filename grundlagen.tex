% !TEX root =  master.tex
\chapter{Hardware}

\nocite{*}

\begin{enumerate}
	\item Genereller Überblick über mögliche Lösungsvorgehen
	\item Bereits existierende selbst-spielende Klaviere
	\item Arduino
	\item Schieberegister, LED-Matrix, Motoren etc.
	\item Theoretische Grundsätze für UIs
	\item Groundless der Musiktheorie, Musiknotation
	\item Grundlagen MIDI
	\item ...
\end{enumerate}
\newline

\section{Vorgehen}\label{Vorgehen - Hardware}

Für die Umsetzung eines selbstspielenden Klaviers wurden verschiedene technische Überlegungen angestellt.
Die Grundfragen, denen wir bei der Konstruktion begegneten, betrafen die Auswahl und Implementierung der
Aktuatoren, die Methode des Anspielens der Klaviertasten sowie die Signalübertragung des Arduinos.
Im Hinblick auf die Aktuatoren fiel unsere Wahl auf Hubmagneten, die eine präzise Steuerung der Tasten
ermöglichen. Durch ihre Verwendung können die Tasten des Klaviers mit der erforderlichen Geschwindigkeit und
Genauigkeit angespielt werden. \newline
Die Aktuatoren wurden mit den Klaviertasten mittels Seilen verbunden, um die mechanische Integrität des Klaviers
zu erhalten und gleichzeitig eine präzise Steuerung zu gewährleisten.\newline
Die Signalübertragung erfolgte über einen Arduino-Mikrocontroller. Da der Arduino nicht über ausreichend
viele Ausgänge verfügt, um alle Tasten anzusteuern, wurde das Signal mithilfe von Schieberegistern erweitert.
Diese Erweiterung ermöglichte es, alle 88 Tasten des Klaviers anzusteuern, ohne zusätzliche Hardware einsetzen
zu müssen.

\subsection{Ansteuerung und Signal}\label{Ansteuerung}

\newline\subsubsection{Arduino R3}
Zur Signalsteuerung wird ein Arduino R3 genutzt.
Es handelt sich dabei um eine Open-Source-Plattform für die Entwicklung von elektronischen Prototypen. Er besteht aus
einem Mikrocontroller-Board, das mit verschiedenen Sensoren, Aktuatoren und anderen elektronischen Komponenten
verbunden werden kann. Der Arduino verfügt über digitale und analoge Ein- und Ausgangspins, die für die Interaktion mit
externen Geräten verwendet werden können. \newline
Der Arduino Uno R3 eignet sich hervorragend zur Ansteuerung der Aktuatoren eines selbstspielenden Klaviers aus
mehreren Gründen. Erstens bietet der Arduino eine benutzerfreundliche Plattform für die Entwicklung und Implementierung
von Steuerungslogik. Die einfache Programmierung mittels der Arduino-IDE ermöglicht es, komplexe Steuerungsalgorithmen
für die Aktuatoren zu erstellen. \newline
Zweitens bietet er PWM (Pulsweitenmodulation) Pins, die im Rahmen des Projekts genutzt wurden.
PWM ist eine Technik, bei der die Zeitdauer eines digitalen Signals variiert wird, um einen durchschnittlichen Wert zu
erzeugen. Dies ermöglicht es, die Aktuatoren mit verschiedenen Stärken anzusteuern, indem die Pulsweite des Signals
entsprechend angepasst wird. Dadurch können die Klaviertasten mit unterschiedlichen Intensitäten angespielt werden,
was zu einer realistischeren und dynamischeren Wiedergabe führt.

\newline \subsubsection{Erweiterung der Ports}
Es werden 88 Tasten angesteuert. Da ein Arduino keine 88 PWM-Ports besitzt, müssen die Signale über
eine Erweiterung der Ausgänge an die Motoren weitergegeben werden. Dafür gibt es mehrere Möglichkeiten:
\newline
\begin{enumerate}
	\item Schieberegister
	\item Motor-Matrix
	\item Demultiplexer
\end{enumerate}

\newline \paragraph{Schieberegister}
//TODO
\paragraph{Motor-Matrix}
//TODO
\paragraph{Demultiplexer}
//TODO

