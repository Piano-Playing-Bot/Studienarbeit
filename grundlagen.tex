% !TEX root =  master.tex
\chapter{Hardware}

\nocite{*}

\begin{enumerate}
	\item Genereller Überblick über mögliche Lösungsvorgehen
	\item Bereits existierende selbst-spielende Klaviere
	\item Arduino
	\item Schieberegister, LED-Matrix, Motoren etc.
	\item Theoretische Grundsätze für UIs
	\item Groundless der Musiktheorie, Musiknotation
	\item Grundlagen MIDI
	\item ...
\end{enumerate}
\newline

\section{Vorgehen}\label{Vorgehen - Hardware}

\subsection{}

Die Hardware besteht konzeptionell aus zwei Modulen: \newline
\begin{enumerate}
	\item Ansteuerung und Signal
	\item das Anspielen der Tasten
\end{enumerate}
\newline

\subsection{Ansteuerung und Signal}\label{Ansteuerung}

Es wird ein Arduino R3 genutzt, um ein Signal an die Aktuatoren weiter zu geben.
\newline\subsubsection{Arduino}
Bei einem Arduino handelt es sich um einen Mikrocontroller...

//TODO

\newline \subsubsection{PWM}
\newline \subsubsection{Erweiterung der Ports}
Es werden 88 Tasten angesteuert. Da ein Arduino keine 88 PWM-Ports besitzt, müssen die Signale über
eine Erweiterung der Ausgänge an die Motoren weitergegeben werden. Dafür gibt es mehrere Möglichkeiten:
\newline
\begin{enumerate}
	\item Schieberegister
	\item Motor-Matrix
	\item Demultiplexer
\end{enumerate}

\newline \paragraph{Schieberegister}
//TODO
\paragraph{Motor-Matrix}
//TODO
\paragraph{Demultiplexer}
//TODO

