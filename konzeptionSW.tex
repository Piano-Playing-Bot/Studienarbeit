%%%%%%%%%%%%%%%%%%%%%%%%%%%%%%%%%%%%%%%%%%%%%%%%%%%%%%%%%%
%   Autoren des Abschnitts:
%   Val Richter
%%%%%%%%%%%%%%%%%%%%%%%%%%%%%%%%%%%%%%%%%%%%%%%%%%%%%%%%%%

% !TEX root =  master.tex
\nocite{*}
\chapter{Konzeption - Software} \label{vorgehenSW}
\chapterauthor{Val Richter}

Wie in Kapitel \ref{Zielstellung} beschrieben, benötigt dieses Projekt einen Desktop-Anwendung für die Nutzer:innen-Interaktion mit dem selbst-spielenden Klavier.
Außerdem wurde in Kapitel \ref{Hardware - Konzeption} bereits aufgezeigt, dass für die Ansteurung der Motoren ein \ac{MC} verwendet werden soll, der natürlich auch eine bestimmte Software ausführen muss.
Beide Komponenten, das \ac{UI} und der \ac{MC}, sollen in diesem Kapitel konzipiert werden.

Abbildung \ref*{fig:high-level-komponenten} zeigt diese unterschiedlichen Komponenten, sowie den Datenfluss zwischen diesen.
Nutzer:innen werden in der Abbildung auch dargestellt, um eindeutig zu machen, dass Nutzer-Eingaben nur an die \ac{UI} gehen.
Der \ac{MC} erhält Daten zum abzuspielenden Musikstück von der \ac{UI} und schickt darauf basierend in regelmäßigen Intervallen \ac{PWM}-Signale an die Hardware.

\begin{figure}[htbp]
    \centering
    \begin{tikzpicture}[node distance=2cm, line width=0.25mm]
        % Code for drawing a stick figure as taken from https://tex.stackexchange.com/a/441952
        % \begin{tikzpicture}
        % 	\thicklines
        % 	\put(5,18){\circle{5}}
        % 	\put(5,7){\line(0,1){8}}
        % 	\put(5,7){\line(1,-2){5}}
        % 	\put(5,7){\line(-1,-2){5}}
        % 	\put(1,12){\line(1,0){8}}
        % \end{tikzpicture}
        \node (user) [rect] {User}; % @TODO: draw Useer as stick figure
        \node (ui) [rect, below of=user] {UI};
        \node (mc) [rect, right of=ui, xshift=1.5cm] {\ac{MC}};
        \node (hw) [rect, right of=mc, xshift=1.5cm] {Hardware};

        \draw [arrow] (user) -- (ui);
        \draw [arrow] (ui) -- (mc);
        \draw [arrow] (mc) -- (hw);
    \end{tikzpicture}
    \caption{Komponenten-Diagramm mit Datenfluss}
    \label{fig:high-level-komponenten}
\end{figure}

Da die Hardware-Komponente in vorherigen Kapiteln bereits behandelt wurde, soll hier nur die Architektur der \ac{UI} (in Abschnitt \ref{vorgehenSW-UI}) und des \ac{MC} (in Abschnitt \ref{vorgehenSW-MC}) erarbeitet werden.
Dabei soll die Kommunikation zwischen diesen beiden Komponenten in Abschnitt \ref{vorgehenSW-SPPP} ebenfalls seperat betrachtet und spezifiziert werden.
Damit können die beiden Komponenten relativ aufwandsarm verändert oder ausgetauscht werden, solange sie sich weiterhin an das selbe Kommunikations-Protokoll halten.

Weiterhin soll auch das Format, in dem Musikstücke auf dem \ac{MC} gespeichert werden, in Abschnitt \ref{vorgehenSW-PIDI} seperat entworfen werden.
Da der \ac{MC} eine möglichst hohe Geschwindigkeit für die Ansteuerung der Motoren erbringen soll, muss das Format eine performante Nutzung ermöglichen.
Auch muss das Format möglichst wenig Speicherplatz aufbrauchen, damit möglichst viele Daten auf einmal auf dem \ac{MC} gespeichert werden können.
Je mehr Speicherplatz das Format nämlich benötigt, desto öfter muss die \ac{UI} Daten an den \ac{MC} schicken, was einen hohen Zeitaufwand mit sich bringen kann.

Zuletzt soll außerdem auch die Architektur der \ac{UI} in zwei Teile geteilt werden.
Die \ac{UI} hat an sich drei klar separierbare Aufgaben.
Zum einen muss die \ac{UI} natürlich Nutzer:innen-Eingaben verarbeiten und ein graphisches Interface rendern.
Zum anderen muss die \ac{UI} wie bereits erwähnt auch Daten an den \ac{MC} schicken.
Wie in Unterkapitel \ref{vorgehenSW-SPPP} aufgezeigt wird, muss die \ac{UI} auch Daten vom \ac{MC} empfangen können.
Die UI muss zusätzlich zum Rendering und der Kommunikation aber auch noch das Importieren von Musikstücken erlauben.
Wie in \ref{Zielstellung} spezifiziert, sollen Nutzer:innen Musikstücke in bestimmten Dateiformaten an die Anwendung geben können, um diese danach auf dem selbst-spielenden Klavier abspielen zu können.
Konzeptionelle Überlegungen dazu werden in \ref{vorgehenSW-MIDI} erläutert.


\section{Datenformat für Musikstücke} \label{vorgehenSW-PIDI}

Es gibt bereits viele Datenformate, die zum Speichern von Musikstücken verwendet werden können. Dabei lassen sich solche Formate generell in zwei fundamental verschiedenen Designs unterscheiden.

% @Note(Val): Sollten wir Quellen für die unterschiedlichen Formate hier in den Fußzeilen angeben? Z.B. auf die Standards verweisen?
Zum Einen gibt es generelle Audioformate. Dazu gehören solche Formate wie MP3, WAV oder FLAC.
Diese Formate speichern für jeden Zeitpunkt des Audios die Information über das Geräusch, das zu diesem Zeitpunkt abzuspielen ist.

Zum Anderen gibt es Formate, die speziell für Musikstücke entwickelt wurden.
Das Vorzeige-Beispiel wäre hier das \ac{MIDI}-Format.
Bei diesen Formaten werden stattdessen Informationen über Noten, die für eine gewisse Zeit zu spielen sind, gespeichert.
Das genaue Geräusch, das am Ende abgespielt werden sollte, wird dabei noch von anderen Faktoren beeinflusst.
So könnten mehrere Noten gleichzeitig zu spielen sein.
Auch könnte zum Beispiel das Instrument von Nutzer:innen dynamisch änderbar sein.

% @Note(Val): Ich sage öftermal "dieses Projekt". Gefällt uns das stilistisch? Machen wir das einheitlich?
Die zweite Art an Formaten ist im Kontext dieses Projekts zu bevorzugen, da es dem Format der Ausgabe näher kommt.
Die erstgenannten Formate speichern direkt die Informationen, die von digitalen Lautsprechern zum Abspielen von Audio benötigt werden.
Auch wenn je nach Format und Platform womöglich noch einige Datentransformationen getan werden müssen, bevor der Lautsprecher die Daten erhalten kann, ist eine Umwandlung von Informationen nicht notwendig.
% @Note(Val): Sollte hier vielleicht erwähnt werden, warum Informations-Umwandlung zu vermeiden ist?

Zum Anspielen eines Klaviers, werden stattdessen zu jedem Zeitpunkt die Information über derzeit zu spielende Tasten benötigt.
Formate der zweiten Art speichern genau jene Informationen und sind damit perfekt für diese Aufgabe geeignet.
Auf den ersten Blick erscheint die Verwendung eines Formats wie \ac{MIDI} damit eine gute Idee.
Es gibt hier allerdings noch weitere Aspekte, die zu beachten sind.

Dieses Projekt will explizit eine höhere Performanz als die schnellsten menschlichen Pianist:innen erreichen (siehe Kapitel \ref*{Zielstellung}).
Entsprechend sollte das gewählte Datenformat der Musikstücke auch möglichst schnell in das Format umgeformt werden können, das für die Ausgabe an die Aktuatoren benötigt wird.
Am performantesten wäre es hierbei, eine Liste jener Daten zu speichern, die zu jedem Zeitpunkt an die Aktuatoren ausgegeben werden müssen, da dann gar keine Datentransformation mehr nötig ist.

% @Note(Val): Muss ich erklären, warum erhöhte Kommunikation die Performanz beeinträchtigen könnte oder ist das selbsterklärend?
Allerdings muss das Datenformat auch möglichst wenig Speicherplatz verbrauchen.
Da der Code auf einem \ac{MC} laufen muss, ist Speicherplatz relativ begrenzt.
Der \ac{MC} muss dabei nicht das gesamte Musikstück auf einmal gespeichert haben, da restliche Teile des Musikstücks auch später von der \ac{UI} nachgereicht werden können.
Das verringert die Wichtigkeit des Speicherplatzes jedoch nur zu Teilen, da ein solches Streamen der Daten mehr Kommunikation zwischen \ac{UI} und {MC} benötigt.
Da die Latenzen der Kommunikation relativ hoch sein können, gefährdet dies wiederrum das Ziel der Performanz.
% @TODO(Eine Anmerkung wie hoch Latenz & Throughput sein können (oder wo diese Zahlen genannt werden), wäre hier gut)

% @TODO(Val): Quellen für \ac{MIDI} Details einfügen
\ac{MIDI} schneidet hierbei an sich sehr gut ab, da Noten sehr kompakt abgespeichert werden.
Dafür wird eine Enkodierung variabler Länge verwendet.
Da \ac{MIDI} allerdings ein generelles Format ist, das ein Vielzahl von digitalen Instrumenten und Synthesizers unterstützen möchte, beinhält der \ac{MIDI}-Standard eine Vielzahl optionaler Metadaten, die für dieses Projekt irrelevant sind.
Entsprechend würde nur ein Subset des \ac{MIDI}-Standards zum Speichern der Musikdaten in Frage kommen.
Nichtsdestotrotz kann das Format für diesen eingeschränkten Anwendungsfall noch optimiert werden.
Um diese Verbesserungen zu motivieren, soll zuerst die Enodierung von Noten in \ac{MIDI} erläutert werden.

\ac{MIDI} speichert die Notenfolgen in sogenannten \enquote{Nachrichten}.
Nachrichten können auch andere Daten tragen, hier sollen aber nur die Nachrichten zum Spielen von Tönen betrachtet werden.
Jede dieser Nachrichten folgt einer relativen Zeitangabe, im Standard \enquote{delta time} genannt.
\enquote{delta time} gibt an, wie viele Schläge der virtuellen \ac{MIDI}-Uhr zwischen der letzten Nachricht und dieser vergehen sollen.
Die \ac{MIDI}-Uhr kann direkt in Angaben einer Wand-Uhr umgerechnet werden.
Hier ist jedoch wichtig zu bemerken, dass diese \ac{MIDI}-Uhr sehr viel genauer als eine Angabe in Millisekunden ist.
Auch ist zu erwähnen, dass diese \enquote{delta time} eine variable Länge hat, mindestens aber immer ein Byte verwendet.

Nach der \enquote{delta time} folgt dann genau ein Byte für den \enquote{Ereignis-Code} der Nachricht.
Dieser gibt in jeweils 4 Bits den \enquote{Status} und den \enquote{Kanal} des Ereignisses an.
Während der Status angibt, was für eine Aktion ausgeführt werden soll und damit auch wie viele Datenbytes für diese Nachricht noch folgen, besagt der Kanal, für welches Instrument dieses Ereignis gilt.
Wenn die Note gespielt werden soll, folgen darauf noch zwei Bytes, die jeweils die Note und die Anspielstärke abspeichern.
Wenn mit dem Spielen der Note aufgehört werden soll, folgt nur ein Byte, das die jeweilige Note angibt.

Hierbei fallen einige Daten auf, die in diesem Projekt unnötig sind.
So gibt es hier zum Beispiel nur ein einziges Instrument, weswegen die Notwendigkeit unterschiedlicher Kanäle wegfällt.
Auch ist die sehr hohe Auflösung der \enquote{delta time} hier nicht benötigt.
Stattdessen würde eine Auflösung in Millisekunden oder Centisekunden für dieses Projekt ausreichen.

Weiterhin fällt auf, dass für das Spielen jedes Tons in \ac{MIDI} mindestens zwei Nachrichten benötigt werden.
Einmal eine Nachricht zum Starten des Ton-Spielens und einmal eine Nachricht zum Aufhören.
Diese zweite Nachricht wäre nicht nötig, wenn zusammen mit der ersten Nachricht eine Länge, mit der der Ton gespielt werden soll, gespeichert würde.
Diese Speicher-Optimierung bringt allerdings auch Nachteile mit sich.
So verkompliziert es die Transformation der Musikdaten, bevor sie zum Ansteuern der Aktuatoren verwendet werden können.
Auch müssen die zeitlichen Längen, für die die zurzeit spielenden Töne noch zu spielen sind, gespeichert werden.
Dies könnte im Generellen einiges an Speicherplatz verbrauchen.
In diesem speziellen Anwendungsfall lässt sich allerdings annehmen, dass Noten in der Regel nur für relativ kurze Zeiten gespielt werden und dass nur relativ wenige Noten auf einmal gespielt werden.
Da wie in Kapitel \ref{???} erklärt, auch nur eine geringe Anzahl der Aktuatoren auf einmal aktiviert werden sollen, kann die Menge zusätzlich benötigten Speichers hier sehr gering gehalten werden.

Aufgrund all dieser möglichen Verbesserungen des Formats, wurde sich entschieden, ein eigenes Datenformat für diesen Anwendungsfall zu entwickeln, welches im Folgenden \enquote{\ac{PIDI}} genannt wird.
Abgesehen von einer effizienten Speichernutzung, soll das \ac{PIDI}-Format auch performant zum Ansteuern der Aktuatoren transformiert werden können und möglichst simpel zu benutzen sein.
Außerdem soll es trotz der starken Einschränkung auf diesen Anwendungsfall variabel genug sein, um für unterschiedliche Konfigurationen von Klavieren verwendbar zu sein.
Das ist deshalb sinnvoll, da es die Möglichkeit bietet, die Anzahl verbundener Tasten des Klaviers jederzeit zu verändern.
Während Testphasen müssen somit nicht alle Tasten des Klaviers über Aktuatoren anspielbar sein.
Auch kann das \ac{PIDI}-Format dann in ähnlichen Projekten mit einer womöglich anderen Anzahl anspielbarer Tasten auf dem Klavier verwendet werden.

Listing \ref{code:PidiCmd struct} zeigt die endgültige Datenstruktur, die für ein individuellen \lstinline{PidiCmd} gewählt wurde, in Form eines C-Bitfields.
Ein \lstinline{PidiCmd} stellt dabei das Anspielens eines Tons dar.
Die Töne selbst wurden als ein Paar aus Oktave und Taste in der Oktave dargestellt, da die Anzahl Tasten in einer Oktave auf praktisch allen Klavieren gleich ist.
Die nullte Oktave bezeichnet hierbei die mittlere Oktave des Pianos, während positive und negative Zahlen in Relation zur mittleren Oktave zu verstehen sind.
Mit jeweils 4 Bits können alle 12 Tasten in einer Oktave und alle Oktaven, die auch auf großen Klavieren Platz finden, verwendet können.

Die \enquote{velocity} eines \lstinline{PidiCmd}s gibt die Anschlagsstärke an.
Um Speicherplatz zu sparen und da genauere Präzision beim Anspielen sich auf Hardware-Seiten eh als Schwierigkeit ergeben hat (siehe Kapitel \ref{???}), wurde sich entschieden, nur 4 Bits für die \enquote{velocity} bereitzustellen.
Zuletzt gibt es dann noch die \lstinline{dt} und \lstinline{len}, die respektiv jeweils die Zeit seit der zuletzt gespielten Note und die Spiellänge der Note darstellen.
Während in \ac{MIDI} eine sehr hohe und dynamisch veränderbare Auflösung für die \enquote{delta time} verwendet wurde, ist \lstinline{dt} hier immer in Millisekunden enkodiert.
Mit den dafür allokierten 12 Bits können $\frac{2^{12}}{1000} = 4,096$ Sekunden ohne Anspielen eines neuen Tons enkodiert werden.
Da die Anspiellänge zusätzlich noch für jeden zurzeit gespielten Ton gespeichert werden muss, wurde entschieden genau 1 Byte dafür zu verwenden.
Dies erlaubt eine leichte und speicher-effiziente Nutzung außerhalb der \lstinline{PidiCmd}-Datenstruktur.
Da 8 Bits nur Spiellängen von bis zu $\frac{2^8}{1000} = 0,256$ Sekunden abspeichern können, wurde entschieden hier stattdessen eine Enkodierung in Centisekunden zu benutzen.
Damit können bis zu $\frac{2^8}{1000} = 2,56$ Sekunden abgespeichert werden.

Insgesamt benötigt ein \lstinline{PidiCmd} dann genau 4 Byte.
Damit passt die Datenstruktur gut in CPU-Register, was eine schnellere Benutzung ermöglicht.
Wenn davon ausgegangen wird, dass die meisten Töne mit einer Anspielstärke gespielt werden und ausklingen bevor sie mit einer anderen oder derselben Stärke wieder gespielt werden, dann benötigt \ac{MIDI} für jeden Ton mindestens 7 Bytes.
Dabei werden 4 Bytes zum initialen Anspielen und 3 Bytes zum Ausklingen des Tons verwendet.
\ac{PIDI} benötigt dagegen immer genau 4 Bytes für jeden Ton.
Damit ist verbraucht \ac{PIDI}-Format mindestens $\frac{7}{4} = 1,75$-Mal weniger Speicher als das \ac{MIDI}-Format.

\begin{UnbrokenCodePage}[style=CStyle, caption={Definition eines \ac{PIDI}-Kommands}, label={code:PidiCmd struct}]
typedef enum {
    PIANO_KEY_C,
    PIANO_KEY_CS,
    PIANO_KEY_D,
    PIANO_KEY_DS,
    PIANO_KEY_E,
    PIANO_KEY_F,
    PIANO_KEY_FS,
    PIANO_KEY_G,
    PIANO_KEY_GS,
    PIANO_KEY_A,
    PIANO_KEY_AS,
    PIANO_KEY_B,
} PianoKey;

typedef struct PidiCmd {
    uint32_t dt : 12,
    velocity    : 4,
    len         : 8,
    octave      : 4, // needs to be sign-extended when used, because it's a signed 4-Bit integer
    key         : 4;
} PidiCmd;
\end{UnbrokenCodePage}

\section{Logik des Microcontrollers} \label{vorgehenSW-MC}

Der \ac{MC} hat grundlegend zwei Aufgaben.
Einerseits muss der \ac{MC} Daten an die Aktuatoren senden, um das Klavier richtig anzuspielen.
Andererseits muss der \ac{MC} auch mit der \ac{UI} kommunizieren können.
Da der \ac{MC} nur einen einzelnen CPU-Kern hat, können diese Aufgaben nur abwechselnd nacheinander und nicht parallel bearebeitet werden können.

Um die richtigen Werte zu den Aktuatoren zu senden, müssen nacheinander alle Werte an die Schieberegister gesendet werden.
Der Wert des letzten Schieberegister-Ausgangs muss dabei zuerst gesendet werden, da es dann als erstes komplett durchgeschoben wurde.
Sobald alle Werte ausgesandt wurden, müssen die Ausgänge der Schieberegister gesperrt werden, damit diese nicht ständig fluktuieren.
All das kann beim \ac{MC}, der in Kapitel \ref{Ansteuerung} gewählt wurde, über das Ansprechen der \ac{PWM}-Pins des \ac{MC}s getan werden.

Wann immer die Signale für die Aktuatoren ausgesandt werden sollen, müssen diese Werte vorher bereitstehen.
Da 88 Aktuatoren angesteuert werden können sollen, muss eine Liste von 88 \ac{PWM}-Werten befüllt werden.
Im Folgenden soll diese Liste \lstinline{piano} genannt werden.
Um \lstinline{piano} zu befüllen müssen die Informationen des Musikstücks verwendet werden, die wie in Abschnitt \ref{vorgehenSW-PIDI} beschrieben, als eine Liste von \lstinline{PidiCmd}s gespeichert werden.

Der große Vorteil der \lstinline|PidiCmd|-Struktur, liegt in einfachen Anpassung der \lstinline|piano| Liste.
Nachdem das \lstinline|piano| korrekt initialisiert wurde, muss der Code nur linear durch die Liste der zeitlich sortierten Kommandos gehen und das \lstinline|piano| entsprechend anpassen.
Da \lstinline|PidiCmd|s jeweils erst zu einer gewissen Zeit nach Start des Musikstücks angewandt werden sollen, muss neben dem Index in die Liste von \ac{PidiCmd}s auch die seitdem vergangene Zeit gespeichert werden.
Mit diesen beiden Werten, kann dann zu jedem Zeitpunkt geprüft werden, ob der nächste \lstinline{PidiCmd} angewandt werden soll oder nicht.
Diese Logik ist in Listing \ref{code:Apply PIDICmds} in Pseudocode dargestellt.

\begin{UnbrokenCodePage}[style=CStyle, caption={Nutzung der \lstinline|PidiCmd|-Struktur}, label={code:Apply PIDICmds}]
PIDICmds *cmds; // assume this array to be initialized already
u8  piano[88];  // list of values to send to magnets
u64 index        = 0;
u64 start_time   = currentTime(); // time in milliseconds
u64 elapsed_time = 0;
while (index < length(cmds)) {
    while (index < length(cmds) && cmds[index].time <= elapsed_time) {
        applyCmd(piano, cmds[index]);
        index++;
    }
    elapsed_time = getElapsedTimeSince(start_time);
}
\end{UnbrokenCodePage}

Die in Listing \ref{code:Apply PIDICmds} erwähnten Funktionen \lstinline|currentTime| und \lstinline|getElapsedTimeSince| sind über die eingebaute Uhr des \ac{MC} umsetzbar.
Die Funktion \lstinline|aplyCmd| muss dagegen die Stärke des Tastenanschlags an der richtigen Stelle in der \lstinline|piano| Liste speichern.
Da die \lstinline|PidiCmd|-Struktur die Anordnung und Anzahl verfügbarer Aktuatoren ignoriert, muss diese Umrechnung von Oktave und Ton in den genauen Aktutor, der diese Taste betätigt, hier geschehen.

% @TODO(Val): Rest schreiben...

\section{UI-Arduino Kommunikation} \label{vorgehenSW-SPPP}
\begin{enumerate}
    \item Wie wird Kommunikation mit Arduino i.d.R. gelöst
    \item Grundlagen der Kommunikation (Files, Polling, etc.)
    \item Prinzip der minimalen Arbeit → Custom Protocol → SPPP
    \item Asynchrone Umsetzung → Threading bei UI; Message-Buffer bei Arduino
\end{enumerate}

\section{UI} \label{vorgehenSW-UI}
\begin{enumerate}
    \item Grundlagen: Immediate vs. Retained Mode UI
    \item Cached Immediate Mode
    \item Generelle Diskussion bzgl. API-Design vllt
\end{enumerate}

\section{Parsing von \ac{MIDI}} \label{vorgehenSW-MIDI}
\begin{enumerate}
    \item Überblick über Midi
    \item Vergleich \ac{MIDI} vs PIDI
\end{enumerate}
(Kapitel vllt uninteressant?)