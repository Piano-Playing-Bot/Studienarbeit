%%%%%%%%%%%%%%%%%%%%%%%%%%%%%%%%%%%%%%%%%%%%%%%%%%%%%%%%%%
%   Autoren des Abschnitts:
%   Val Richter
%%%%%%%%%%%%%%%%%%%%%%%%%%%%%%%%%%%%%%%%%%%%%%%%%%%%%%%%%%

% !TEX root =  master.tex
\nocite{*}
\chapter{Konzeption - Software} \label{vorgehenSW}

Wie in Kapitel \ref{Zielstellung} beschrieben, benötigt dieses Projekt einen Desktop-Anwendung für die Nutzer:innen-Interaktion mit dem selbst-spielenden Klavier.
Außerdem wurde in Kapitel \ref{Hardware - Konzeption} bereits aufgezeigt, dass für die Ansteurung der Motoren ein \ac{MC} verwendet werden soll, der natürlich auch eine bestimmte Software ausführen muss.
Beide Komponenten, das \ac{UI} und der \ac{MC}, sollen in diesem Kapitel konzipiert werden.

Abbildung \ref*{fig:high-level-komponenten} zeigt diese unterschiedlichen Komponenten, sowie den Datenfluss zwischen diesen.
Nutzer:innen werden in der Abbildung auch dargestellt, um eindeutig zu machen, dass Nutzer-Eingaben nur an die \ac{UI} gehen.
Der \ac{MC} erhält Daten zum abzuspielenden Musikstück von der \ac{UI} und schickt darauf basierend in regelmäßigen Intervallen \ac{PWM}-Signale an die Hardware.

\begin{figure}[htbp]
    \centering
    \begin{tikzpicture}[node distance=2cm, line width=0.25mm]
        % Code for drawing a stick figure as taken from https://tex.stackexchange.com/a/441952
        % \begin{tikzpicture}
        % 	\thicklines
        % 	\put(5,18){\circle{5}}
        % 	\put(5,7){\line(0,1){8}}
        % 	\put(5,7){\line(1,-2){5}}
        % 	\put(5,7){\line(-1,-2){5}}
        % 	\put(1,12){\line(1,0){8}}
        % \end{tikzpicture}
        \node (user) [rect] {User}; % @TODO: draw Useer as stick figure
        \node (ui) [rect, below of=user] {UI};
        \node (mc) [rect, right of=ui, xshift=1.5cm] {\ac{MC}};
        \node (hw) [rect, right of=mc, xshift=1.5cm] {Hardware};

        \draw [arrow] (user) -- (ui);
        \draw [arrow] (ui) -- (mc);
        \draw [arrow] (mc) -- (hw);
    \end{tikzpicture}
    \caption{Komponenten-Diagramm mit Datenfluss}
    \label{fig:high-level-komponenten}
\end{figure}

Da die Hardware-Komponente in vorherigen Kapiteln bereits behandelt wurde, soll hier nur die Architektur der \ac{UI} (in Abschnitt \ref{vorgehenSW-UI}) und des \ac{MC} (in Abschnitt \ref{vorgehenSW-MC}) erarbeitet werden.
Dabei soll die Kommunikation zwischen diesen beiden Komponenten in Abschnitt \ref{vorgehenSW-SPPP} ebenfalls seperat betrachtet und spezifiziert werden.
Damit können die beiden Komponenten relativ aufwandsarm verändert oder ausgetauscht werden, solange sie sich weiterhin an das selbe Kommunikations-Protokoll halten.

Weiterhin soll auch das Format, in dem Musikstücke auf dem \ac{MC} gespeichert werden, in Abschnitt \ref{vorgehenSW-PIDI} seperat entworfen werden.
Da der \ac{MC} eine möglichst hohe Geschwindigkeit für die Ansteuerung der Motoren erbringen soll, muss das Format eine performante Nutzung ermöglichen.
Auch muss das Format möglichst wenig Speicherplatz aufbrauchen, damit möglichst viele Daten auf einmal auf dem \ac{MC} gespeichert werden können.
Je mehr Speicherplatz das Format nämlich benötigt, desto öfter muss die \ac{UI} Daten an den \ac{MC} schicken, was einen hohen Zeitaufwand mit sich bringen kann.

Zuletzt soll außerdem auch die Architektur der \ac{UI} in zwei Teile geteilt werden.
Die \ac{UI} hat an sich drei klar separierbare Aufgaben.
Zum einen muss die \ac{UI} natürlich Nutzer:innen-Eingaben verarbeiten und ein graphisches Interface rendern.
Zum anderen muss die \ac{UI} wie bereits erwähnt auch Daten an den \ac{MC} schicken.
Wie in Unterkapitel \ref{vorgehenSW-SPPP} aufgezeigt wird, muss die \ac{UI} auch Daten vom \ac{MC} empfangen können.
Die UI muss zusätzlich zum Rendering und der Kommunikation aber auch noch das Importieren von Musikstücken erlauben.
Wie in \ref{Zielstellung} spezifiziert, sollen Nutzer:innen Musikstücke in bestimmten Dateiformaten an die Anwendung geben können, um diese danach auf dem selbst-spielenden Klavier abspielen zu können.
Konzeptionelle Überlegungen dazu werden in \ref{vorgehenSW-MIDI} erläutert.


\section{Datenformat für Musikstücke} \label{vorgehenSW-PIDI}
% \begin{enumerate}
%     \item Existierendes vs. Custom Format
%     \item Darstellung von Tönen (nicht an geg. Piano mit 88 Tasten gebunden)
%     \item Speicher vs Performanz im Design des Formats (inkl. Alternativen)
% \end{enumerate}

Es gibt bereits viele Datenformate, die zum Speichern von Musikstücken verwendet werden können. Dabei lassen sich solche Formate generell in zwei fundamental verschiedenen Designs unterscheiden.

% @Note(Val): Sollten wir Quellen für die unterschiedlichen Formate hier in den Fußzeilen angeben? Z.B. auf die Standards verweisen?
Zum Einen gibt es generelle Audioformate. Dazu gehören solche Formate wie MP3, WAV oder FLAC.
Diese Formate speichern für jeden Zeitpunkt des Audios die Information über das Geräusch, das zu diesem Zeitpunkt abzuspielen ist.

Zum Anderen gibt es Formate, die speziell für Musikstücke entwickelt wurden.
Das Vorzeige-Beispiel wäre hier das MIDI-Format.
Bei diesen Formaten werden stattdessen Informationen über Noten, die für eine gewisse Zeit zu spielen sind, gespeichert.
Das genaue Geräusch, das am Ende abgespielt werden sollte, wird dabei noch von anderen Faktoren beeinflusst.
So könnten mehrere Noten gleichzeitig zu spielen sein.
Auch könnte zum Beispiel das Instrument von Nutzer:innen dynamisch änderbar sein.

% @Note(Val): Ich sage öftermal "dieses Projekt". Gefällt uns das stilistisch? Machen wir das einheitlich?
Die zweite Art an Format ist im Kontext dieses Projekts zu bevorzugen, da es dem Format der Ausgabe näher kommt.
Die erstgenannten Formate speichern direkt die Informationen, die von digitalen Lautsprechern zum Abspielen von Audio benötigt werden.
Auch wenn je nach Format und Platform womöglich noch einige Datentransformationen getan werden müssen, bevor der Lautsprecher die Daten erhalten kann, ist eine Umwandlung von Informationen nicht notwendig.
% @Note(Val): Sollte hier vielleicht erwähnt werden, warum Informations-Umwandlung zu vermeiden ist?

Zum Anspielen eines Klaviers, werden stattdessen zu jedem Zeitpunkt die Information über derzeit zu spielende Tasten benötigt.
Formate der zweiten Art speichern genau jene Informationen und sind damit perfekt für diese Aufgabe geeignet.
Auf den ersten Blick erscheint die Verwendung eines Formats wie MIDI damit eine gute Idee.
Es gibt hier allerdings noch weitere Aspekte, die zu beachten sind.

Dieses Projekt will explizit eine höhere Performanz als die schnellsten menschlichen Pianist:innen erreichen (siehe Kapitel \ref*{Zielstellung}).
Entsprechend sollte das gewählte Datenformat der Musikstücke auch möglichst schnell in das Format umgeformt werden können, das für die Ausgabe an die Aktuatoren benötigt wird.
Am performantesten ist es natürlich, einfach eine Liste jener Daten zu speichern, die zu jedem Zeitpunkt an die Aktuatoren ausgegeben werden müssen, da dann überhaupt keine Datentransformation mehr nötig ist.

% @Note(Val): Muss ich erklären, warum erhöhte Kommunikation die Performanz beeinträchtigen könnte oder ist das selbsterklärend?
Allerdings muss das Datenformat auch möglichst wenig Speicherplatz verbrauchen.
Da der Code auf einem \ac{MC} laufen muss, ist Speicherplatz relativ begrenzt.
Der \ac{MC} muss dabei nicht das gesamte Musikstück auf einmal gespeichert haben, da restliche Teile des Musikstücks auch später von der \ac{UI} nachgereicht werden können.
Das verringert die Wichtigkeit des Speicherplatzes jedoch nur zu Teilen, da ein solches Streamen der Daten mehr Kommunikation zwischen \ac{UI} und {MC} benötigt und erhöhte Kommunikation die Performanz gefährden kann.

% @TODO(Val): Quellen für MIDI Details einfügen
MIDI schneidet hier nun nicht allzu gut ab.
Auch wenn MIDI die relevanten Informationen direkt abspeichert, speichert es ebenfalls noch eine relativ große Menge weiterer Daten ab.
Das liegt daran, das MIDI ein generelles Format ist, das eine Vielzahl von digitalen Instrumenten und Synthesizers untersützen möchte.
Deswegen enthalten MIDI-Dateien oftmals eine Vielzahl von Metadaten, die hier irrelevant sind.
Weiterhin gibt es in einer MIDI-Datei auch mehrere sogenannte Channels, damit unterschiedliche Instrumente gleichzeitig angespielt werden können.
% @TODO(Val):  Vllt. Nicht-linearität von MIDI-Dateien erwähnen (i.e. zeitliche Sortierung ist nur innerhalb von Chunks festgelegt)

% @Note(Val): Muss das mit einer Quelle belegt werden?
MIDI-Alternativen haben die selben Probleme, da sie ebenfalls einen allgemeinen Anwendungsfall abbilden.
Da der Anwendungsfall dieser Arbeit eingeschränkter ist, lassen sich weitere Optimierungen am Datenformat durchführen.
Entsprechend muss dann aber auch ein eigenes Datenformat für diese Arbeit entwickelt werden.
Dabei ist es sinnvolll klar festzuhalten, welche Rahmenbedingungen vom Format angenommen werden können.

Das Datenformat, das im Folgenden \enquote{\ac{PIDI}} genannt wird, soll zu spielende Tonfolgen für ein einzelnes Piano speichern.
Dabei darf \ac{PIDI} keine Annahmen über die Anzahl an Tasten auf dem Piano oder deren Anordnung machen.
Das ist speziell deshalb sinnvoll, weil nicht alle Pianos die selbe Anzahl Tasten haben und Hardware-seitig dann nicht unbedingt alle Tasten anspielbar sein müssen.
Zusätzlich wird das Architektur-Prinzip der \enquote{Seperation of Concerns} dabei bewahrt, da die \ac{UI} dann nichts über die genaue Anzahl vorhandener Tasten beim Piano wissen muss.
\ac{PIDI} darf weiterhin annehmen, das jede Taste immer entweder nur gedrückt oder nicht gedrückt sein kann.
Die Stärke, mit der eine Taste gespielt wird, soll dabei aber in einem Bereich von mindestens 8 unterschiedlichen Werten liegen.
Zuletzt darf die Geschwindigkeit, mit der Daten an die Aktuatoren weitergegeben werden, nicht vorausgesetzt werden.

Eine Option wäre die Speicherung der Zustände jeder Taste des Pianos.
Da nicht angenommen werden soll, wie viele Tasten das Piano hat, müsste dabei eine arbiträre Anzahl an Tasten für das virtuelle Piano gewählt werden.
Für jeden Zeitpunkt des Musikstücks müsste dann für jede Taste des virtuellen Pianos ein Anschlagswert gespeichert werden.
Die Anzahl benötigten Speichers wächst dabei linear mit der Länge des Musikstücks und wäre schon bei Stücken mehrer Sekunden sehr hoch.
Dafür ist die Transformation dieses Formats in für die Aktuatoren benötigte Format trivial und sehr effizient.
Zur Speicherplatzoptimierung könnte dieses Format noch mithilfe von Kompressionstechniken noch verdichtet werden.
Je besser die Kompression dabei sein soll, desto komplexer und langsamer muss auch die Dekompression werden.

Eine alternative Idee ist nur das Speichern von Änderungen.
Dabei müsste nur jeder Zeitpunkt gespeichert werden, in dem sich die Anschlagsstärke mindestens eines Tons verändert.
Der verbrauchte Speicherplatz wächst damit nur noch linear mit der Anzahl sich verändernder Tasten-Anschläge.
Im Regelfall sollte damit also eine niedrigere Speicher-Auslastung möglich sein.
Zusätzlich ist die Performanz mit dieser Speicherart ähnlich gut, wenn nicht sogar besser.
Für jede Anschlagsveränderung muss nur ein Wert in der Ausgabe an die Aktuatoren angepasst werden.
Sind die Daten noch zeitlich sortiert, kann ein laufender Index verwendet zu werden, um die nächste Anschlagsänderung zu erhalten.
Es wurde sich entsprechend entschieden, diese Speicherung für das \ac{PIDI}-Format zu verwenden.

Es ist hierbei noch zu bemerken, dass für das Spielen jedes Tons mindestens zwei Anschlagsänderungen vorliegen, da der Ton sowohl anfangen als auch aufhören muss zu spielen.
Die Anschlagsstärke muss dabei auch nur beim Anfangen oder Ändern der Stärke gespeichert werden, da beim Aufhören des Spielens immer eine Anschlagsstärke von Null daliegt.
Aufgrund dieser Beochbachtungen, kann das \ac{PIDI}-Format bzgl. Speicherplatz weiter optimiert werden.
So könnte zum Beispiel immer eine Länge des Anschlagens gespeichert werden, statt eine weitere Änderung zum Stoppen des Anschlags zu speichern.
Allerdings bedeutet das auch, dass ein einfacher, fortlaufender Index nicht mehr ausreichen würde, um zu wissen, ob sich die Daten für die Aktuatoren ändern müssen.
Das bedeutet also eine ineffizientere Verwendung der Datenstruktur.
Hier wurde sich entschieden, dass eine Halbierung des Speicherplatzes diesen Performanz-Verlust nicht wert ist.

Listing \ref{code:PidiCmd struct} zeigt die endgültige Datenstruktur, die für ein individuellen \lstinline{PidiCmd} gewählt wurde.
Ein \lstinline{PidiCmd} stellt dabei die Änderung einer Anschlagsstärke eines Tons dar.
Die Töne selbst wurden als ein Paar aus Oktave und Taste in der Oktave dargestellt, da angenommen werden durfte, wie viele Tasten es pro Oktave auf einem Klavier gibt.
Die nullte Oktave bezeichnet die mittlere Oktave des Pianos.
Die \enquote{velocity} gibt in dieser Struktur die Anschlagsstärke an.
Zuletzt gibt die \enquote{time} noch die Zeit in Millisekunden nach Start des Musikstücks an, zu dem diese Änderung durchgeführt werden soll.
Eine Liste von \lstinline{PidiCmd}os muss aufsteigend nach der Zeit sortiert sein.

\begin{UnbrokenCodePage}[style=CStyle, caption={Definition eines \ac{PIDI}-Kommands}, label={code:PidiCmd struct}]
typedef enum {
    PIANO_KEY_C,
    PIANO_KEY_CS,
    PIANO_KEY_D,
    PIANO_KEY_DS,
    PIANO_KEY_E,
    PIANO_KEY_F,
    PIANO_KEY_FS,
    PIANO_KEY_G,
    PIANO_KEY_GS,
    PIANO_KEY_A,
    PIANO_KEY_AS,
    PIANO_KEY_B,
} PianoKey;

typedef struct {
    u64  time;
    u8   velocity;
    i8   octave;
    PianoKey key;
} PidiCmd;
\end{UnbrokenCodePage}

\section{Logik des Microcontrollers} \label{vorgehenSW-MC}

Der \ac{MC} hat grundlegend zwei Aufgaben.
Einerseits muss der \ac{MC} Daten an die Aktuatoren senden, um das Klavier richtig anzuspielen.
Andererseits muss der \ac{MC} auch mit der \ac{UI} kommunizieren können.
Da der \ac{MC} nur einen einzelnen CPU-Kern hat, können diese Aufgaben nur abwechselnd nacheinander und nicht parallel bearebeitet werden können.

Um die richtigen Werte zu den Aktuatoren zu senden, müssen nacheinander alle Werte an die Schieberegister gesendet werden.
Der Wert des letzten Schieberegister-Ausgangs muss dabei zuerst gesendet werden, da es dann als erstes komplett durchgeschoben wurde.
Sobald alle Werte ausgesandt wurden, müssen die Ausgänge der Schieberegister gesperrt werden, damit diese nicht ständig fluktuieren.
All das kann beim \ac{MC}, der in Kapitel \ref{Ansteuerung} gewählt wurde, über das Ansprechen der \ac{PWM}-Pins des \ac{MC}s getan werden.

Wann immer die Signale für die Aktuatoren ausgesandt werden sollen, müssen diese Werte vorher bereitstehen.
Da 88 Aktuatoren angesteuert werden können sollen, muss eine Liste von 88 \ac{PWM}-Werten befüllt werden.
Im Folgenden soll diese Liste \lstinline{piano} genannt werden.
Um \lstinline{piano} zu befüllen müssen die Informationen des Musikstücks verwendet werden, die wie in Abschnitt \ref{vorgehenSW-PIDI} beschrieben, als eine Liste von \lstinline{PidiCmd}s gespeichert werden.

Der große Vorteil der \lstinline|PidiCmd|-Struktur, liegt in einfachen Anpassung der \lstinline|piano| Liste.
Nachdem das \lstinline|piano| korrekt initialisiert wurde, muss der Code nur linear durch die Liste der zeitlich sortierten Kommandos gehen und das \lstinline|piano| entsprechend anpassen.
Da \lstinline|PidiCmd|s jeweils erst zu einer gewissen Zeit nach Start des Musikstücks angewandt werden sollen, muss neben dem Index in die Liste von \ac{PidiCmd}s auch die seitdem vergangene Zeit gespeichert werden.
Mit diesen beiden Werten, kann dann zu jedem Zeitpunkt geprüft werden, ob der nächste \lstinline{PidiCmd} angewandt werden soll oder nicht.
Diese Logik ist in Listing \ref{code:Apply PIDICmds} in Pseudocode dargestellt.

\begin{UnbrokenCodePage}[style=CStyle, caption={Nutzung der \lstinline|PidiCmd|-Struktur}, label={code:Apply PIDICmds}]
PIDICmds *cmds; // assume this array to be initialized already
u8  piano[88];  // list of values to send to magnets
u64 index        = 0;
u64 start_time   = currentTime(); // time in milliseconds
u64 elapsed_time = 0;
while (index < length(cmds)) {
    while (index < length(cmds) && cmds[index].time <= elapsed_time) {
        applyCmd(piano, cmds[index]);
        index++;
    }
    elapsed_time = getElapsedTimeSince(start_time);
}
\end{UnbrokenCodePage}

Die in Listing \ref{code:Apply PIDICmds} erwähnten Funktionen \lstinline|currentTime| und \lstinline|getElapsedTimeSince| sind über die eingebaute Uhr des \ac{MC} umsetzbar.
Die Funktion \lstinline|aplyCmd| muss dagegen die Stärke des Tastenanschlags an der richtigen Stelle in der \lstinline|piano| Liste speichern.
Da die \lstinline|PidiCmd|-Struktur die Anordnung und Anzahl verfügbarer Aktuatoren ignoriert, muss diese Umrechnung von Oktave und Ton in den genauen Aktutor, der diese Taste betätigt, hier geschehen.

% @TODO(Val): Rest schreiben...

\section{UI-Arduino Kommunikation} \label{vorgehenSW-SPPP}
\begin{enumerate}
    \item Wie wird Kommunikation mit Arduino i.d.R. gelöst
    \item Grundlagen der Kommunikation (Files, Polling, etc.)
    \item Prinzip der minimalen Arbeit → Custom Protocol → SPPP
    \item Asynchrone Umsetzung → Threading bei UI; Message-Buffer bei Arduino
\end{enumerate}

\section{UI} \label{vorgehenSW-UI}
\begin{enumerate}
    \item Grundlagen: Immediate vs. Retained Mode UI
    \item Cached Immediate Mode
    \item Generelle Diskussion bzgl. API-Design vllt
\end{enumerate}

\section{Parsing von MIDI} \label{vorgehenSW-MIDI}
\begin{enumerate}
    \item Überblick über Midi
    \item Vergleich MIDI vs PIDI
\end{enumerate}
(Kapitel vllt uninteressant?)