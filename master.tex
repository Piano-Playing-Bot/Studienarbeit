%%%%%%%%%%%%%%%%%%%%%%%%%%%%%%%%%%%%%%%%%%%%%%%%%%%%%%%%%%
%   Autoren des verwendeten Templates:
%   Prof. Dr. Bernhard Drabant
%   Prof. Dr. Dennis Pfisterer
%   Prof. Dr. Julian Reichwald
%%%%%%%%%%%%%%%%%%%%%%%%%%%%%%%%%%%%%%%%%%%%%%%%%%%%%%%%%%

\documentclass[fontsize=12pt,BCOR=5mm,DIV=12,parskip=half,listof=totoc,
               paper=a4,toc=bibliography,pointlessnumbers]{scrreprt}

               %toc=listof,listof=entryprefix,

\makeindex

%% Elementare Pakete, Konfigurationen und Definitionen werden geladen (gegebenenfalls anpassen)
% !TEX root =  master.tex

%%%%%%%%%%%%%%%%%%%%%%%%%%%%%%%%%%%%%%%%%%%%%%%%%%%%%%%%%%%%%%%%%%
%	ANLEITUNG:
% Passen Sie gegebenenfalls alle Stellen im Dokument an, die mit
% @stud
% markiert sind.
%%%%%%%%%%%%%%%%%%%%%%%%%%%%%%%%%%%%%%%%%%%%%%%%%%%%%%%%%%%%%%%%%%

%%
%% @stud
%%
%% LANGUAGE SETTINGS
%% Schriftarten- und Zeichenpakete
\usepackage[ngerman]{babel}				% german language
\usepackage[T1]{fontenc}				% For font encodings
\usepackage[utf8]{inputenc}				% For input encodings
\usepackage[german=quotes]{csquotes}	% correct quoting using \enquote{}

\usepackage{makeidx}					% allows index generation
\usepackage{listings}					% Format Listings properly
\usepackage{lipsum}						% Blindtext
\usepackage{graphicx}					% use various graphics formats
\usepackage[german]{varioref}			% nicer references \vref
\usepackage[format=plain]{caption}		% better Captions (format=plain to avoid hanging indentation)
\usepackage{booktabs}					% nicer Tabs
\usepackage[hidelinks=true]{hyperref}	% keine roten Markierungen bei Links
\usepackage{fnpct}						% Correct superscripts
\usepackage{calc}						% Used for extra space below footsepline, in particular
\usepackage{array}						% Better Array & Tabular environments
\usepackage{acronym}					% Allows using acronyms
\usepackage{algorithm}					% provides block command \algorithm
\usepackage{algpseudocode}				% Typesetting pseudocode
\usepackage{setspace}					% Allows setting line-spacing
\usepackage{tocloft}					% better table of contents, lists of figures/tables
\usepackage{tikz}						% Used for drawing directly in LaTeX
\usepackage{amsmath}					% Advanced Maths
\usepackage{multirow}

%% Table-Header
%% taken from: https://tex.stackexchange.com/a/102970
\newcommand*{\theadstart}[1]{\hline \multicolumn{1}{|c|}{\bfseries #1}}
\newcommand*{\theadcol}[1]{\multicolumn{1}{c|}{\bfseries #1}}

%%
%% Tikz Style Configuration
%%
\usetikzlibrary{shapes.geometric, arrows} % Use the tikz arrows library
\tikzstyle{rect} = [rectangle, rounded corners, text centered, draw=black, minimum width=2cm, minimum height=1cm]
% \tikzstyle{startstop} = [rectangle, rounded corners, minimum width=3cm, minimum height=1cm,text centered, draw=black, fill=red!30]
% \tikzstyle{io} = [trapezium, trapezium left angle=70, trapezium right angle=110, minimum width=3cm, minimum height=1cm, text centered, draw=black, fill=blue!30]
% \tikzstyle{process} = [rectangle, minimum width=3cm, minimum height=1cm, text centered, draw=black, fill=orange!30]
% \tikzstyle{decision} = [diamond, minimum width=3cm, minimum height=1cm, text centered, draw=black, fill=green!30]
\tikzstyle{arrow} = [thick,->,>=stealth]

%%
%% @stud
%%
%%	FONT SELECTION: Schriftarten und Schriftfamilie
%%%%%%%%%%%%%
%% SCHRIFTART
%%%%%%%%%%%%%
% 0) without decomment: normal font families
% ...
% 1) Latin Modern
%\usepackage{lmodern}
% 2) Times
%\usepackage{mathptmx}
% 3) Helvetica
%\usepackage[scaled=.92]{helvet}
%%%%%%%%%%%%%%%%%%
%%	SCHRIFTFAMILIE
%%%%%%%%%%%%%%%%%%
% ohne Serifen
\renewcommand*{\familydefault}{\sfdefault}
\addtokomafont{disposition}{\sffamily}
%
% mit Serifen
%\renewcommand*{\familydefault}{\rmdefault}
%\addtokomafont{disposition}{\rmfamily}
%
% Typewriter
%\renewcommand*{\familydefault}{\ttdefault}
%\addtokomafont{disposition}{\ttfamily}

%%
%% @stud
%%
%% Uncomment the following lines to support hard URL breaks in bibliography
%\apptocmd{\UrlBreaks}{\do\f\do\m}{}{}
%\setcounter{biburllcpenalty}{9000}% Kleinbuchstaben
%\setcounter{biburlucpenalty}{9000}% Großbuchstaben

%%
%% @stud
%%
%% FOOTNOTES: Count footnotes over chapters
%% \counterwithout{footnote}{chapter}

%	ACRONYMS
\makeatletter
\@ifpackagelater{acronym}{2015/03/20}
{\renewcommand*{\aclabelfont}[1]{\textbf{{\acsfont{#1}}}}}{}
\makeatother

%	LISTINGS
% @stud: ggf. Namen/Text anpassen (englisch)
\renewcommand{\lstlistingname}{Quelltext}
\renewcommand{\lstlistlistingname}{Quelltextverzeichnis}
\lstset{numbers=left,
	numberstyle=\tiny,
	captionpos=b,
	basicstyle=\ttfamily\small}

\definecolor{cGreen}{rgb}{0,0.6,0}
\definecolor{cGray}{rgb}{0.5,0.5,0.5}
\definecolor{cPurple}{rgb}{0.58,0,0.82}
\definecolor{cBackgroundColour}{rgb}{0.95,0.95,0.92}

\lstdefinestyle{CStyle}{
    backgroundcolor=\color{cBackgroundColour},
    commentstyle=\color{cGreen},
    keywordstyle=\color{magenta},
    numberstyle=\tiny\color{cGray},
    stringstyle=\color{cPurple},
    basicstyle=\footnotesize,
    breakatwhitespace=false,
    breaklines=true,
    captionpos=b,
    keepspaces=true,
    numbers=left, % @Note(Val): Ist es schöner mit Line-Numbers links vom Code oder sollten wir die ausstellen mit `numbers=none`?
    numbersep=5pt,
    showspaces=false,
    showstringspaces=false,
    showtabs=false,
    tabsize=2,
    language=C
}

% Code-Listing, that doesn't spill over page-breaks (see: https://tex.stackexchange.com/a/22889)
\lstnewenvironment{UnbrokenCodePage}[1][]%
{
   \noindent
   \minipage[htbp]{\linewidth}
   \vspace{0.5\baselineskip}
   \lstset{#1}}
{\endminipage}

%	ALGORITHMS
% @stud: ggf. Namen/Text anpassen (englisch)
\renewcommand{\listalgorithmname}{Algorithmenverzeichnis}
\floatname{algorithm}{Algorithmus}

%	PAGE HEADER / FOOTER
%	Warning: There are some redefinitions throughout the master.tex-file!  DON'T CHANGE THESE REDEFINITIONS!
\RequirePackage[automark]{scrlayer-scrpage}
%alternatively with separation lines: \RequirePackage[automark,headsepline,footsepline]{scrlayer-scrpage}

\renewcommand{\chaptermarkformat}{}
\RedeclareSectionCommand[beforeskip=0pt]{chapter}
\clearpairofpagestyles

%\ifoot[\rule{0pt}{\ht\strutbox+\dp\strutbox}DHBW Mannheim]{\rule{0pt}{\ht\strutbox+\dp\strutbox}DHBW Mannheim}
\ofoot[\rule{0pt}{\ht\strutbox+\dp\strutbox}\pagemark]{\rule{0pt}{\ht\strutbox+\dp\strutbox}\pagemark}
\ohead{\headmark}

\newcommand{\TitelDerArbeit}[1]{\def\DerTitelDerArbeit{#1}\hypersetup{pdftitle={#1}}}
\newcommand{\AutorDerArbeit}[1]{\def\DerAutorDerArbeit{#1}\hypersetup{pdfauthor={#1}}}
\newcommand{\Kurs}[1]{\def\DieKursbezeichnung{#1}}
\newcommand{\Studiengangsleiter}[1]{\def\DerStudiengangsleiter{#1}}
\newcommand{\WissBetreuer}[1]{\def\DerWissBetreuer{#1}}
\newcommand{\Bearbeitungszeitraum}[1]{\def\DerBearbeitungszeitraum{#1}}
\newcommand{\Abgabedatum}[1]{\def\DasAbgabedatum{#1}}
\newcommand{\Matrikelnummer}[1]{\def\DieMatrikelnummer{#1}}
\newcommand{\Studienrichtung}[1]{\def\DieStudienrichtung{#1}}
\newcommand{\ArtDerArbeit}[1]{\def\DieArtDerArbeit{#1}}
\newcommand{\Literaturverzeichnis}{Literaturverzeichnis}

\newcommand{\settingBibFootnoteCite}{
	\setlength{\bibparsep}{\parskip}		  % Add some space between biblatex entries in the bibliography
	\addbibresource{bibliography.bib}	    % Add file bibliography.bib as biblatex resource
	\DefineBibliographyStrings{ngerman}{andothers = {{et\,al\adddot}},}
}

\newcommand{\setTitlepage}{
	\input{titlepage}
	\pagenumbering{roman} % Römische Seitennummerierung
	\normalfont
}

\newcommand{\initializeText}{
	\clearpage
	\ihead{\chaptername~\thechapter} % Neue Header-Definition
	\pagenumbering{arabic}           % Arabische Seitenzahlen
}

\newcommand{\initializeBibliography}{
	\ihead{}
	\printbibliography[title=\Literaturverzeichnis]
	\cleardoublepage
}

\newcommand{\initializeAppendix}{
	\appendix
  \ihead{}
  \cftaddtitleline{toc}{chapter}{Anhang}{}
}

\makeatletter
\newcommand{\chapterauthor}[1]{%
  {\parindent0pt\vspace*{-25pt}%
  \linespread{1.1}#1%
  \par\nobreak\vspace*{30pt}}
  \@afterheading%
}
\makeatother


%%
%% @stud
%%
%% PERSÖNLICHE ANGABEN (BITTE VOLLSTÄNDIG EINGEBEN zwischen den Klammern: {...})
%%
\ArtDerArbeit{Studien}
\TitelDerArbeit{<Titel Ihrer Arbeit>}
\AutorDerArbeit{Jakob Kautz, Olivier Stenzel, Val Richter}
\Kurs{TINF21AI1}
\Studienrichtung{Angewandte Informatik}
\Matrikelnummer{<Martikelnummern>}
\Studiengangsleiter{Prof. Dr. Holger D. Hofmann}
\WissBetreuer{Prof. Dr. Eckhardt Kruse}
\Bearbeitungszeitraum{01.10.2023 -- 16.04.2024}
\Abgabedatum{dd.mm.yyyy}

%%
%% @stud
%%
%% BIBLIOGRAPHY (@stud: Bibliographie-Stil wählen - Position und Indizierung)
%%  Auswahl zwischen:
%%   NUMERIC Style
%%   IEEE Style
%%   ALPHABETIC Style
%%   HARVARD Style
%%   CHICAGO Style
%%   (oder eigenen zulässigen Stil wählen)
%%
%%%%%%%%%%%%%
%% Zitierstil
%%%%%%%%%%%%%
% NUMERIC Style - e. g. [12]
%\newcommand{\indextype}{numeric}
%
% IEEE Style - numeric kind of style
%\newcommand{\indextype}{ieee}
%
% ALPHABETIC Style - e. g. [AB12]
\newcommand{\indextype}{alphabetic}
%
% HARVARD Style
%\newcommand{\indextype}{apa}
%
% CHICAGO Style
%\newcommand{\indextype}{authoryear}
%
%%%%%%%%%%%%%%%%%%%%%%
%% Position des Zitats
%%%%%%%%%%%%%%%%%%%%%%
\newcommand{\position}{inline}
%
% (!!) FOOTNOTE POSITION NOT RECOMMENDED IN MINT DOMAIN:
%\newcommand{\position}{footnote}

%% Final: Setzen des Zitierstils und der Zitatposition
\usepackage[backend=biber, autocite=\position, style=\indextype]{biblatex}
\usepackage{enumerate}
\usepackage{cleveref}
\usepackage{glossaries}
\settingBibFootnoteCite

%%
%% Definitionen und Commands
%%
\newcommand{\abs}{\par\vskip 0.2cm\goodbreak\noindent}
\newcommand{\nl}{\par\noindent}
\newcommand{\mcl}[1]{\mathcal{#1}}
\newcommand{\nowrite}[1]{}
\newcommand{\NN}{{\mathbb N}}
\newcommand{\imagedir}{img}
\DeclareRobustCommand{\micro}{\mu}

\makeindex

\begin{document}

\setTitlepage
\addcontentsline{toc}{chapter}{Titelseite}

%%%%%%%%%%%%%%%%%%%%%%%%%%%%%%%%%%%
% EHRENWÖRTLICHE ERKLÄRUNG
%
% @stud: ewerkl.tex bearbeiten
%
\input{ewerkl}
\addcontentsline{toc}{chapter}{Ehrenwörtliche Erklärung}
\cleardoublepage
%%%%%%%%%%%%%%%%%%%%%%%%%%%%%%%%%%%

%%%%%%%%%%%%%%%%%%%%%%%%%%%%%%%%%%%
% VERZEICHNISSE und ABSTRACT
%
% @stud: ggf. nicht benötigte Verzeichnisse auskommentieren/löschen
%
\tableofcontents
\addcontentsline{toc}{chapter}{Inhaltsverzeichnis}
\cleardoublepage

% Abbildungsverzeichnis
\phantomsection
\addcontentsline{toc}{chapter}{\listfigurename}
\listoffigures
\cleardoublepage

%	Tabellenverzeichnis
\phantomsection
\addcontentsline{toc}{chapter}{\listtablename}
\listoftables
\cleardoublepage

%	Listingsverzeichnis / Quelltextverzeichnis
\lstlistoflistings
\cleardoublepage

% Algorithmenverzeichnis
\listofalgorithms
\cleardoublepage

% Abkürzungsverzeichnis
% @stud: abkürzungen.tex bearbeiten
% !TEX root =  master.tex
\clearpage
\chapter*{Abkürzungsverzeichnis}
\addcontentsline{toc}{chapter}{Abkürzungsverzeichnis}

\begin{acronym}[XXXXXXX]
	\acro{UI}{User Interface}
	\acro{MC}[$\micro$C]{Mikrocontroller}
	\acro{PWM}{Pulsweitenmodulation }
	\acro{Vcc}{Voltage at the common collector}
	\acro{PIDI}{PIano Digital Interface Format}
	\acro{PDIL}{Piano Digital Interface Library Format}
	\acro{SPPP}{Self Playing Piano Protocol}
	\acro{MIDI}{Musical Instrument Digital Interface}
	\acro{I2C}{Inter-Integrated-Circuit}
	\acro{SPC}{Serial Peripheral Interface}
	\acro{UART}{Universal Asynchronous Receiver-Transmitter}
	\acro{SPPP}{Self-Playing Piano Protocol}
	\acro{SAM}{Self-Applying Musician}
	\acro{HTTP}{Hypertext Transfer Protocol}
	\acro{FFI}{Foreign Function Interface}
	\acro{I/O}{Input/Output}
	\acro{WMP}{Windows Media Player}
\end{acronym}
\cleardoublepage

% 1,5 Zeilenabstand schon ab dem Abstrakt
\onehalfspacing

%	Kurzfassung / Abstract
% @stud: abstract.tex bearbeiten
%%%%%%%%%%%%%%%%%%%%%%%%%%%%%%%%%%%%%%%%%%%%%%%%%%%%%%%%%%
%   Autoren des Abschnitts:
%   ???
%%%%%%%%%%%%%%%%%%%%%%%%%%%%%%%%%%%%%%%%%%%%%%%%%%%%%%%%%%

% !TEX root =  master.tex
\chapter*{Kurzfassung (Abstract)}
\addcontentsline{toc}{chapter}{Kurzfassung (Abstract)}

\paragraph*{Deutsch:}

Hier können Sie die Kurzfassung (engl.~Abstract) der Arbeit schreiben. Beachten Sie dabei die Hinweise zum Verfassen der Kurzfassung.

\paragraph*{Englisch:}

The same abstract in english.

\cleardoublepage

%%%%%%%%%%%%%%%%%%%%%%%%%%%%%%%%%%%%%%%%%%%%%%%%%%%%%%%%%%%%%%%%%%%%%%%%%%%%%%%%%%%%%%%%%%
% KAPITEL UND ANHÄNGE
%
% @stud:
%   - nicht benötigte: auskommentieren/löschen
%   - neue: bei Bedarf hinzufügen mittels input-Kommando an entsprechender Stelle einfügen
%%%%%%%%%%%%%%%%%%%%%%%%%%%%%%%%%%%%%%%%%%%%%%%%%%%%%%%%%%%%%%%%%%%%%%%%%%%%%%%%%%%%%%%%%%

\initializeText

%%%%%%%%%%%%%%%%%%%%%%%%%%%%%%%%%%%
% KAPITEL
%
% @stud: einzelne Kapitel bearbeiten und eigene Kapitel hier einfügen
%%%%%%%%%%%%%%%%%%%%%%%%%%%%%%%%%%%%%%%%%%%%%%%%%%%%%%%%%%
%   Autoren des Abschnitts:
%   Olivier Stenzel
%%%%%%%%%%%%%%%%%%%%%%%%%%%%%%%%%%%%%%%%%%%%%%%%%%%%%%%%%%

% !TEX root =  master.tex
\chapter{Einleitung} \label{einleitung}
\chapterauthor{Olivier Stenzel}

\nocite{*}

Im Rahmen dieser Studienarbeit wird ein selbstspielendes Klavier konzipiert und gebaut.
Klaviermusik findet in zahlreichen Umgebungen ihren Einsatz: von der Untermalung in Hotel-Lobbys durch Bar-Pianist:innen bis hin zu privaten Haushalten,
wo sie zur Atmosphäre und Unterhaltung beiträgt.
Vor diesem Hintergrund könnte ein selbstspielendes Klavier, speziell in Bereichen, wo traditionellerweise Live-Musiker:innen engagiert werden,
einen signifikanten Mehrwert darstellen.

Obwohl die Technologie der selbstspielenden Klaviere bereits etabliert ist, bleiben die Kosten für solche Instrumente mit Preisen,
die leicht 10.000 Euro überschreiten können, eine beträchtliche Hürde \cite*{YamahaU1}.
Diese finanzielle Barriere limitiert den Zugang zu dieser Technologie für eine breite Nutzerbasis erheblich.
Angesichts dieser Preisgestaltung ist eine wesentliche Motivation dieser Arbeit die Entwicklung einer deutlich preisgünsigeren Alternative.
Das Ziel ist es, ein Instrument zu konzipieren, das die wesentlichen Funktionalitäten seines professionellen Pendants beibehält,
jedoch zu einem Bruchteil der Kosten realisiert werden kann.
Durch dieses Vorhaben sollen selbstspielende Klaviere einem breiteren Publikum zugänglich gemacht werden.




% !TEX root =  master.tex
\chapter{Zielstellung} \label{Zielstellung}

\nocite{*}

Die Idee der Arbeit besteht ist die Entwicklung eines Klaviers, sowohl von Personen, als auch automatisiert von einem Computer bespielt werden kann.
Hierfür wird ein kleiner Computer eingesetzt, um Hardware anzusteuern, die in der Lage ist, die Klaviertasten zu betätigen.
Die Steuerung dieses selbstspielenden Klaviers erfolgt über eine
Desktop-Anwendung, mittels derer Nutzer:innen zwischen Musikstücken wählen können.
Darüber hinaus soll die Möglichkeit bestehen, dem Katalog weitere Musikstücke hinzuzufügen.
Des Weiteren sollen Nutzer:innen die Wiedergabe-Geschwindigkeit -und Lautstärke verändern können.
Inspiriert von gängigen Musikplayern, soll das Spielen pausiert und zu beliebigen Stellen des Stücks gesprungen werden können.


\section{Vorgehensweise} \label{sec:zielstellung-vorgehen}
Die Vorgehensweise des Projekts gliedert sich in mehrere Schlüsselschritte.
Zunächst erfolgt die Konzeptualisierung des selbstspielenden Klaviers, wobei die spezifischen Anforderungen und Funktionalitäten
definiert werden.//TODO: noch An die Struktur anpassen!!!!
Hierbei liegt ein besonderer Fokus auf der Identifikation der Schnittstellen zwischen der
Hardware und Software, wobei die Arduino-Plattform als zentrales Steuerungselement berücksichtigt wird. \newline

Im Anschluss erfolgt die Hardware-Implementierung mithilfe des Arduinos. Dies beinhaltet die sorgfältige
Auswahl geeigneter Hardwarekomponenten, die in der Lage sind, die Klaviertasten präzise anzusteuern.
Die Programmierung des Arduinos erfolgt mit dem Ziel, eine nahtlose Integration in das Gesamtsystem zu
gewährleisten. \newline

Die nachfolgende Etappe konzentriert sich auf die Entwicklung einer
Desktop-Anwendung, die als Schnittstelle für die Steuerung des selbstspielenden Klaviers dient. Hierbei wird
besonderes Augenmerk auf die Benutzerfreundlichkeit gelegt, und die Anwendung ermöglicht Nutzer:innen die
Auswahl und Wiedergabe von Musikstücken aus einem vordefinierten Katalog. \newline

Ein weiterer Schwerpunkt liegt auf der Integration von MIDI-Dateien, um dem Katalog kontinuierlich weitere
Musikstücke hinzufügen zu können.
Dieser Prozess beinhaltet die Entwicklung einer Schnittstelle, die eine
unkomplizierte Integration neuer Musikstücke in die bestehende Datenbank ermöglicht.
%%%%%%%%%%%%%%%%%%%%%%%%%%%%%%%%%%%%%%%%%%%%%%%%%%%%%%%%%%
%   Autoren des Abschnitts:
%   Jakob Kautz
%   Olivier Stenzel
%%%%%%%%%%%%%%%%%%%%%%%%%%%%%%%%%%%%%%%%%%%%%%%%%%%%%%%%%%

% !TEX root =  master.tex
\graphicspath{ {./img/} }

\chapter{Konzeption - Hardware}\label{konzeptionHW}
\chapterauthor{Jakob Kautz, Olivier Stenzel}
% @Note(Val): Abschnitt fängt im Präteritum an und geht dann mit Präsenz weiter
% Glaube, dass Präsenz leichter lesbar ist, aber muss halt einfach einheitlich sei
% @Note(Val): Der ganze Abschnitt argumentiert falsch herum.
% Ihr redet von Aktuatoren, Mikrocontrollern, etc. ohne vorher zu erklären, warum die überhaupt benötigt werden.
% Statt Lösungen zu liefern, sollte hier erstmal erklärt werden, welche Probleme gelöst werden müssen
Für die Umsetzung eines selbstspielenden Klaviers werden verschiedene technische Überlegungen angestellt.
Die Grundfragen, welche bei der Konstruktion auftreten, betreffen die Methode des Anspielens der Klaviertasten, die
Signalübertragung des \ac{MC}s, sowie die Konzeption der Schaltung.
Folgende Anforderungen muss die Konzeption abdecken:
\begin{enumerate}
	\item Ansteuerungskonzept: Die Tasten müssen möglichst effizient von der Hardware angespielt werden, dies beinhaltet Positionierung der Aktuatoren und die Verbindung zwischen Hardware und Klavier.
	\item Auswahl der Aktuatoren: Die Aktuatoren \ref{subsec:aktuator} müssen eine präzise Steuerung der Tasten ermöglichen, ohne dass Akkuratheit und Geschwindigkeit des Anspielens der Tasten vernachlässigt wird
	\item Auswahl des \ac{MC}: Die Aktuatoren benötigen zur Ansteuerung der Klaviertasten die Signale der Software, wobei die Kommunikation via \ac{MC} geschieht.
\end{enumerate}
Die Überlegungen und Entscheidungen für die genutzte Hardware sowie den Aufbau dieser werden in diesem Kapitel erläutert.
Die Hardware-Komponente des Projektes wird im folgenden auch als \enquote{Piano Player} bezeichnet.

\section{Mechanik}\label{konzeptionHW-mechanik}

\subsection{Auswahl des Klaviers}

\chapterauthor{Olivier Stenzel}

Da sich dieses Projekt nicht auf die theoretische Konzeption beschränkt, muss ein entsprechendes Testobjekt - ein reales Klavier - gefunden werden.
Dieses muss einige Voraussetzungen erfüllen:
\begin{enumerate}
	\item 	Der Preis muss im Rahmen des selbst gestellten Budgets (< 2000 €) liegen
	\item 	Es muss leicht auseinander zu bauen sein, damit die Mechanik zugänglich ist
	\item 	Es muss vollständig sein (alle 88 Tasten)
	\item 	Es muss stimmbar sein
	\item 	Der Transport muss einfach durchzuführen sein
\end{enumerate}

Mittels dieser Anforderungen wurde online ein passendes Klavier für 100€ ersteigert.

\subsection{Ansteuerungskonzept} \label{subsec:konzeptionhw-ansteuerungskonzept}

\chapterauthor{Jakob Kautz}

Das Ansteuerungskonzept bezieht sich auf die Problemstellung, dass die Saiten des Klaviers auf irgendeine Art zum
schwingen gebracht werden müssen.
Dafür ist eine Vorrichtung hilfreich, die elektrische Signale in Bewegung umwandeln kann.
Dies ist die Eigenschaft von Aktuatoren, wie z.b. Motoren.

In diesem Kapitel wird untersucht, wie genau man das Anspielen der Saiten technisch umsetzen kann.
Dabei ist eine grundlegende Überlegung, ob man die Saiten direkt anschlägt,
oder die bestehenden Tasten weiter dafür nutzt.

Wenn die Aktuatoren direkt die Saiten des Klaviers anspielen, erfordert dies weniger Kraft im Vergleich zum
Anspielen der Tasten. Das liegt daran, dass die Tasten eine größere mechanische Übersetzung bieten, um die Saiten
anzuschlagen. Wenn die Aktuatoren direkt auf die Saiten wirken, müssen sie nur die erforderliche Kraft aufbringen, um
die Saiten in Schwingung zu versetzen, was im Allgemeinen weniger Kraft erfordert als das Drücken einer Taste mit
ausreichend Kraft, um den Hammer gegen die Saiten zu schlagen.

Die Verwendung von weniger kraftaufwendigen Aktuatoren kann die Gesamtkosten des Systems senken. Aktuatoren, die weniger
Kraft erzeugen müssen, sind oft einfacher und kostengünstiger herzustellen (und zu kaufen) und erfordern möglicherweise weniger
energieintensive Komponenten. Darüber hinaus kann die Verwendung von leistungsschwächeren
Aktuatoren die Größe und das Gewicht des Systems verringern, was zusätzliche Vorteile hinsichtlich Kosten, Transport und
Montage bieten kann. \newline
Allerdings geht dabei die gesamte Mechanik des Klaviers verloren, was bedeutet, dass Aspekte wie Dämpfung nicht genutzt werden können,
was wiederum zu einem weniger ansprechenden Klang führt. Zusätzlich müsste das Klavier permanent geöffnet bleiben, und die
Aktuatoren müssten äußerst präzise die Saiten anschlagen, um akzeptable Ergebnisse zu erzielen.
% @Note(Val): Mit Außnahme des letzten Punkts verstehe ich die Argumente hier ehrlich gesagt nicht. Vieleicht kenne ich mich einfach zu wenig mit Klavieren aus, um das nachvollziehen zu können und wir setzen offensichtlch ein bisschen Klavier-Wissen voraus, aber fals hier für jeden Punkt ein Nebensatz hin könnte, warum das so ist, wäre das wahrscheinlich besser
% @Question(Jay): Ist das so sinnvoller/nachvollziehbarer?
Da der zweite Punkt - ein ansprechender Klang - für dieses Projekt eine höhere Priorität annimmt, fiel die Entscheidung
darauf, die Tasten anzuspielen. Somit kann die Klaviermechanik genutzt werden, was für einen
authentischeren Klang sorgt. \newline
Nun stellt sich also noch die Frage, wie genau die Aktuatoren mit den Tasten des Klaviers verbunden werden.

Eine Möglichkeit besteht darin, die Aktuatoren oben über den Tasten anzubringen, entweder in Form einer nachgebildeten
\enquote{Klavierhand} mit zehn \enquote{Fingern} oder in Form einer Schiene mit 88 Aktuatoren auf den Tasten.
Beide Lösungen bringen allerdings das Problem mit sich, dass sie gegen die in \ref{sec:zielstellung-anforderungen} definierte Anforderung
der Flexibilität verstoßen, nämlich dass das Klavier sowohl automatisch als auch manuell bespielbar sein soll.\newline
Außerdem würde die Klavierhand-Option - die dem tatsächlichen Klavierspiel ähnlich sieht -
aufgrund der Bewegung und Präzision eine komplexere Logik und
Montage erfordern. Die Schienenoption bietet hier eine viel einfachere Ansteuerung, benötigt jedoch einen Aktuator für jede Taste.
Außerdem steht diese Variante im Gegensatz zu der in Kapitel \ref{sec:zielstellung-anforderungen} Anforderung, dass der Aufbau für unterschiedliche Klaviere anpassbar
sein soll. Das liegt daran, dass diese Schiene eine fest vorgegebene Länge hätte, welche für Klaviere welche eine
andere größe haben, nicht mehr passen würde. \newline

Eine Alternative hierzu besteht darin, die Aktuatoren unter den Tasten anzubringen, wodurch die Anforderung der Flexibilität
im Gegensatz zur ersten Option nicht beeinflusst wird.\newline
Das Klavier kann also sowohl vom \enquote{Piano Player} als auch von einem menschlichen Spieler gleichzeitig bedient werden.
Da diese Anforderung eine hohe Priorität hat, wird die Ansteuerung von untern erfolgen. \newline
Grundsätzlich gibt es hierbei zwei Möglichkeiten für die Ansteuerung:
\begin{enumerate}
	\item Ziehen der Tasten
	\item Drücken der Tasten
\end{enumerate}
Bei beiden Varianten wird jede Taste mit einem Aktuator
ausgestattet, es werden also 88 Aktuatoren benötigt, was die Kosten % @Question(Jay) soll ich späteres Kapitel zu Kosten referenzieren oder so lassen?
im Gegensatz zu der von oben Spielenden \enquote{Klavierhand}
erhöht.
Generell ist die Drückoption ästhetisch ansprechender, da die Hardware sehr einfach versteckt werden könnte.
Somit würde die Illusion entstehen, dass die Tasten sich von alleine bewegen. Die Option erfordert jedoch eine hohe Präzision und könnte die
Tastenempfindlichkeit beeinträchtigen. Außerdem wäre aufgrund der präzisen Montage der Aktuatoren die Anforderung
der Anpassbarkeit (siehe Kapitel \ref{sec:zielstellung-anforderungen}) komplexer umzusetzen.

Für das Drücken der Tasten fallen noch weitere Probleme an. Um die Tasten drücken zu können, muss ein möglichst großer
Hebel aufgebracht werden, damit die Aktuatoren möglichst wenig Kraft für das anspielen aufbrauchen müssen.
Dieser Punkt ist besonders wichtig, wenn der \enquote{Piano Player} mit unterschiedlichen Lautstärken und Dynamiken spielen soll.
% @Question(Jay): Ist es offensichtlich dass das daran liegt dass wir dann auch leiser spielen können oder soll ich das noch erwähnen?
Dies führt dazu, dass die Aktuatoren soweit hinten wie möglich angebracht werden müssen. % @Note(Jay) weil größter Hebel ist offensichtlich, oder?
\newline
Hier tritt das Problem auf:
Der Holraum, der sich im hinteren Teil des ausgewählten Klaviers befindet, ist nicht durchgängig. Daher ist es
erforderlich, Löcher einzubauen. Diese müssen räumlich so platziert werden, dass sie keine wichtigen Teile der Klaviermechanik
stören. Aufgrund des Klavierbaus bedeutet dies, dass die Löcher etwa im mittleren Drittel der Taste angebracht werden müssen.
Dies führt zu einer verringerten Hebelkraft. Das Problem betrifft nur etwa ein Viertel der Tasten. Die restlichen könnten mit
voller Hebelkraft von ganz hinten angespielt werden. Allerdings würde dies dazu führen, dass die Tasten
\begin{enumerate}
	\item unterschiedlich stark angespielt werden und der Klang somit seltsam ist,
	\item der Anstoß der Tasten genormt wird und somit nicht die gesamte Dynamik der hinten angebrachten Aktuatoren genutzt
	wird, damit alle gleich klingen. Dies würde die Software komplexer machen, da dies berücksichtigt werden muss, um eine
	konsistente Leistung zu gewährleisten.
\end{enumerate}
Zur Vereinfachung wäre es daher sinnvoller, alle Tasten von der gleichen Höhe aus anzuspielen. Dies bedeutet, dass alle
Tasten von weiter vorne angespielt werden und somit ein Teil der möglichen Dynamik verloren geht. \newline

Diese Probleme fallen beim Ziehen der Tasten weg beziehungsweise sind weniger ausgeprägt. % @Note(Val): Klingt ein bisschen repetetiv mit den vorherigen Sätzen, inhaltlich aber ok
Bei der Präzision liegt der Grund dafür darin, dass die Taste nicht
direkt getroffen werden muss. Stattdessen wird lediglich ein Seil (siehe Kapitel \ref{subsec:VerbindungTastenAktuatoren})
gezogen, das den Aktuator dazu veranlasst, die Taste
gerade nach unten zu ziehen.
Da das Ziehen des Seils weniger präzise Kontrolle erfordert
und keine direkte Interaktion mit der Tastenmechanik erfordert, kann es auch die Tastenempfindlichkeit weniger beeinträchtigen
als das Drücken der Tasten. Es ist allerdings zu beachten, dass das ziehen der Seile - im Gegensatz zum drücken - keine
optimalen Ergebnisse im Bereich des Präzisen spielens bringen. Dies hat mehrere Gründe.
\begin{enumerate}
	\item Verzögerung: Wenn das Seil sich über die Zeit lockert, kann es sein, dass das betätigen der Taste nicht direkt mit der Bewegung des Aktuators kommt. Außerdem würde die Taste so nicht komplett gedrückt werden.
	\item Verschleiß: Die Bewegung über Führsysteme oder Umlenkrollen, welche für eine gerade Bewegung der Seile benötigt werden, können zu Reibungsverlust führen. Außerdem könnten die Seile sich durch verschleiß lockern oder verschieben.
\end{enumerate}
Letztendlich handelt es sich bei beiden Problemen um Gebrauchserscheinungen, welche durch eine gute Auswahl von Seilen
und Wartungen am Klavier wenn es öfter in Benutzung ist minimiert werden können.
Die Hebelkraft ist ebenfalls ein geringeres Problem, da die Aktuatoren Problemlos am vordersten Punkt der Tasten
befästigt werden können, da hier - im Gegensatz zum hinteren Teil des Klaviers - nichts im Weg ist.

Insgesamt fiel die Entscheidung deshalb auf die Strategie des Ziehens von unten.
Bei diesem Ansatz wird die geringste mechanische Präzision benötigt und das Klavier ist trotz Montage des Bots noch manuell spielbar.

\section{Elektronik}\label{sec:konzeptionhw-elektronik}

\subsection{Mikrocontroller}\label{Ansteuerung}
\chapterauthor{Jakob Kautz}
Um die Signale des Programmes an die Aktuatoren weitergeben zu können, ist ein Mikrocontroller (später: \ac{MC}) gut geeignet.
Auf dem Markt steht eine hohe Anzahl an \ac{MC}s zur Verfügung,
wobei für diese Arbeit ein Arduino und ein Rasperry Pi\footnote{Genau genommen handled es sich bei einem Rasperry Pi um einen Einplatinencontroller.} betrachtet wurden.

%Arduino Intro
	\paragraph{Arduino}
	Bei einem Arduino handelt es sich um eine Plattform für die Entwicklung von elektronischen Prototypen. % @Note(Val): Ist ein Arduino wirklich eine "Platform"? Das klingt irgendwie komisch...
	Er besteht aus einem \ac{MC}-Board, das mit verschiedenen Sensoren, Aktuatoren und anderen elektronischen Komponenten verbunden werden kann.
	Der Arduino verfügt über digitale und analoge Ein- und Ausgangspins, die für die Interaktion mit Geräten verwendet werden können.
	\begin{figure}[htbp]
		\centering
		\includegraphics [width=8cm] {img/ArduinoR3}
		\caption{Arduino Uno}
		\label{img:Arduino}
	\end{figure}
\newline
%Pi Intro

\paragraph{Rasperry Pi}
Ein Rasperry Pi ist ein Einplatinencontroller, der auf einem ARM-Prozessor basiert.
Er ist dafür konzipiert, eine breite Palette von Anwendungen zu unterstützen, vom Prototypenbau bis zu IoT-Geräten.
Im Gegensatz zum Arduino ist er besonders für komplexe und rechenlastige Projekte geeignet.
\begin{figure}[htbp]
	\centering
	\includegraphics [width=8cm] {img/RasperryPi}
	\caption{Rasperry Pi}
	\label{img:Raspi}
\end{figure}
\newline

%Vergleich
Es wurde sich letztendlich für den Arduino entschieden.
Der Rasperry Pi stand vor allem aufgrund seiner höhere Speicherkapazität und Rechenleistung zur Diskussion, wodurch eine komplexere Ansteuerungslogik möglich wäre.
Allerdings ist er aufgrund seines Betriebssystems nicht für Echtzeit-Anwendungen bzw. geschwindigkeitskritische Anwendungen ausgelegt.
% @Note(Jay): Vllt noch genauer
Im Gegensatz dazu bietet der Arduino eine Echtzeitverarbeitung mit geringer Latenz.
Außerdem verfügt ein Arduino über eine recht simple Hardware-Interaktion - er ist darauf spezialisiert, die Hardware direkt anzusprechen - und ist somit besser für Projekte geeignet, welche eine präzise Ansteuerung von Aktuatoren benötigt. % @Note(Val): Simpel != Präzise. Nur weil der Arduino eine simple Hardware-Interaktion hat, heißt es nicht, dass er die Hardware präzise ansteueren könnte
Zudem sollte die Ansteuerungslogik nicht besonders komplex sein, weshalb ein Arduino genug Rechenleistung aufweisen können sollte.

Für das Projekt wird spezifisch ein Arduino R3\footnote{Der Arduino R3 wird auch als \enquote{Mega 2563} bezeichnet.} verwendet.
Bei der Auswahl des spezifischen Arduinos wurden der Arduino Uno, Nano und R3 betrachtet.
Im Prinzip wäre jedes der genannten Modelle für die Anwendung möglich, allerdings verfügt der R3 im Gegensatz zu den anderen beiden Modellen über mehr Speicherkapazitäten (256KB Flash Speicher
gegenüber der 32KB vom Arduino Uno und den 16KB des Arduino Nanos). Der gewählte Arduino verfügt auch über einen CPU-Kern.
Der Arduino Uno verfügt zwar über mehr Ports, allerdings bringt dies keinen Mehrwert, da die Anzahl der benötigten Ausgänge auch bei einem Arduino Uno nicht erreicht werden
(siehe Kapitel \ref{output}).
Der Arduino R3 bietet - wie die anderen beiden Modelle auch - \ac{PWM} Pins, die im Rahmen des Projekts genutzt werden können.

\subsection{Pulsweitenmodulation}\label{PWM}
\chapterauthor{Jakob Kautz}
Zur Vollständigkeit wird in diesem Abschnitt das Prinzip der Pulsweitenmodulation erläutert.
Oftmals ist bei der Ansteuerung der Aktuatoren nicht die gesamte Versorgungsspannung erwünscht bzw. benötigt.
In diesen Fällen muss die anliegende Spannung varriiert werden, damit die Spannung am Ziel dem gewünschten Wert entspricht.
Angenommen es ist eine Spannung von 2.5V erwünscht, wobei die Vollversorgungsspannung 5V beträgt.
Das Signal kann dafür die Hälfte der Zeit ausgeschaltet werden, womit zwischen 0V und den vollen 5V durchschnittlich insgesamt 2.5V anliegen.
Je höher die Frequenz zwischen An- und Ausschalten des Signals eingestellt wird, desto weniger wird diese \enquote{künstlich} simulierte Halbierung wahrgenommen.

\ac{PWM} bedient sich im Grunde genau dieser Technik.
Dabei wird die Zeitdauer eines digitalen Signals variiert wird, um einen durchschnittlichen Wert zu erzeugen.
Bei den \ac{PWM}-Ausgängen wird die Pulsweite - die Dauer der Einschaltzeit - des Signals angepasst, um die gewünschte Spannung zu erreichen.

\begin{figure}[htbp]
	\centering
	\includegraphics [width=13cm, height=8cm] {img/pulsweite}
	% @Note(Val): Quelle des Bilds fehlt hier noch
	\caption{Pulsweitenmodulation}
	\label{fig:pulsweite}
\end{figure}

Spezifischer ausgedrück passiert folgendes:
Eine digitale Steuerung wird verwendet, um eine Rechteckwelle - ein Signal das zwischen Ein und Aus umgeschaltet wird - zu erzeugen.
Dieses Ein-Aus-Muster kann Spannungen zwischen der vollen Versorgungsspannung und 0V simulieren.
Dabei wird der Anteil der Zeit geändert, für den das Signal eingeschaltet ist relativ zur Zeit, in der das Signal ausgeschaltet ist.
Um unterschiedliche, analoge Werte zu erhalten, wird die Pulsweite moduliert.
Wenn dieses Ein-Aus-Muster zum Beispiel schnell genug mit einer LED wiederholt wird, resultiert daraus eine konstante Spannung zwischen 0 und Vcc, die die Helligkeit der LED steuert.

Auf den Arduino bezogen sieht die Umsetzung eines \ac{PWM} Signals wie folgt aus:
Ein Taktsignal gelangt in die entsprechende Clock.
Die Clock stellt den entsprechenden \ac{PWM}-Modus ein.
Dabei werden zwei wichtige Werte gesetzt:
Der erste bestimmt, wann das Signal von HIGH auf LOW umschaltet, während der zweite bestimmt, wann es zurückkommt.
Das Verhältnis zwischen HIGH und LOW wird als Tastverhältnis bezeichnet und bestimmt die Helligkeit der LED bzw. die Stärke, mit der der Hubmagnet anschlägt.
Je länger die Ausgabe im HIGH-Zustand bleibt, desto schneller erfolgt der Tastenanschlag.

Neben dem Tastverhältnis, also dem Verhältnis der Einschaltzeit zur Periodendauer, welches oft in Prozent ausgedrückt wird, ist auch die Auflösung ein variierbarer Parameter.
Die Auflösung bezieht sich auf die Anzahl der möglichen diskreten Werte, die das Signal annehmen kann.


\subsection{Vermehrung der Ausgänge}\label{output}
\chapterauthor{Jakob Kautz}
Da ein Klavier über 88 Tasten verfügt, müssen 88 Aktuatoren angesteuert werden. Der gewählte Arduino hat keine 88 \ac{PWM}-Ports, daher
müssen die Signale über eine Erweiterung der Ausgänge an die Motoren weitergegeben werden. Dafür gibt es mehrere Möglichkeiten,
wobei in dieser Arbeit 2 im Detail betrachtet werden:

\begin{enumerate}
	\item Schieberegister
	\item Aktuator-Matrix
\end{enumerate}

% @Note(Jay): Wäre es smart zu erwähnen was es noch für Möglichkeiten gibt? Also Arduino Mega muss ich soweiso erwähnen, aber z.B
% @Note(Jay): Shields verbinden oder mehrere Arduinos nutzen weil das haben wir ja kurz überlegt und dann relativ schnell verworfen

\subsubsection{Schieberegister}
Ein Schieberegister ist ein integrierter Schaltkreis, der zur Speicherung und sequenziellen Verschiebung von
Datenbits verwendet wird.\newline
Das grundlegende Prinzip eines Schieberegisters ist, dass Datenbits seriell in das Register eingegeben und dann
sequenziell aus dem Register ausgegeben werden können.
Dies geschieht durch die Verwendung von Taktimpulsen, die das Verschieben der Bits steuern.\newline
Schieberegister bestehen aus einer Reihe von Flip-Flops.
Ein Flip-Flop kann als ein einfacher Speicher betrachtet werden, der binäre Informationen speichert und je nach
Eingangssignal den Zustand 1, 0, oder einen \enquote{Flip}-Zustand annimmt.
Diese FlipFlops sind so verbunden, dass sie Daten in einer bestimmten Reihenfolge speichern und weitergeben können.

Um 88 Ausgänge mit Hilfe von Schieberegistern, die von nur 3 \ac{PWM}-Pins des Arduinos agesteuert werden, umzusetzen,
können 11 8-Bit-Schieberegister  - also Schieberegister mit 8 Ausgängen - verwendet werden.

\paragraph{74HC959 Schieberegister}
In dieser Arbeit wird spezifisch ein 74HC959 Schieberegister betrachtet.
Dieses hat folgenden Aufbau:
\begin{figure}[htbp]
\begin{minipage}{0.4\textwidth}

		\includegraphics [width=1\textwidth] {img/Schieberegister}
		\caption{Schematik Schieberegister}
		\label{img:Shift}

\end{minipage}
\begin{minipage}{0.6\textwidth}
	\begin{enumerate}
		\item Q0-Q7: Ausgänge (parallelgeschaltet)
		\item Vcc: Anschluss Versorgungsspannung (+)
		\item Gnd: Anschluss Versorgugsspannung (-)
		\item DS: Data Signal, serieller Dateneingang
		\item OE: Output Enable, zur Aktivierung der Ausgänge
		\item SHCP: Shift Clock, Clock-Eingang zur Übernahme Data Signals in Schieberegister
		\item STCP: Store Clock, Beim Wechsel von LOW auf HIgh wird der Inhalt des Schieberegisters in das Ausgaberegister kopiert % @Note(Val): Selbe Bezeichnung wie oben verwenden. Entweder LOW/HIGH oder AN/AUS
		\item MR: Master Reset, Leerung des Schift-Registers
		\item Q7S: Überlauf für Kaskadierung
	\end{enumerate}
\end{minipage}
\end{figure}

\subsubsection{Aktuator-Matrix}
Die Aktuator-Matrix ist einer LED-Matrix nachgeahmt.
In einer Matrix, werden zwei Reihen nach folgendem Muster an Ports angeschlossen:
$$
\begin{pmatrix}
	(11) & (12) & (13) & (14) & (15) & (16) & (17) & (18) \\
	(21) & (22) & (23) & (24) & (25) & (26) & (27) & (28) \\
	(31) & (32) & (33) & (34) & (35) & (36) & (37) & (38) \\
	(41) & (42) & (43) & (44) & (45) & (46) & (47) & (48) \\
	(51) & (52) & (53) & (54) & (55) & (56) & (57) & (58) \\
	(61) & (62) & (63) & (64) & (65) & (66) & (67) & (68) \\
	(71) & (72) & (73) & (74) & (75) & (76) & (77) & (78) \\
	(81) & (82) & (83) & (84) & (85) & (86) & (87) & (88)
\end{pmatrix}
$$

Eine LED-Matrix besteht aus einer Anordnung von LEDs in Zeilen und Spalten.
Jede LED kann unabhängig von den anderen ein- oder ausgeschaltet werden.
Die Steuerung der Matrix kann sowohl mit als auch ohne Multiplexing erfolgen.
Prinzipiell wird jede Zeile der Matrix nacheinander aktiviert, während die
entsprechenden LEDs in den Spalten gleichzeitig eingeschaltet werden.
Durch schnelles Wechseln zwischen den Zeilen mithilfe von \ac{PWM}-Signalen erscheint es den Betrachter:nnen, als ob alle LEDs
gleichzeitig leuchten würden, obwohl sie tatsächlich nacheinander aktiviert werden.

\begin{figure}[htbp]
	\centering
	\includegraphics[width=0.45\textwidth]{img/LED-Matrix}
	\caption{Beispielhafte Darstellung einer LED-Matrix}
	\label{img:LED-Matrix}
\end{figure}

%ohne Multiplexer
\paragraph{Matrix ohne Multiplexer}

Bei einer LED-Matrix ohne Multiplexer werden die LEDs direkt über die \ac{GPIO}-Pins des
\ac{MC}s angesteuert. Jede LED ist einzeln mit einem Pin des \ac{MC}s verbunden. Um die LEDs anzusteuern,
muss der \ac{MC} jeden Pin einzeln aktivieren oder deaktivieren, um die entsprechende LED ein- oder auszuschalten.
Hierbei werden viele Pins benötigt, um die gesamte Matrix anzusteuern, was besonders bei größeren Matrizen unpraktisch
sein kann. Für dieses Projekt, wird wie oben aufgeführt, eine 8x8-Matrix benötigt. Es werden also 16 PWM-Fähige Pins
benötigt. Der hier verwendete \ac{MC} verfügt allerdings nur über 6 von diesen Pins.

\begin{figure}[htbp]
	\centering
	\includegraphics[width=0.6\textwidth]{img/LEDMatrixSchaltung}
	\caption{Matrix ohne Multiplexer Schematisch}
	\label{fig:Matrix}
\end{figure}
% @Note(Jay): Welches der Bilder?
%Um die Komplexität darzustellen, eine Aktuator-Matrix ohne Multiplexer würde für neun Aktuatoren wie folgt aussehen:
%\begin{figure}[htbp]
%	\centering
%	\includegraphics[width=0.9\textwidth]{img/AktMatrix}
%	\caption{Aktuator-Matrix ohne Multiplexer}
%	\label{fig:AktMatrix}
%\end{figure}
%mit Multiplexer
%Was ist ein Multiplexer, warum hilft er bei Martix wie sieht allg Schaltung aus?
\paragraph{Matrix mit Multiplexer}
Ein Multiplexer (bzw. ein Demultiplexer) ermöglicht das kombinieren mehrerer Signale zu einem bzw. dem Trennen eines einzelnen Signals zu mehreren, unterschiedlichen Signalen.
Bei einer LED-Matrix erfolgt das Prinzip durch die Verwendung von Schieberegistern und Transistoren.
Somit kann damit die Notwendigkeit zum Verbinden jedes individuellen Pins mit einem GPIO-Pin des \ac{MC}s elimniert werden.
Entsprechend werden weniger Pins benötigt, um die Matrix anzusteuern.
Die Nutzung von Multiplexern bringt somit eine effizientere Ressourcennutzung mit sich, kommt allerdings auch mit einer komplexeren Schaltung und Latenzen durch die Schieberegister einher. % @Note(Val): Sind die Latenzen nur "möglich"? Müsste es nicht entweder klar Latenzen erhöhen oder nicht?
Wobei die Latenzen der Schieberegister in diesem Projekt vernachlässigt werden können.
\begin{figure}[htbp]
	\centering
	\includegraphics[width=0.75\textwidth]{img/matrixMuxSchaltung}
	\caption{Aktuator-Matrix mit Multiplexing}
	\label{fig:AktMatrixMux}
\end{figure}

\subsubsection{Entscheidungsfindung}
Letztendlich fiel die Entscheidung auf die Verwendung von Schieberegistern. Dies liegt an mehreren Gründen:
\begin{enumerate} % @Note(Jay): Ersten Punkt sinnvoller schreiben pls der doppelt sich und ist nicht so kohärent
	\item Interferenz und ungenaue Steuerung durch die Matrixstruktur:  Eine LED-Matrix wird, wie bereits erklärt,
	typischerweise durch \ac{PWM} angesteuert, um den Eindruck zu erwecken, dass alle
	LEDs gleichzeitig leuchten, obwohl sie tatsächlich nacheinander aktiviert werden. Dieses Prinzip funktioniert gut für LEDs,
	da sie einen langsamen Reaktionsmechanismus haben und das menschliche Auge träge ist, bei Aktuatoren, welche möglichst dynamisch und Präzise angespielt werden sollen, ist dieses Prinzip nicht gewinnbringend.
	Bei der Verwendung einer Aktuatoren-Matrix über \ac{PWM}
	kann es zu Interferenzen kommen, die die Präzision der Steuerung beeinträchtigen. Da die \ac{PWM}-Signale zeilenweise
	multiplexiert werden, kann es vorkommen, dass die \ac{PWM}-Signale nicht mit der benötigten Stärke für jeden Aktuator geliefert
	werden. Das liegt daran, dass verschiedene Aktuatoren in derselben Zeile unterschiedliche \ac{PWM}-Signale benötigen können,
	was zu ungenauer Steuerung führt. Dies kann zu Kompromissen bei der Präzision und Leistung der Aktuatoren führen. % @Note(Val): Verstehe ich nicht. Jede LED auf der Matrix kann doch unabhängig der anderen angesteuert werden und andere Signale erhalten. Sonst wäre das ja eh nicht nutzbar für uns.
	% @Note(Val): Wieder viele Konjunktive. Entweder es ist ein Problem oder nicht.
	% @Note(Jay): Die LED-Matrix basiert auf trägheit des menschlichen Auges, weswegen dass da kein Problem ist. Bei uns würde das spielen halt trotzdem klappen, aber unsere Präzision für Akkuratheit wär eingeschrenkt.
	\item Komplexität der Schaltung bei selbstgebauter Aktuator-Matrix: Die Implementierung einer Aktuator-Matrix über eine
	LED-Matrix hinaus wäre technisch anspruchsvoller und erfordert eine komplexere Schaltung. Im Gegensatz dazu ist die
	Verwendung von Schieberegistern eine einfachere Methode zur Ansteuerung der Aktuatoren.
	\item Anzahl der Pins: Wie bereits erklärt, benötigt die Matrix ohne Multiplexer eine Vielzahl an Pins, welche der \ac{MC} % @Note(Val): Dieser letzte Punkt ist valide aber können wir mMn auch auslassen, wenn wir genug andere Punkte haben
	nicht liefert. Da die Matrix verwendet werden sollte um eben dieses Problem zu lösen, schien es nicht sinnvoll
	das Prinzip einzusetzen. % @Note(Val): Aber mit Multiplexer braucht es doch nicht mehr so viele Pins, oder? Vielleicht konkret berechnen wie viele Pins dann benötigt wären und sagen, dass es immer noch mehr Pins sind als beim Schieberegister?
	% TODO(Jay): mit würden wir 3 Schieberegister brauchen - entweder schieben, oder bräuchten 9 Ausgänge und wenn wir schieben können wir direkt die andere Variante nutzen weils dann für SW einfacher ist
	\item Ressourcen: Desweiteren basiert die Entscheidung auf unzureichenden Ressourcen für die Matrix. Im Internet gab es
	keine ausreichenden Anleitungen/Ressourcen, die die Implementierung einer LED-Matrix mit Aktuatoren ausreichend
	unterstützen würden. Zusätzlich dazu lieferten die Simulationen via Tinkercad\footnote{Tinkercad: https://www.tinkercad.com} keine befriedigenden Ergebnisse
	für die Aktuator-Matrix.
	Im Gegensatz dazu sind Schieberegister gut dokumentiert und es gibt ausreichende Ressourcen, um ihre
	Verwendung für die Ansteuerung von Aktuatoren zu verstehen und umzusetzen.
\end{enumerate}

\subsection{Transistor}
\chapterauthor{Jakob Kautz}
Ein Transistor ist ein elektronisches Bauteil, das in der Lage ist, den Stromfluss zwischen zwei seiner Anschlüsse
(den sogenannten Source und Drain) mithilfe eines dritten Anschlusses (der Gate) zu steuern. Es gibt verschiedene Arten
von Transistoren, wobei grob zwischen Transistor und MOSFET unterschieden wird.
Die grundlegenden Prinzipien ihrer Funktionsweise sind allerdings ähnlich.

\begin{figure}[htbp]
	\centering
	\includegraphics[width=0.45\textwidth]{img/Mosfet}
	\caption{Aufbau MOSFET}
	\label{img:transistor}
\end{figure}

Funktionsweise eines Transistors:
Ein Transistor besteht typischerweise aus einem Halbleitermaterial wie Silizium und kann in drei Haupttypen unterteilt
werden: Bipolartransistor (NPN und PNP) und MOSFET (Metalloxid-Halbleiter-Feldeffekttransistor).  \newline
Bipolartransistor: Die Steuerung des Stromflusses erfolgt durch die Änderung des Stroms, der in die Basis des
Transistors fließt, was den Stromfluss zwischen Emitter und Collector beeinflusst. \newline
%Note(Jay) soll ich Emitter und Collector noch erklären?
MOSFET: Bei einem MOSFET wird der Stromfluss zwischen Source und Drain durch das Anlegen einer Spannung an das Gate
gesteuert, wodurch ein elektrisches Feld im Halbleiter erzeugt wird, das den Stromfluss reguliert. \newline

Unterschied zwischen Transistor und MOSFET: \newline
Ein MOSFET ist eine spezielle Art von Transistor, der auf einem anderen Prinzip basiert als der Bipolartransistor. Der
Hauptunterschied liegt in der Steuerung des Stromflusses: Während beim Bipolartransistor der Strom durch die Basis
gesteuert wird, erfolgt die Steuerung beim MOSFET durch das Anlegen einer Spannung an das Gate.
Bei n-MOSFETs (negativ dotierte MOSFETs, welche in diesem Projekt verwendet wurden) wird durch das Anlegen einer
positiven Spannung am Gate ein
elektrisches Feld erzeugt, das die Leitfähigkeit zwischen Source und Drain beeinflusst. Dies führt dazu, dass der n-MOSFET
in einem bestimmten Bereich des Gate-Spannungsbereichs als Schalter oder Verstärker arbeiten kann. \newline

Verwendung von Transistoren in Aktuatoren-Schaltungen:
Transistoren werden in Aktuatoren-Schaltungen verwendet, um die Aktuatoren
zu steuern. Sie dienen als Schalter, der den Stromfluss zu den Aktuatoren regelt.
Dabei ist es wichtig, auf die Durchlassspannung und Mindestspannung des MOSFETs zu achten. Die Durchlassspannung ist die
minimale Spannung, die an das Gate angelegt werden muss, um den MOSFET in den leitenden Zustand zu versetzen. Die
Mindestspannung ist die minimale Spannung, die zwischen Source und Gate anliegen muss, um den MOSFET zuverlässig zu
sperren.
Es ist entscheidend, sicherzustellen, dass die Spannungen in der Schaltung diese Anforderungen erfüllen, um
eine ordnungsgemäße Funktion des MOSFETs sicherzustellen und Schäden zu vermeiden.

\subsection{Aktuator}\label{subsec:aktuator}
\chapterauthor{Jakob Kautz}
Die Aktuatoren werden für die Steuerung der Tasten benötigt.
Ein Aktuator wandelt Energie in Bewegung (oder andere physikalische Größen) um, um eine gewünschte Aktion herbeizuführen
oder einen Mechanismus zu steuern. \newline % @TODO(Val): Quelle für Aktuator-Definition
Bei den möglichen Aktuatoren für die Ansteuerung der Klaviertasten wurden drei Möglichkeiten betrachtet: % @Note(Val): Falls es einen Grund gibt, nur diese 3 zu betrachten, gerne nennen, ansonsten sollte das auch so fine sein, glaube ich
\begin{enumerate}
	\item Linearmotor % @Note(Jay): Ich glaub es wär smart hier Elektromotor zu sagen und dann zu unterteilen
	\item Servomotor % TODO(Jay): Lin. Servomotor
	\item Hubmagnet % TODO(Jay): Welcher genau und wie viel Kraft
\end{enumerate}

% @TODO(Val): Quelle für jede Definition der 3 Aktuator-Typen
\paragraph{Linearmotor}
Ein Linearmotor ist ein Aktuator, der eine lineare Bewegung erzeugt. Er besteht aus einer festen Spule und einem
beweglichen Magneten oder umgekehrt. Wenn Strom durch
die Spule fließt, erzeugt sie % @Question(Val): Ist es richtig zu sagen, dass die Spule ein Magnetfeld erzeugt? (Ich hab physikalisch keine Ahnung, also kann gut sein, dass das so richtig formuliert ist)
ein Magnetfeld, das den beweglichen Teil des Motors in eine lineare Bewegung zieht. \newline
Physikalische Grundlagen:
\begin{itemize}
	\item Die Funktionsweise eines Linearmotors beruht auf dem Prinzip der elektromagnetischen Induktion. Wenn Strom durch die
	Spule fließt, erzeugt sie ein Magnetfeld, das den beweglichen Magneten anzieht oder abstößt, je nach Polarität des Stroms.
	\item Die Richtung und Geschwindigkeit der Bewegung des Linearmotors hängt von der Stärke und Richtung des angelegten Stroms
	sowie von der Geometrie des Motors ab.
\end{itemize}
\begin{figure}[htbp]
	\centering
	\includegraphics[width=5cm]{img/Linearmotor}
	\caption{Linearmotor}
	\label{fig:Linearmotor}
\end{figure}

\paragraph{Servomotor}
% @Note(Val): Hier scheint etwas falsch zu sein. Bei der Entscheidungsfindung später sagt ihr, dass Servomotoren eine rotierende Bewegung erzeugen. Aber hier sagt ihr, dass es mit einer Drehung anfängt und diese dann vom Servomotor in eine Bewegung (ich nehme an lineare Bewegung) umgewandelt wird.
Ein Servomotor ist ein elektromechanischer Aktuator, der eine Bewegung durch eine Drehung ermöglicht.
Er besteht aus einem Elektromotor, einem Getriebe und einem Feedback-Mechanismus. Der Elektromotor erzeugt eine rotierende
Bewegung, die durch das Getriebe in eine Bewegung umgewandelt wird. Der
Feedback-Mechanismus, oft ein Potentiometer oder ein Encoder, misst die genaue Position des Motors und gibt diese
Informationen an die Steuerung zurück.\newline % @Note(Val): Verstehe nicht inwieweit der Feedback-Mechanismus sinnvoll ist

Physikalische Grundlagen:
\begin{itemize}
	\item Der Elektromotor nutzt das Prinzip der elektromagnetischen Induktion, um eine Drehbewegung zu erzeugen. Dies geschieht
	durch das Anlegen einer Spannung an die Spulen des Motors, die ein magnetisches Feld erzeugen und den Rotor des Motors in
	Bewegung versetzen.
	\item Das Getriebe dient dazu, die Geschwindigkeit und das Drehmoment des Motors zu regulieren und die Bewegung an die
	Anforderungen der Anwendung anzupassen.
	\item Der Feedback-Mechanismus misst die Position des Motors durch Erfassung von Änderungen des magnetischen Feldes oder der
	Winkelposition des Rotors und ermöglicht so eine präzise Steuerung.
\end{itemize}
\begin{figure}[htbp]
	\centering
	\includegraphics[width=5cm]{img/Servomotor}
	\caption{Servomotor}
	\label{fig:Servomotor}
\end{figure}

\paragraph{Hubmagnet}
Ein Hubmagnet (auch Solenoid genannt) ist ein Aktuator, der eine lineare Bewegung durch das Anlegen eines elektrischen
Stroms erzeugt. Er besteht aus einer Spule, die einen magnetischen Kern umgibt. Wenn Strom durch die Spule fließt, erzeugt
sie ein Magnetfeld, das den Kern anzieht und somit eine lineare Bewegung erzeugt.\newline
Physikalische Grundlagen:
\begin{itemize}
	\item Der Hubmagnet nutzt das Prinzip der elektromagnetischen Induktion, um eine lineare Bewegung zu erzeugen. Wenn Strom
	durch die Spule fließt, erzeugt sie ein Magnetfeld, das den magnetischen Kern anzieht. Je nach Polarität des Stroms bewegt
	sich der Kern entweder in Richtung der Spule oder von ihr weg.
	\item Die Stärke der Bewegung des Hubmagneten hängt von der Stärke des Stroms, der Anzahl der Wicklungen und der
	Spule des Magneten ab.
\end{itemize}
\begin{figure}[htbp]
	\centering
	\includegraphics[width=5cm]{img/Hubmagnet}
	\caption{Hubmagnet}
	\label{fig:Hubmagnet}
\end{figure}

Ein großer Vorteil von Hubmagneten ist es, dass beim Anschließen nicht darauf geachtet werden muss, wie rum die
Kabel verbunden werden, weil es
sich um elektromagnetische Komponenten handelt, die auf Basis des elektromagnetischen Prinzips arbeiten. Das bedeutet,
dass der Stromfluss durch den Hubmagneten erzeugt wird, um ein Magnetfeld zu erzeugen, das die Bewegung oder das Halten
von Gegenständen ermöglicht.
Die Polung ist aufgrund der Symmetrie des Magnetfeldes egal:
Das Magnetfeld, das durch den Stromfluss im Hubmagneten erzeugt wird, ist in der Regel
symmetrisch um die Mittelachse des Magneten. Das bedeutet, dass sich die magnetische Kraft gleichmäßig um den Magneten herum
verteilt, unabhängig davon, welche Polung verwendet wird. Daher spielt es keine Rolle, welcher Anschluss an Plus oder Minus
angeschlossen wird, da das resultierende Magnetfeld ähnlich sein wird.

%TODO(Jay): QUELLE

\paragraph{Entscheidungsfindung}
Prinzipiell kann jeder der genannten Aktuatoren für das Projekt verwendet werden. Die Entscheidung fiel aufgrund
mehrerer Vortzeile allerdings auf den Hubmagneten:
\begin{enumerate}
	\item Servo-Motor: Bei diesem Motor ist eine Bewegungsbeschränkung mit eingebaut, welche in jedem Motor ausgebaut
	werden müsste. Dies ist zu umständlich, wenn es bessere Varianten - wie den Hubmagneten - gibt. Außerdem hätte
	eine Umlenkung stattfinden müssen, da die erzeugte Bewegung eine Rotation im Gegensatz zu einer linearen Bewegung
	ist. Zwar gibt es auch lineare Servomotoren, welche allerdings ebenfalls aufgrund ihrer Bewegungsbeschränkung nicht
	weiter betrachtet wurden. % @Note(Val): Warum erwähnt ihr die rotierende Bewegung dann überhaupt als Nachteil. Und warum werden vorher bei der Erklärung nicht die linearen Servomotoren erklärt?
	% TODO(Jay): fix this with lin. Motor cuz i actually didn't think about that.
	Zusätzlich verweilen Servomotoren nach ihrer Ansteuerung an der Position, an der
	sie bei Ende der Stromversorgung waren. Für das Klavier werden Aktuatoren benötigt, welche wieder auf die vorherige
	Position zurückschalten, die Klaviertaste also in dem Sinne wieder \enquote{loslassen} können.
	\item Linear-Motor: Auch hier kam das Problem auf, dass die Motoren nicht automatisch wieder an ihre vorherige
	Position zurückschalten, wenn der Stromfluss stoppt.
	% @Question(Val): Das Problem könnte mit einem gegenteiligen Stromfluss gelöst werden, oder? Weiß aber nicht ob das die Schaltung unbedingt verkomplizieren würde. Wenn das überhaupt nicht möglich wäre, wären Servo- & Linear-Motoren ja gar nicht nutzbar und dann wäre die Einleitung für diesen Abschnitt falsch
	% @Note(Jay): Ja, es geht eher darum dass es unnötiger Aufwand wäre das abzufragen wenn Solenoids es automatisch machen
	\item Hubmagnete: Hubmagnete sind schneller als die anderen beiden Aktuatoren, was insbesondere bei schnellem
	Tastendrücken sinnvoll ist. Außerdem schalten Sie, wenn die Stromversorgung abbricht, wieder auf ihre vorherige
	Position zurück.
\end{enumerate}


\subsection{Schaltplan} \label{subsec:schaltplan}
\chapterauthor{Olivier Stenzel, Jakob Kautz}

Der Schaltplan besteht im Großteil aus den bereits aufgeführten Komponenten:
\begin{enumerate}
	\item Arduino
	\item Schieberegister
	\item MOSFET
	\item Hubmagnet
\end{enumerate}
In diesem Kapitel wird erläutert, wie diese Komponenten miteinander verbunden werden um eine Funktionsfähige
Schaltung zu erstellen. Die gesamte Schaltung mit allen 88 Hubmagneten wird wie folgt aussehen:
\begin{figure}[htbp]
	\centering
	\includegraphics [width=0.85\textwidth] {img/SchaltungGesamt}
	\caption{Überblick Schaltplan}
	\label{img:Schaltplan}
\end{figure}
\subsubsection{Arduino}

Im Zentrum der Schaltung steht der \ac{MC} (hier: Arduino Uno R3).
Dieser erhält Daten und Strom über den integrierten USB-Anschluss, welcher mit dem Computer verbunden wird.
Da der Arduino limitierte \nameref{PWM}-fähige Ausgänge bereitstellt, werde Schieberegister (74HC595) verwendet.
Mit jedem \enquote{in Reihe} geschaltetem Schieberegister kann die Anzahl \ac{PWM}-fähiger Ausgänge um 8 erweitert werden.

\subsubsection{Schieberegister}

Der Arduino wird an fünf Stellen mit dem ersten Schieberegister verbunden:

Arduinoport D2 <-> Serial (SER) Input

Über diese Verbindung werden serielle Daten werden hier bitweise in das Register geschoben.

Arduinoport D3 <-> SHCP (Shift Register Clock Input)

Dieser Pin wird verwendet, um den Takt für das Verschieben der Daten innerhalb des Schieberegisters anzulegen.
Bei jedem Taktimpuls auf diesem Pin wird das Bit am seriellen Dateneingang in das Register verschoben.
Das bedeutet, dass bei jeder steigenden Flanke des Taktsignals das Datenbit, das am Eingang anliegt, in das Schieberegister übernommen und alle vorhandenen Daten um eine Position verschoben werden.

Arduinoport D4 <-> STCP (Storage Register Clock Input)

Nachdem die Daten in das Schieberegister eingelesen wurden, wird dieser Pin verwendet, um die im Schieberegister vorhandenen Daten in das Ausgangsregister zu übertragen.
Ein Taktimpuls auf diesem Pin bewirkt, dass die Daten vom Schieberegister ins Ausgangsregister übernommen werden, sodass alle Ausgänge gleichzeitig aktualisiert werden.
Das ist besonders relevant, da sonst unter Umständen alle Ausgänge von einer Änderung im letzten Schieberegister betroffen wären.

Arduino GND <-> Ground, Output Enable (OE)

Der OE-Pin wird genutzt, um die Ausgänge des Schieberegisters global zu aktivieren oder zu deaktivieren, ohne die Daten selbst zu beeinflussen.
Da das Schieberegister zu keiner Zeit deaktiviert sein soll, wird dieser Pin dauerhaft mit dem GND-Pin verbunden.

Arduino VCC 5V <-> VCC, $\overline{SRCLR}$ (Reset)

Um das Schiebregister mit den benötigten 5V zu betreiben, wird der entsprechende Pin mit dem 5V Output des Arduino verbunden.
Zusätzlich wird der $\overline{ }$ SRCLR Port des Schieberegisters, welcher ein Reset ermöglicht dauerhaft mit 5V verbunden.

Jedes weiteres Schieberegister greift die oben genannten Signale ab.
Der einzige Unterschied befindet sich am Serial (SER) Input Port.
Das Schieberegister an Position i+1 wird mit dem seriellen Output des Schieberegisters an Position i verbunden. ($\forall i = 0,...,10$)

\begin{figure}[htbp]
	\centering
	\includegraphics[width=0.9\textwidth]{img/SchaltungSchieberegister}
	\caption{Schaltung der Schieberegister}
	\label{fig:Shifting}
\end{figure}

\subsubsection{MOSFET}

Die Hubmagnete werden jeweils mit 24V und mit bis zu 400mA betrieben.
Um einen hohen Stromfluss zu steuern, können Transistoren verwendet werden.
Für hohe Spannungen und schnelle Schaltvorgänge eigenen sich besonders MOSFETs (Metall-Oxid-Halbleiter-Feldeffekttransistor).
Im Folgenden werden speziell n-MOSFETs verwendet, der mit einem Signal zwischen 0V (leitet nicht) und +5V (voll leitend) angesteuert werden kann.

Der folgende Aufbau ist für die insgesamt 88 Ausgänge der 11 Schieberegister identisch, da jeder Ausgang für die Ansteuerung genau eines Motors zuständig ist.

Der GATE-Pin des MOSFETs erhält das Signal, dass die \enquote{``Durchlässigkeit''} steuert aus einem der Outputs des Schieberegisters.
Der SOURCE-Pin wird mit Ground des gesamten Systems verbunden.
Der DRAIN-Pin wird direkt mit dem entsprechenden Kontakt am Hubmagneten verbunden.

\begin{figure}[htbp]
	\centering
	\includegraphics[width=0.9\textwidth]{img/MosSchaltung}
	\caption{Schaltung der Mosfets}
	\label{fig:SchaltungMosfet}
\end{figure}

\subsubsection{Hubmagnet}

Um den Stromkreis zu schließen wird der andere Kontakt des Hubmagnetes mit dem +24V verbunden.
Bei Hubmagneten ist es in der Regel egal, welcher Anschluss an Plus und welcher an Minus angeschlossen wird
(siehe Kapitel \ref{subsec:aktuator}).

\subsubsection{Testen}

Um die Fehlersuche zu erleichtern, werden LEDs in den Schaltplan mit eingebaut.
Diese werden jeweils mit einem entsprechenden $1k\Omega$ Widerstand parallel zu den Motoren angeschlossen.
So kann anhand der Helligkeit der LED die Intensität abgelesen werden, mit der eine Taste gespielt wird.
\newline

Für insgesamt 4 Hubmagnete (Wobei in dem Schema Motoren gewählt wurden, da das Programm keine Hubmagnete als Komponenten
zur Verfügung stellt) und 2 Schieberegister sieht die Darstellung des Schaltplans Schematisch wie folgt aus:

\begin{figure}[htbp]
	\centering
	\includegraphics[width=0.9\textwidth]{img/SchematischeSchaltungExp}
	\caption{Beispielhaftes Schema des Schaltplans}
	\label{img:SchaltungExpSchema}
\end{figure}



%%%%%%%%%%%%%%%%%%%%%%%%%%%%%%%%%%%%%%%%%%%%%%%%%%%%%%%%%%
%   Autoren des Abschnitts:
%   Jakob Kautz
%   Olivier Stenzel
%%%%%%%%%%%%%%%%%%%%%%%%%%%%%%%%%%%%%%%%%%%%%%%%%%%%%%%%%%

% !TEX root =  master.tex
\chapter{Umsetzung - Hardware} \label{umsetzungHW}
\chapterauthor{Jakob Kautz, Olivier Stenzel}

\nocite{*}
- Probleme, Schwierigkeiten, Änderungen während der Umsetzung

Nachdem die Planungsphase abgeschlossen war, mussten die Überlegungen umgesetzt werden.
Im Rahmen dieser Arbeit wurde ein Prototyp gebaut, welcher 8 Tasten anspielen kann.
Zusätzlich wurde die Elektronik mit LEDs so erweitert, dass man das Drücken von 40 Tasten simulieren kann. % @Note(Val): Entweder hier oder an späterer Stelle erwähnen, dass es an sich trivial aber halt zeitaufwendig wäre, mehr Tasten anspielbar zu machen
% @Note(Val): Ich würde hier vielleicht noch gar nicht erwähnen, dass nur 8 Tasten anspielbar sind, weil die ganze Umsetzung ja mit dem Ziel 88 Tasten anspielbar zu machen lief. Kostenschätzung, etc. sind deshalb ja auch höher. Vielleicht ist es besser hier zu sagen, dass das Ziel war, den Prototypen mit möglichst vielen spielbaren Tasten zu gestalten, aber nur 8 bis dato erledigt wurden

\section{Materialien}
\chapterauthor{Jakob Kautz}
\subsection{Liste der Bauteile}
%TODO(Jay): Fix List
\begin{table}[htbp]
    \centering
    \begin{tabular}{|m{3.8cm}|m{1.7cm}|m{8cm}|}
        \hline
        \textbf{Bauteil} &  \textbf{Anzahl} & \textbf{Begründung}  \\
        \hline
        Hubmagnete & 88 & jede Taste braucht einen Hubmagneten um angespielt zu werden \\ % @Note(Val): Erwähnen, dass nur 8 der Hubmagnete benötigt wurden
        \hline
        Stromversorgung & 1 & Externe Stromversorgung für die Aktuatoren, da diese mehr als die 5V Vcc des Arduinos brauchen \\
        \hline
        Arduino & 1 & Kommunikation \\
        \hline
        Breadboard & 2 & Kleinstromschaltung \\
        \hline
        Schaltplatine & 5 & Großstromschaltung\\ % @Note(Jay): Heißt das so?
        \hline
        Schieberegister & 11 & Weitergabe Signal\\
        \hline
        Kabel (10cm) & 352stck (bzw. $9\cdot40$ in Packs) & Verbindungen in der Schaltung mit Schätzung 4 Kabel pro Hubmagnet\\
        \hline
        Kabel (20cm) & 176stck (bzw.$5\cdot40$ in Packs) & Verbindungen zu den Hubmagneten \\
        \hline
        LEDs & 88 & Tests \\
        \hline
        1kOhm Widerstände & 90 & Sicherheit and shit \\
        \hline
        MOSFET & 90 & Steuerung Strom \\
        \hline
        Feste Anschlussblöcke & 88 & Anschluss von Schaltplatine zu Hubmagnet\\
        \hline
        Angelschnur (1m) & 88 & Verbindung Hubmagnet und Taste \\
        \hline
    \end{tabular}
    \caption{Ergebnisse der Anforderungen}
    \label{table:Bauteile}
\end{table}

\subsection{Kostenübernahme}
Die vorher spezifizierte Hardware für den Schaltplan musste für die Erstellung des Prototypen offensichtlich besorgt werden.
Aufgrund der relativ hohen Kosten für eine Studienarbeit, wurde bei einer der betreuenden Firmen angefragt, ob diese die Kosten für das Projekt übernehmen würde.
Damit dies möglich war, wurde ein Kostenvoranschlag gestellt, in welchem die benötigten Materialien mit den geschätzten Kosten aufgeführt wurden. % @TODO(Val): Kostenanschlag wird wahrscheinlich eine Tabelle, also referenzier die einfach "(siehe Tabelle \ref{...})" oder so

% @TODO(Jay): Add Kostenschätzung

Der Kostenanschlag erwies sich im Laufe des Projektes als (teils) unrealistisch. Dies lag insbesondere an der Anforderung
der Firma. Die Schätzung der Kosten basierte auf Anbietern, bei welchen die Materialien möglichst günstig zu kaufen sind.
Durch Firmenreglungen mussten diese allerdings alle bei Conrad oder Reichelt
% @Note(Jay): Stimmt das so?
gekauft werden. Diese Anbieter verkaufen die Materialien für sehr viel mehr Geld. Die tatsächlichen Kosten liefen
letztendlich also auf folgende Beträge hinaus: \newline % @TODO(Val): Hier am besten auch einfach wieder nur die Tabelle referenzieren, statt sie unbedingt in die nächste Zeile zu quetschen
% @TODO(Jay): Tatsächliche Kosten

Der Kostenunterschied betrug daher insgesamt .
% @TODO(Jay): Add difference
Hierbei ist allerdings zu erwähnen, dass die Kostenschätzung passend gewesen wäre, wenn die Anbieter frei wählbar wären.

\section{Prototypenbau}
Ursprünglich sollte der Aufbau des in Kapitel \ref{subsec:schaltplan} spezifizierten Schaltplans via Steckbrettern und
Jumperkabeln umgesetzt werden.
Zu Beginn wurde dies auch so umgesetzt. Das Problem welches dadurch entstand, war, dass die gewählten Platinen den
benötigten Stromfluss nicht aushalten.\newline
Jeder Aktuator - also jede gedrückte Taste - zieht einen Strom von 0.7A. Um das Projekt möglichst sinnvoll umzusetzen,
sollte der Aufbau mindestens 10 Tasten gleichzeitig drücken können, was bedeutet, dass der Aufbau mindestens 7.0A
Stromfluss problemlos ausstehen muss.
% @TODO(Jay): Wie viel halten Platinen aus? Warum konnten wir ein paar Jumper-Kabel nutzen? Wann brauchten wir die dickeren? Tabelle für welche Kabeldicke für welchen Stromfluss
% @Note(Val): Wann haben wir entschieden, dass min. 10 Tasten gleichzeitig spielbar sein sollen? Wenn wir das als Anforderung haben wollen, muss das im Anforderungs-Kapitel auch erwähnt werden

Aus diesem Grund wurde die gesamte Schaltung, die nach dem Schieberegister kommt, auf einer Lochrasterplatine fest gelötet.
Hierfür wurde ein 3mm starker Draht für die Stromversorgung verwendet.

\subsection{Verbindung Tasten und Aktuatoren} \label{subsec:VerbindungTastenAktuatoren}
\chapterauthor{Olivier Stenzel}

Wie in Kapitel \ref{subsec:konzeptionhw-ansteuerungskonzept2} beschrieben wurde sich für ein Ziehen der Tasten entschieden.
Dieses Ziehen wird technisch mittels Hubmagneten umgesetzt. \newline
Dieses Kapitel beschreibt wie und wo die Hubmagnete am Klavier und den Tasten befestigt werden. % @Note(Val): Beschreibst du nur was gemacht wurde? Dann gibt es keine Entscheidungen hier & dann ist das nicht wirklich Konzipierung, oder?
Zusätzlich wird der Aufbau bezüglich Reibung und Akkuratheit beim Ansteuerns der Tasten weiter verbessert. % @Note(Val): "verbessert"? Es wurde doch noch gar nichts umgesetzt. Du meinst der Aufbau wird detailierter ausgearbeitet oder?
% @Note(Val) Es heißt Akkuratesse statt Akkuratheit (https://www.duden.de/rechtschreibung/Akkuratesse) aber bessere Wörter wären vll Exaktheit oder Genauigket
\newline
Die Grundidee ist, eine Schnur, oder Ähnliches, mittels eines waagerechten Loches in der Taste, an dieser zu befestigen. % @Note(Val): Erst Problem beschreiben, dann Lösung nennen. Das Problem wurde davor angesprochen und sollte hier mindestens erwähnt oder gar erklärt werden
Die Position an der Taste kann nicht frei gewählt werden. Es muss darauf geachtet werden,
dass man das Loch beim Spielen nicht sieht. % @Note(Val): Warum ist es wichtig, dass es nicht gesehen wird? Klingt eher als wäre das ein optionaler Bonus
Außerdem ist wichtig, dass das Loch möglichst am Ende der Taste angebracht wird, um einen größeren Hebeleffekt zu erzeugen,
und dass das Tastenbrett unter der Taste an der Position gut durchzubohren ist.
In Abbildung \ref{img:Tastenbohrung} sieht man mit Rot gekennzeichnet das gebohrte Loch, durch das ein Seil (in grün dargestellt) geführt wird.

\begin{figure}[htbp]
    \centering
    \includegraphics[width=8cm]{img/Taste_schraeg.jpg}
    \caption{Tastenbohrung}
    \label{img:Tastenbohrung}
\end{figure}


% @Note(Val): Der Satz ist mit den vielen Einschüben und Klammern schwer zu lesen. Lieber in 2 Sätze teilen und lesbarer machen
Anschließend wird jede Schnur jeweils durch ein senkrechtes Loch (rot in Abbildung \ref{fig:klaviatur} gekennzeichnet) im Klaviaturbalken (bzw. Tastenbrett), % @Note(Val): Nur einen Begriff einheitlich verwenden, sonst verwirrt es nur
worauf die Tasten liegen, in den Fußraum geführt.

\begin{figure}[htbp]
    \centering
    \includegraphics[width=5cm]{img/Klaviatur.jpg}
    \caption{Bohrung durch das Tastenbrett}
    \label{fig:klaviatur}
\end{figure}

\subsubsection{Anordnung der Aktuatoren}

Die Idee ist, die Hubmagnete parallel zur Klaviatur zu befestigen, um die Tasten möglichst senkrecht nach unten ziehen zu können.
Hierfür müssen folgende Aspekte berücksichtigt werden:

\begin{enumerate}
    \item Breite der Hubmagnete
    \item Hitzeentwicklung
    \item Seilführung
    \item Stabilität
    \item Modularität
\end{enumerate}

Ein Hubmagnet hat die Maße 2,5 cm x 6 cm.
Das Tastenbrett für 88 Tasten ist allerdings nur 140 cm breit, wodurch pro Tastenansteuerung nur ca. 1,6 cm zur Verfügung stehen.
Bildet man zwei Hubmagnetreihen übereinander, hat jede Tastenansteuerung 3,20 cm Platz.

Durch die 7mm Abstand ist eine Wärmeabfuhr über die Luft in geringem Ausmaß möglich.
Ob dies ausreicht, muss in späteren Tests (siehe Kapitel \ref{tests}) ermittelt werden.

Alle Seile werden pro Hubmagnetreihe, parallel zwischen Tasten und Hubmagneten verbunden.

Die Hubmagnete direkt unter dem Tastenbrett zu befestigen wäre eine triviale Lösung würde allerdings auf Kosten der Stabilität und Funkionalität gehen. % @Note(Val): Inwieweit? Hier fehlt eine Erklärung
Deshalb werden die Hubmagnete am Korpus des Klaviers (untere Frontplatte) befestigt.

Um die Austauschbarkeit der Komponenten zu verbessern, werden die Hubmagnete nicht direkt an der unteren Frontplatte,
sondern auf zwei Pressspanplatten (ca. 70 cm x 25 cm) befestigt, welche an vier Punkten mit dem Klavier verschraubt werden (siehe Abb. \ref{fig:BefestigungHubmagnete}). % @Note(Val): Die Abbildung zeigt mir nicht, wie die Platte mit dem Klavier verschraubt wird

\begin{figure}[htbp]
    \centering
    \includegraphics[width=5cm]{img/Magnetbrett.jpg}
    \caption{Befestigung der Hubmagnete}
    \label{fig:BefestigungHubmagnete}
\end{figure}

\subsubsection{Seilführung}

Würde man die Seile von den Tasten direkt zu den Hubmagneten führen,
hätte man deutliche Hub-Verluste und könnte unter Umständen die Tasten nicht mehr ausreichend stark betätigen, um einen Ton zu erzeugen.
\newline
In den folgenden Abbildungen ist die seitliche Ansicht des Klaviers gezeichnet.
Dabei repräsentieren der schwarze bzw. weiße Block jeweils eine schwarze bzw. weiße Taste.
Der grüne und pinke Strich stehen für die Seile, die den Hubmagneten mit dem Loch in der Taste verbinden.
Unten links sind die zwei übereinanderliegenden Solenoid-Reihen in blau abgebildet.
Der schwarze Strich repräsentiert den Stab im Hubmagneten. % @Note(Val): Die Farbe vom Solenoid kannst du dir fast sparen, wenn du die Grafik so klein abbildest. Das blau kann ich nur mit viel reinzoomen sehen
\newline
Abbildung \ref{img:Umlenkung_locker} zeigt den intuitiven Aufbau mit ausgeschaltetem Solenoid und entsprechend entspanntem Seil und.
Abbildung \ref{img:Umlenkung_gezogen} zeigt den selben Aufbau mit dem Unterschied, dass der Solenoid hier angezogen ist.

\begin{figure}[htbp]
    \centering
    \includegraphics[width=6cm, height=8cm]{img/Umlenkung_locker}
    \caption{Taste locker ohne Umlenkung}
    \label{img:Umlenkung_locker}
\end{figure}

% @Note(Val): Es wäre besser, wenn sich die beiden Bilder mehr unterscheiden würden. Falls das erste Bild stärker zeigen könnte, dass die Seile locker sind oder es eine Kennzeichnung bei den Seilen gibt, wäre das gut. Alternativ könnten auch die Tasten runtergedrückt sein, um anzuzeigen, dass diese damit betätigt werden
\begin{figure}[htbp]
    \centering
    \includegraphics[width=6cm, height=8cm]{img/Umlenkung_gezogen}
    \caption{Taste gezogen ohne Umlenkung}
    \label{img:Umlenkung_gezogen}
\end{figure}

% @Note(Val): Warum gibt es hier einen forcierten Seitenumbruch? Braucht es das?
\newpage

Mit diesem intuitiven Aufbau funktioniert das Betätigen der Tasten nicht, da diese nicht weit genug heruntergezogen werden.
Um den gesamten Hub des Magneten zu Nutzen und an die Taste weiterzugeben, müssen Umlenkungen eingebaut werden.
Mittels dieser Umlenkung, die mit einem PVC-Rohr umgesetzt werden kann, werden die Seile so geführt, das sie senkrecht auf die Hubmagnete fallen.
Somit wird die Tiefe, mit der die Tasten gedrückt werden, wieder auf annähernd 1 cm erhöht, was dem Hub des Magneten entspricht..

% @Note(Val): Wenn die Umlenkung weiter unten im Bild wäre, wären die unterschiedlichen Linien leichter zu sehen
\begin{figure}[htbp]
    \centering
    \includegraphics[width=6cm, height=8cm]{img/mitUmlenkung_locker}
    \caption{Taste mit Umlenkung}
\end{figure}


% @Note(Val): Entweder machst du das hier als Bullet-Points or mit vollständigen Sätzen. Aber so ist das ein bisschen merkwürdig
Dies kann auch mathematisch bewiesen werden:
\newline geg.: % @Note(Val): "gegeben" aber ohne "gesucht" und ohne Demarkierung wann etwas nicht mehr gegeben ist... können wir das "geg." einfach streichen?
\newline Strecke bis zur ersten Reihe Hubmagnete h1 = 17 cm % @Note(Val): Strecke von wo?
\newline Strecke bis zur zweiten Reihe Hubmagnete h2 = 27 cm
\newline Strecke bis zu den schwarzen Tasten t1 = 12 cm
\newline Strecke bis zu den weißen Tasten t2 = 18 cm
\newline Seil zur schwarzen Taste im entspannten Zustand $ht1_{entspannt}$ % @Note(Val): Du meinst Seillänge? Ist die Seillänge nicht konstant? Warum berechnen wir die?
\newline Seil zu den weißen Tasten im entspannten Zustand $ht2_{entspannt}$
\newline $ht1_{entspannt}$ = $\sqrt {h1^{2} + t1^{2}}$ = 20.81 cm
\newline $ht2_{entspannt}$ = $\sqrt {h2^{2} + t2^{2}}$ = 32.45 cm
\newline $ht1_{gespannt}$ = $\sqrt {(h1 + 1) ^{2} + t1^{2}}$ = 21.63 cm
\newline $ht2_{gespannt}$ = $\sqrt {(h2 + 1)^{2} + t2^{2}}$ = 33.29 cm

Die Differenz zwischen $ht1_{gespannt}$ und $ht1_{entspannt}$ (bzw. ht2) beschreibt die Tiefe, die eine Taste gedrückt werden kann.
\newline Diese beträgt also ohne Optimierung nur 0.82 cm, bzw. 0.84 cm. % @Note(Val): "ohne Optimierung" - was für Optimierungen wären hier denn überhaupt möglich? Ist die Verwendung einer Umlenkun nicht schon eine Optimierung?

Um also die Intensität des Tastendrucks besser steuern zu können und um die Haltbarkeit des Materials zu verlängern,
sollte die Reibung am Seil möglichst gering gehalten werden.
Dazu können mittels Rohren, welche waagerecht unter dem Klaviaturbalken entlang der Löcher montiert werden können (siehe Abbildung \ref{fig:fussraum}), die Seile umgelenkt und
entlang der Verkleidung zur unteren Frontplatte des Klaviers geführt werden.
Das hat den positiven Nebeneffekt, dass die Pianist:innen nicht durch Platzmangel im Beinbereich eingeschränkt sind.


\begin{figure}[htbp]
    \centering
    \includegraphics[width=5cm,angle=-90]{img/Fussraum.jpg}
    \caption{Fußraum des Klaviers}
    \label{fig:fussraum}
\end{figure}


\subsubsection{Auswahl des Seils}

Wie eben beschrieben wird das Ziehen durch ein Seil ermöglicht.
Dafür wird ein leichtes, formbares, unelastisches und dehnungsresistentes Material benötigt, welches die Hubmagnete mit den Tasten verbindet.
Je leichter das Material ist, desto weniger geht die Genauigkeit der Kraftübertragung zwischen den Magneten und den Tasten verloren.
Durch die Formbarkeit kann der Knoten sehr eng an der Taste geschnürt werden, wodurch das Ansprechverhalten schneller erfolgen kann.
Da die Anschläge ruckartige Bewegungen sind, ist es wichtig ein unelastisches Seil zu verwenden.
Um häufiges Nachspannen oder Austauschen des Seils entgegenzuwirken, sollte dies so dehnungsresistent wie möglich sein.

% @Note(Val): Ab hier wird in der Vergangenheit gesprochen. Ich nehme an, dass das hier aber ok ist, da davon gesprochen wird, was umgesetzt und getestet wurde.
\paragraph{Nähgarn}

Als Erstes wurde Nähgarn, welches zur Hand war, getestet.
Auch doppelt verlegt hielt es der ruckartigen Ziehbewegung (mit 25N) des Hubmagnetens nicht stand.
Vier Fäden funktionierten zu Beginn gut, leierten allerdings schnell aus.

\paragraph{Nylonsaiten}

Als Nächstes wurde die g-Saite einer Gitarre verwendet.
Durch den höheren Durchmesser und das stärkere Material riss und leierte die Saite nicht aus.
Da die Saite kaum Flexibilität liefert, war die Befestigung an der Taste jedoch äußerst schwierig.
Beim Ziehen des Magnetens wurde erst die lockere Schlaufe an der Taste gestreckt, wodurch nicht der vollständige Hub des Magnetens auf die Taste übertragen wurde.

\paragraph{Angelschnur}

% @Note(Val) "Aus Erfahrungen [...] konnte eine Schnur gefunden werden" klingt sehr falsch
Aus Erfahrungen aus dem Angelbereich konnte schnell eine geeignetere Schnur gefunden werden.
Genauer handelt es sich um eine geflochtene Schnur aus Polyethylene.
Mit einem Durchmesser von 1.6 mm ist sie nicht nur sehr dünn und flexibel, sondern kann auch bis zu 7 kg standhalten.
Nach ausgiebigem Testen ist eindeutig, dass die Angelschnur die Anforderungen erfüllt.
Damit ist nun auch das Spielen des Forellenquintetts von Schubert ein Leichtes.

\subsection{Klangdämpfung der Aktuatoren}

\chapterauthor{Olivier Stenzel}

Die Hubmagnete machen beim Anschlagen laute \enquote{Klack} Geräusche, welche von der Melodie des Klaviers ablenken.
Genauer handelt es sich um den Metall-Anker, der gegen das Ende des Metall-Gehäuses stößt.
Auch hierfür gab es mehrere Ideen und Tests, um das Geräusch zu dämpfen:

\subsubsection{Isolierfolie}

Die erste Überlegung war die Auskleidung des Innenraums der Hubmagnete mit Isolierfolie.
Diese Idee wurde wieder verworfen, da das Wissen über die Hitzeentwicklung zu diesem Zeitpunkt noch zu gering war, um sicherzustellen, dass die Isolierung dem standhält. % "dem" meint die Hitzeentwicklung? Dann müsste es "ihr" sein.

\subsubsection{Gummi-Stopper}

Die nächste Idee war das Limitieren des Schlags durch Gummiringe (Dichtungsringe) % @Note(Val): Ein Wort verwenden, oder in einen Satz einbringen (z.B. "Dichtungsringe aus Gummi" oder so)
zwischen Anker und dem äußeren Gehäuse.
Da für die Befestigung keine zufriedenstellende Lösung gefunden wurde, wurde auch diese Idee verworfen.
zurück schellen zu hoch, als dass wir die Klangdämpfung umsetzen würden. % @Note(Val): Das ist kein richtiger Satz und ich habe keine Ahnung was du damit aussagen wolltest. Satz streichen oder verbessern

\subsubsection{Schaumstoff}

Wie im Abbildung \ref{fig:schaumstoff} zu sehen ist, wurde die \enquote{Gummi-Stopper}-Idee durch den Einsatz von Schaumstoff leicht modifiziert. \newline
+ Klopfgeräusch wird vollständig verhindert \newline
- Durch den Schaumstoff wird 1mm des Hubs nicht verwendet, weshalb nur noch 9mm übrig bleiben \newline
- Der Aufbau ist optisch nicht besonders ansprechend

\begin{figure}[htbp]
    \centering
    \includegraphics [width=4cm] {img/Daempfung_Schaumstoff}
    \caption{Hubmagnet: Dämpfung mit Schaumstoff}
    \label{fig:schaumstoff}
\end{figure}

\paragraph{Seil (2mm Durchmesser) um den Anker}

Um das \enquote{Problem} der schlechten Ästetik beim Schaumstoff zu beseitigen, wird nun ein 2mm dickes Seil im Inneren des Gehäuses um den Anker gewickelt. \newline
+ Auch hier konnte das Klopfen vollständig beseitigt werden \newline
+ Von außen sieht man keine Veränderung \newline
+ Das Seil scheint ausführlichen Tests nach der Hitzeentwicklung gut Stand zu halten  \newline
- Durch die Dicke des Seils, werden 2mm des Hubs nicht verwendet, weshalb nur noch 8mm übrig bleiben.

\begin{figure}[htbp]
    \centering
    \includegraphics [width=4cm] {img/Hubmagnet_Seil_Daempfung.jpg}
    \caption{Hubmagnet: Dämpfung mit Seil}
\end{figure}


\subsection{Klavieranbau}
Im Rahmen des Protoypen wurde die Elektrik nicht fest am Klavier verschraubt.
Das Brett mit den Hubmagneten wurde stattdessen nur an das Klavier angelehnt, während die Angelschnüre die Verbindung zu den Tasten herstellen.
Gespannt werden die Seile durch eine Schrauben-Mutter Konstruktion (siehe Abschnitt \ref{subsec:VerbindungTastenAktuatoren}).

\subsection{Ergebnisse des Prototypen}
% @TODO(Val):
Das Anspielen klappt iwie aber noch hakelig.
% TODO(Jay): was funktioniert genau, was ist noch einfach hinzuzufügen, wo gabs Probleme



%%%%%%%%%%%%%%%%%%%%%%%%%%%%%%%%%%%%%%%%%%%%%%%%%%%%%%%%%%
%   Autoren des Abschnitts:
%   Jakob Kautz
%   Olivier Stenzel
%%%%%%%%%%%%%%%%%%%%%%%%%%%%%%%%%%%%%%%%%%%%%%%%%%%%%%%%%%

% !TEX root =  master.tex
\chapter{Ergebnisse - Hardware} \label{ergebnisse}
\chapterauthor{Jakob Kautz, Olivier Stenzel}

\nocite{*}
- Was wurde umgesetzt
- Messungen der Umsetzungen
- Limitationen der Umsetzung

% !TEX root =  master.tex
\chapter{Vorgehen - Software} \label{vorgehenSW}

\nocite{*}
- Teilung des Problems (siehe Unterkapitel) \newline
- Grobe Architektur der Komponenten

\section{PIDI}
1. Existierendes vs. Custom Format\newline
2. Darstellung von Tönen (nicht an geg. Piano mit 88 Tasten gebunden)\newline
3. Speicher vs Performanz im Design des Formats (inkl. Alternativen)

\section{Arduino Logik}
kurzer Überblick, vllt. mit SADT-Diagramm

\section{UI-Arduino Kommunikation}
1. Wie wird Kommunikation mit Arduino i.d.R. gelöst\newline
2. Grundlagen der Kommunikation (Files, Polling, etc.)\newline
3. Prinzip der minimalen Arbeit → Custom Protocol → SPPP\newline
4. Asynchrone Umsetzung → Threading bei UI; Message-Buffer bei Arduino

\section{UI}
1. Grundlagen: Immediate vs. Retained Mode UI\newline
2. Cached Immediate Mode\newline
3. Generelle Diskussion bzgl. API-Design vllt

\section{Parsing von MIDI}
1. Überblick über Midi \newline
2. Vergleich MIDI vs PIDI \newline
(Kapitel vllt uninteressant?)
%%%%%%%%%%%%%%%%%%%%%%%%%%%%%%%%%%%%%%%%%%%%%%%%%%%%%%%%%%
%   Autoren des Abschnitts:
%   Val Richter
%%%%%%%%%%%%%%%%%%%%%%%%%%%%%%%%%%%%%%%%%%%%%%%%%%%%%%%%%%

% !TEX root =  master.tex
\chapter{Umsetzung - Software} \label{umsetzungSW}

\nocite{*}
- Probleme, Schwierigkeiten, Änderungen während der Umsetzung \newline
- vllt. kurze Code-Schnippsel von besonders interessanten Teilen

\section{Arduino - Technische Gegebenheiten und Optimierungen}
- Piano Layout (7 Oktaven, 3 Tasten davor, 1 Taste danach)\newline
- Maximale Anzahl gleichzeitig laufender Motoren\newline
- Werte durch Ausprobieren erraten (Minimaler Wert um Motor anzukriegen; Clock-Rate)\newline
- Begrenzter Speicherplatz\newline
- Konstante ‘Framerate’ ohne delay

\section{Probleme bei der Kommunikation}
% !TEX root =  master.tex
\chapter{Ergebnisse Software} \label{ergebnisseSW}

\nocite{*}
- Was wurde umgesetzt
- UX und UI Design (kurz)
- Limitationen der Umsetzung (nur MIDI, kein Image→/Ton→PIDI; etc.)
%%%%%%%%%%%%%%%%%%%%%%%%%%%%%%%%%%%%%%%%%%%%%%%%%%%%%%%%%%
%   Autoren des Abschnitts:
%   ???
%%%%%%%%%%%%%%%%%%%%%%%%%%%%%%%%%%%%%%%%%%%%%%%%%%%%%%%%%%

% !TEX root =  master.tex
\chapter{Zusammenfassung} \label{fazit}
\chapterauthor{Jakob Kautz}
\nocite{*}
\section{Fazit}
Die Arbeit umfasst eine erfolgreiche Konzeption und Entwicklung des selbstspielenden Pianos, wobei alle in der Zielstellung
definierten Features erfüllt werden konnten: \newline

\begin{tabular}{| m{4cm} | m{8cm} |}
    \hline
    \textbf{Anforderung} &  \textbf{Status}  \\
    \hline
    Flexibilität & Erfüllt: Die Aktuatoren werden von manuellen Tastendrücken nicht beeinflusst \\
    \hline
    Benutzerinterface & Erfüllt \\
    \hline
    Responsivität & Erfüllt \\
    \hline
    Tastenbetätigung & Erfüllt: Die verbundenen Tasten können erfolgreich betätigt werden, wobei die Anzahl der Tasten auf
    40 reduziert wurde\\
    \hline
    Anpassbarkeit & Erfüllt: Die Elektronik und Mechanik sind nicht Klavier-abhängig, wobei der Fußraum zum befästigen der Aktuatoren
    gegeben sein muss oder das Ansteuerungskonzept geändert werden müsste\\
    \hline
    Musikstück-Auswahl & Erfüllt \\
    \hline
    Wiedergabe-Kontrolle & Erfüllt \\
    \hline
    Navigation & Erfüllt \\
    \hline
    MIDI-Integration & Erfüllt \\
    \hline
\end{tabular} \newline
%Note(Jay) help this sounds
Insgesamt konnten im laufe des Projektes viele Erkenntnisse getroffen werden.
Im Bereich der Hardware umfassen diese ein vertieftes Verständnis für Elektronik, zum Beispiel der Bedeutung der
Kabeldicke, der Stromversorgungsanforderungen und der Notwendigkeit einer gründlichen Vorplanung.
Letztendlich gab es viele Dinge die gut liefen, wie auch welche, die Nachträglich anders angegangen werden sollten.
Positiv zu vermerken ist die erfolgreiche Funktionalität des Schaltplans nach einigen Anpassungen sowie die hilfreiche
Nutzung von Tinkercad als Simulationswerkzeug.
Herausforderungen ergaben sich aus dem Zeitmanagement, dem erstmaligen Testen der Schaltung und
der Sicherheit im Umgang mit Elektronik, insbesondere aufgrund des begrenzten Vorwissens der Teammitglieder. Die
Projektgröße und die höheren als erwarteten Kosten, insbesondere durch Preisunterschiede bei den Aktuatoren zwischen
verschiedenen Bezugsquellen, stellten zusätzliche Herausforderungen dar.
Die Erfahrung aus dem Projekt bietet wertvolle Erkenntnisse im Projektmanagement, in der Recherche und im
Fachwissen. Es ist jedoch zu beachten, dass der Bau eines selbstspielenden Klaviers mit erheblichem Arbeitsaufwand und
unvorhersehbaren Kosten verbunden ist.

Im größeren Kontext betrachtet, wurde das Projekt initiiert, um die eigene Umsetzung selbstspielender Klaviere
im Gegensatz zu fertig gekauften Player-Pianos zu untersuchen.
Die Ergebnisse zeigen, dass selbstgebaute Varianten im Vergleich zu käuflichen Produkten erhebliche Kosteneinsparungen
ermöglichen, jedoch mit einem erheblichen Zeitaufwand und komplexeren Anpassungsmöglichkeiten einhergehen. Gekaufte Versionen
bieten möglicherweise eine höhere Qualität und eine elegantere Ästhetik, erfordern jedoch erhebliche finanzielle
Investitionen.
Was Features betrifft, sind diese beim selbst bauen leichter anzupassen.
Dafür ist allerdings die Funktionsfähigkeit und  Mindesthaltbarkeit nicht garantiert. \newline
Insgesamt steht fest, dass das Projekt ein Player-Piano selber zu bauen für die Menschen sinnvoll ist, die
gerne ausprobieren und viel Zeit (rund 40h nur für das Löten der hälfte der Schaltung) für solche Projekte aufbringen.
Es kann auf jeden Fall sehr viel Erfahrung im Bereich Projektmanagement, Recherche und natürlich Fachbezogenes Wissen
mitgenommen werden. Vorab sollte allerdings klar sein, dass das Projekt einen hohen Zeitaufwand und - je nach Anbieter der
Hardware - unerwartet Hohe Kosten mit sich bringt.


\section{Ausblick}

Für zukünftige Entwicklungen bieten sich zahlreiche Möglichkeiten sowohl im Software- als auch im Hardwarebereich des
Projekts. Auf der Hardwareseite könnten ein verbessertes Sicherheitskonzept für Elektronik, die Fertigstellung der
Schaltungen für die restlichen Aktuatoren und die Untersuchung von Schalldämmungstechniken in Betracht gezogen werden.
In Bezug auf Softwareverbesserungen könnte die Unterstützung weiterer Dateiformate wie PDF neben MIDI erwogen werden.
Es ist zu beachten, dass Projekte dieser Art selten abgeschlossen sind, da ständig neue Funktionen hinzugefügt und
Verbesserungen vorgenommen werden können.


%%%%%%%%%%%%%%%%%%%%%%%%%%%%%%%%%%%

%%%%%%%%%%%%%%%%%%%%%%%%%%%%%%%%%%%
% ANHÄNGE
%
% @stud: einzelne Anhänge bearbeiten und eigene Anhänge hier einfügen
%        die nachfolgenden Zeilen deaktivieren, wenn keine Anhänge verwendet werden
%
\initializeAppendix
\input{appendix1}
\input{appendix2}
%%%%%%%%%%%%%%%%%%%%%%%%%%%%%%%%%%%

\singlespacing

%%%%%%%%%%%%%%%%%%%%%%%%%%%%%%%%%%%
% LITERATURVERZEICHNIS
% @stud: Literaturverzeichnis in Datei bibliography.bib anpassen.
%
% Alternative zu Verwendung von \initializeBibliography: Citavi ...
% (dann \initializeBibliography auskommentieren und eigenes LaTex Coding verwenden)
%
\initializeBibliography
%%%%%%%%%%%%%%%%%%%%%%%%%%%%%%%%%%%

%%%%%%%%%%%%%%%%%%%%%%%%%%%%%%%%%%%
% INDEX
% @stud: ggf. Index auskommentieren, wenn nicht benötigt
%
\addcontentsline{toc}{chapter}{Index}
\printindex

\end{document}
