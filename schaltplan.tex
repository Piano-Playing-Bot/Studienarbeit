%! Author = OS
%! Date = 18.02.2024
\chapter{Einleitung} \label{einleitung}

% Preamble
\documentclass[11pt]{article}

% Packages
\usepackage{amsmath}

% Document
\begin{document}

Im Zentrum der Schaltung steht der Mikrocontroller (hier: Arduino Uno R3).
Dieser erhält Daten und Strom über den integrierten USB-Anschluss, welcher mit dem Computer verbunden wird.
Da der Arduino limitierte PWM-fähige Ausgänge bereitstellt, werde Schieberegister (74HC595) verwendet.
Mit jedem "in Reihe" geschaltetem Schieberegister kann die Anzahl PWM-fähiger Ausgänge um 8 erweitert werden.
Der Arduino wird an fünf Stellen mit dem ersten Schieberegister verbunden:

Arduinoport D2:
Serial (SER) Input: serielle Daten werden hier bitweise in das Register geschoben.

Arduinoport D3:
SHCP (Shift Register Clock Input): Dieser Pin wird verwendet, um den Takt für das Verschieben der Daten innerhalb des Schieberegisters anzulegen.
Bei jedem Taktimpuls auf diesem Pin wird das Bit am seriellen Dateneingang in das Register verschoben.
Das bedeutet, dass bei jeder steigenden Flanke des Taktsignals das Datenbit, das am Eingang anliegt, in das Schieberegister übernommen und alle vorhandenen Daten um eine Position verschoben werden.

Arduinoport: D4
STCP (Storage Register Clock Input): Nachdem die Daten in das Schieberegister eingelesen wurden, wird dieser Pin verwendet, um die im Schieberegister vorhandenen Daten in das Ausgangsregister zu übertragen.
Ein Taktimpuls auf diesem Pin bewirkt, dass die Daten vom Schieberegister ins Ausgangsregister übernommen werden, sodass alle Ausgänge gleichzeitig aktualisiert werden.
Das ist besonders relevant, da sonst unter Umständen alle Ausgänge von einer Änderung im letzten Schieberegister bedingst wären.

Arduino GND:
Ground, Output Enable (OE): Der OE-Pin wird genutzt, um die Ausgänge des Schieberegisters global zu aktivieren oder zu deaktivieren, ohne die Daten selbst zu beeinflussen.
Da das Schieberegister zu keiner Zeit deaktiviert sein soll, wird dieser Pin dauerhaft mit dem GND-Pin verbunden.



Die Hubmagnete werden jeweils mit 24V und mit bis zu 400mA betrieben.
Um einen hohen Stromfluss zu steuern, können Transistoren verwendet werden.
Für hohe Spannungen und schnelle Schaltvorgänge eigenen sich besonders Mosfets (Metall-Oxid-Halbleiter-Feldeffekttransistor).
Im Folgenden werden speziell n-Mosfets verwendet, der mit einem Signal zwischen 0V (leitet nicht) und +5V (voll leitend) angesteuert werden kann.
\n
    Der Source-Pin wird mit Ground des Systems, und der Drain-Port mit dem entsprechendem Kontakt am Hubmagneten verbunden.
    Über den

\end{document}