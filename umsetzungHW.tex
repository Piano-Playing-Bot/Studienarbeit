%%%%%%%%%%%%%%%%%%%%%%%%%%%%%%%%%%%%%%%%%%%%%%%%%%%%%%%%%%
%   Autoren des Abschnitts:
%   Jakob Kautz
%   Olivier Stenzel
%%%%%%%%%%%%%%%%%%%%%%%%%%%%%%%%%%%%%%%%%%%%%%%%%%%%%%%%%%

% !TEX root =  master.tex
\chapter{Umsetzung - Hardware} \label{umsetzungHW}
\chapterauthor{Jakob Kautz, Olivier Stenzel}

\nocite{*}
- Probleme, Schwierigkeiten, Änderungen während der Umsetzung

Nachdem die Planungsphase abgeschlossen war, mussten die Überlegungen umgesetzt werden.
Im Rahmen dieser Arbeit wurde ein Prototyp gebaut, welcher 8 Tasten anspielen kann.
Zusätzlich wurde die Elektronik mit LEDs so erweitert, dass man das Drücken von 40 Tasten simulieren kann. % @Note(Val): Entweder hier oder an späterer Stelle erwähnen, dass es an sich trivial aber halt zeitaufwendig wäre, mehr Tasten anspielbar zu machen
% @Note(Val): Ich würde hier vielleicht noch gar nicht erwähnen, dass nur 8 Tasten anspielbar sind, weil die ganze Umsetzung ja mit dem Ziel 88 Tasten anspielbar zu machen lief. Kostenschätzung, etc. sind deshalb ja auch höher. Vielleicht ist es besser hier zu sagen, dass das Ziel war, den Prototypen mit möglichst vielen spielbaren Tasten zu gestalten, aber nur 8 bis dato erledigt wurden

\section{Materialien}
\chapterauthor{Jakob Kautz}
\subsection{Liste der Bauteile}
%TODO(Jay): Fix List
\begin{table}[htbp]
    \centering
    \begin{tabular}{|m{3.8cm}|m{1.7cm}|m{8cm}|}
        \hline
        \textbf{Bauteil} &  \textbf{Anzahl} & \textbf{Begründung}  \\
        \hline
        Hubmagnete & 88 & jede Taste braucht einen Hubmagneten um angespielt zu werden \\ % @Note(Val): Erwähnen, dass nur 8 der Hubmagnete benötigt wurden
        \hline
        Stromversorgung & 1 & Externe Stromversorgung für die Aktuatoren, da diese mehr als die 5V Vcc des Arduinos brauchen \\
        \hline
        Arduino & 1 & Kommunikation \\
        \hline
        Breadboard & 2 & Kleinstromschaltung \\
        \hline
        Schaltplatine & 5 & Großstromschaltung\\ % @Note(Jay): Heißt das so?
        \hline
        Schieberegister & 11 & Weitergabe Signal\\
        \hline
        Kabel (10cm) & 352stck (bzw. $9\cdot40$ in Packs) & Verbindungen in der Schaltung mit Schätzung 4 Kabel pro Hubmagnet\\
        \hline
        Kabel (20cm) & 176stck (bzw.$5\cdot40$ in Packs) & Verbindungen zu den Hubmagneten \\
        \hline
        LEDs & 88 & Tests \\
        \hline
        1kOhm Widerstände & 90 & Sicherheit and shit \\
        \hline
        MOSFET & 90 & Steuerung Strom \\
        \hline
        Feste Anschlussblöcke & 88 & Anschluss von Schaltplatine zu Hubmagnet\\
        \hline
        Angelschnur (1m) & 88 & Verbindung Hubmagnet und Taste \\
        \hline
    \end{tabular}
    \caption{Ergebnisse der Anforderungen}
    \label{table:Bauteile}
\end{table}

\subsection{Kostenübernahme}
Die vorher spezifizierte Hardware für den Schaltplan musste für die Erstellung des Prototypen offensichtlich besorgt werden.
Aufgrund der relativ hohen Kosten für eine Studienarbeit, wurde bei einer der betreuenden Firmen angefragt, ob diese die Kosten für das Projekt übernehmen würde.
Damit dies möglich war, wurde ein Kostenvoranschlag gestellt, in welchem die benötigten Materialien mit den geschätzten Kosten aufgeführt wurden. % @TODO(Val): Kostenanschlag wird wahrscheinlich eine Tabelle, also referenzier die einfach "(siehe Tabelle \ref{...})" oder so

% @TODO(Jay): Add Kostenschätzung

Der Kostenanschlag erwies sich im Laufe des Projektes als (teils) unrealistisch. Dies lag insbesondere an der Anforderung
der Firma. Die Schätzung der Kosten basierte auf Anbietern, bei welchen die Materialien möglichst günstig zu kaufen sind.
Durch Firmenreglungen mussten diese allerdings alle bei Conrad oder Reichelt
% @Note(Jay): Stimmt das so?
gekauft werden. Diese Anbieter verkaufen die Materialien für sehr viel mehr Geld. Die tatsächlichen Kosten liefen
letztendlich also auf folgende Beträge hinaus: \newline % @TODO(Val): Hier am besten auch einfach wieder nur die Tabelle referenzieren, statt sie unbedingt in die nächste Zeile zu quetschen
% @TODO(Jay): Tatsächliche Kosten

Der Kostenunterschied betrug daher insgesamt .
% @TODO(Jay): Add difference
Hierbei ist allerdings zu erwähnen, dass die Kostenschätzung passend gewesen wäre, wenn die Anbieter frei wählbar wären.

\section{Prototypenbau}
Ursprünglich sollte der Aufbau des in Kapitel \ref{subsec:schaltplan} spezifizierten Schaltplans via Steckbrettern und
Jumperkabeln umgesetzt werden.
Zu Beginn wurde dies auch so umgesetzt. Das Problem welches dadurch entstand, war, dass die gewählten Platinen den
benötigten Stromfluss nicht aushalten.\newline
Jeder Aktuator - also jede gedrückte Taste - zieht einen Strom von 0.7A. Um das Projekt möglichst sinnvoll umzusetzen,
sollte der Aufbau mindestens 10 Tasten gleichzeitig drücken können, was bedeutet, dass der Aufbau mindestens 7.0A
Stromfluss problemlos ausstehen muss.
% @TODO(Jay): Wie viel halten Platinen aus? Warum konnten wir ein paar Jumper-Kabel nutzen? Wann brauchten wir die dickeren? Tabelle für welche Kabeldicke für welchen Stromfluss
% @Note(Val): Wann haben wir entschieden, dass min. 10 Tasten gleichzeitig spielbar sein sollen? Wenn wir das als Anforderung haben wollen, muss das im Anforderungs-Kapitel auch erwähnt werden

Aus diesem Grund wurde die gesamte Schaltung, die nach dem Schieberegister kommt, auf einer Lochrasterplatine fest gelötet.
Hierfür wurde ein 3mm starker Draht für die Stromversorgung verwendet.

\subsection{Klavieranbau}
Im Rahmen des Protoypen wurde die Elektrik nicht fest am Klavier verschraubt.
Das Brett mit den Hubmagneten wurde stattdessen nur an das Klavier angelehnt, während die Angelschnüre die Verbindung zu den Tasten herstellen.
Gespannt werden die Seile durch eine Schrauben-Mutter Konstruktion (siehe Abschnitt \ref{subsec:VerbindungTastenAktuatoren}).

\subsection{Ergebnisse des Prototypen}
% @TODO(Val):
Das Anspielen klappt iwie aber noch hakelig.
% TODO(Jay): was funktioniert genau, was ist noch einfach hinzuzufügen, wo gabs Probleme


