% !TEX root =  master.tex
\chapter{Umsetzung - Hardware} \label{umsetzung}

\nocite{*}
- Probleme, Schwierigkeiten, Änderungen während der Umsetzung

\section{Prototypenbau}
Breadboard mit einem Motor → Hochskalieren wäre riskant
\section{Ergebnisse des Prototypen}
\section{Klavieranbau}
\section{Tests}

Um eine Überbelastung der Komponenten und eine korrekte Funktionalität zu gewährleisten, müssen Software-seitig Begrenzungen eingeführt werden.
Um zum Beispiel die maximale Geschwindigkeit bestimmen zu könnne mit der ein Stück gespielt werden kann, muss die maximale Frequenz eines physischen Tastendrucks berüclsichtigt werden.
Es wurden folgende Tests durchgeführt:

\paragraph{T1 Max. Frequenz eines Tastendrucks:}
Erklärung: Software-seitig kann sehr genau angegeben werden, wann welche Taste gespielt wird.(siehe: \nameref{vorgehenSW-PIDI})
Hierbei kann auch die Frequenz mit der dieselbe Taste hintereinander angespielt wird bis auf mehrere hundert Hertz erhöht werden.
Da der Hubmagnet, bevor er einen erneuten Tastendruck initiieren kann, wieder in seinen Ausgangszustand zurückkehren muss, ist dieser der limitierende Faktor.
Im Endprodukt soll verhindert werden, dass es halt blöd klingt....
Ablauf: Es
Ergebnis:

\paragraph{T1 Min. Lautstärke:}
Erklärung: iwas mit Hubmagnete anfang überwinden
Ablauf:
Ergebnis:

Max geschwindigkeit, Lautstärke, Akkurarität