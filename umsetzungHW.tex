%%%%%%%%%%%%%%%%%%%%%%%%%%%%%%%%%%%%%%%%%%%%%%%%%%%%%%%%%%
%   Autoren des Abschnitts:
%   Jakob Kautz
%   Olivier Stenzel
%%%%%%%%%%%%%%%%%%%%%%%%%%%%%%%%%%%%%%%%%%%%%%%%%%%%%%%%%%

% !TEX root =  master.tex
\chapter{Umsetzung - Hardware} \label{umsetzungHW}
\chapterauthor{Jakob Kautz, Olivier Stenzel}

\nocite{*}
- Probleme, Schwierigkeiten, Änderungen während der Umsetzung

Nachdem die Planungsphase abgeschlossen war, mussten die Überlegungen umgesetzt werden.
Im Rahmen dieser Arbeit wurde ein Prototyp gebaut, welcher 8 Tasten anspielen kann.
Zusätzlich wurden die Elektronik mit LEDs so erweitert, dass man das Drücken von 40 Tasten simulieren kann.

\section{Prototypenbau}

Die ursprüngliche Idee war ein Aufbau des in Kapitel \ref{subsec:schaltplan} spezifizierten Schaltplans ausschließlich mit Steckbrettern und Jumperkabeln.
Dies wurde zu Beginn auch umgesetzt, bis aufgefallen ist, dass sie nicht für einen hohen Strom ausgelegt sind.
Da wir für jede gedrückte Taste 0.7A benötigen und wir ca. 10 Tasten zeitgleich spielen wollen, muss unser Aufbau mind. 7A unterstützen.
% @Note(Val): "wir" muss weg
% @Note(Val): Wann haben wir entschieden, dass min. 10 Tasten gleichzeitig spielbar sein sollen? Wenn wir das als Anforderung haben wollen, muss das im Anforderungs-Kapitel auch erwähnt werden

Aus diesem Grund wurde die gesamte Schaltung, die nach dem Schieberegister kommt, auf einer Lochrasterplatine fest gelötet.
Hierfür wurde ein 3mm starker Draht für die Stromversorgung verwendet.

\subsection{Klavieranbau}
Im Rahmen des Protoypen wurde die Elektrik nicht fest am Klavier verschraubt.
Das Brett mit den Hubmagneten lehnt am Klavier, während die Angelschnüre die Verbindung zu den Tasten herstellen.
Gespannt werden die Seile durch die Schrauben-Mutter Konstruktion. (siehe Abschnitt \ref{subsec:VerbindungTastenAktuatoren})

\subsection{Ergebnisse des Prototypen}

Das Anspielen klappt iwie aber noch hakelig.


