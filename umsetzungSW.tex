%%%%%%%%%%%%%%%%%%%%%%%%%%%%%%%%%%%%%%%%%%%%%%%%%%%%%%%%%%
%   Autoren des Abschnitts:
%   Val Richter
%%%%%%%%%%%%%%%%%%%%%%%%%%%%%%%%%%%%%%%%%%%%%%%%%%%%%%%%%%

% !TEX root =  master.tex
\chapter{Umsetzung - Software} \label{umsetzungSW}
\chapterauthor{Val Richter}

\nocite{*}
- Probleme, Schwierigkeiten, Änderungen während der Umsetzung \newline
- vllt. kurze Code-Schnippsel von besonders interessanten Teilen

% @TODO(Val): Lesen/Schreiben der Daten auf Arduino

% @TODO(Val): Lesen/Schreiben der Daten auf Windows
% @TODO(Val): Erklärung virtueller COM-Ports auf Windows
% @Note(Val): Documentaion of COM-Ports: https://learn.microsoft.com/en-us/windows-hardware/drivers/serports/configuration-of-com-ports

\section{Arduino - Technische Gegebenheiten und Optimierungen}
- Piano Layout (7 Oktaven, 3 Tasten davor, 1 Taste danach)\newline
- Maximale Anzahl gleichzeitig laufender Motoren\newline
- Werte durch Ausprobieren erraten (Minimaler Wert um Motor anzukriegen; Clock-Rate)\newline
- Begrenzter Speicherplatz\newline
- Konstante ‘Framerate’ ohne delay

\section{Probleme bei der Kommunikation}