%%%%%%%%%%%%%%%%%%%%%%%%%%%%%%%%%%%%%%%%%%%%%%%%%%%%%%%%%%
%   Autoren des Abschnitts:
%   Val Richter
%%%%%%%%%%%%%%%%%%%%%%%%%%%%%%%%%%%%%%%%%%%%%%%%%%%%%%%%%%

% !TEX root =  master.tex
\chapter{Umsetzung - Software} \label{umsetzungSW}
\chapterauthor{Val Richter}

\nocite{*}
- Probleme, Schwierigkeiten, Änderungen während der Umsetzung \newline
- vllt. kurze Code-Schnippsel von besonders interessanten Teilen

% @Note(Val): Documentaion of COM-Ports: https://learn.microsoft.com/en-us/windows-hardware/drivers/serports/configuration-of-com-ports
- Code-Strukturierung \newline
	- keine Klassen, kein OOP \newline
	- Compression-Oriented Programming (siehe cmuratori) \newline
	- Speicherplatz-Trennung über custom Allocators (not that interesting) \newline
	- Parallelisierung basierend auf unabhängigen Codepaths (in der UI) (siehe Blogpost dazu) \newline
	- Thread-Kommunikation über geteilten read-only Speicher \& Mutexes wenn notwendig \newline
	- Error-Handling? \newline
- Kommunikation \newline
	- Lesen/Schreiben der Daten auf Arduino \newline
		- Ring-Buffer zum Lesen \newline
		- Greedy Reading/Parsing bei Music-Nachrichten \newline
	- Lesen/Schreiben der Daten auf Windows \newline
		- Finden von COM-Ports (EnumPorts vs. Register, etc.) \newline
		- Lesen/Schreiben wie bei Dateien \newline
		- keine native Möglichkeit fürs Polling \newline
		- Synchrones Lesen einer gesamten Nachricht kann zu langem Blocken führen \newline
		- Asynchrones Lesn funktioniert nicht \newline
		- Lesen einzelner Bytes zum Befüllen eines buffers (symmetrisch zu MC) \newline
		- Lesen in Endlosschleife in eigenem Thread \newline
		- Schreiben in Chunks (Chunk-Größe über Ausprobieren herausgefunden) \newline
- Arduino-Programmierung \newline
	- Bugs durch begrenzten Speicherplatz \newline
	- Berechnung von Oktave+Note zu Aktuator-Index \newline
	- Strategie für PlayedKeys Elimination \newline
- Code der zwischen UI und MC geteilt wurde \newline

\section{Arduino - Technische Gegebenheiten und Optimierungen}
- Piano Layout (7 Oktaven, 3 Tasten davor, 1 Taste danach)\newline
- Maximale Anzahl gleichzeitig laufender Motoren\newline
- Werte durch Ausprobieren erraten (Minimaler Wert um Motor anzukriegen; Clock-Rate)\newline
- Begrenzter Speicherplatz\newline
- Konstante ‘Framerate’ ohne delay

\section{Probleme bei der Kommunikation}