% !TEX root =  master.tex
\nocite{*}
\chapter{Vorgehen - Software} \label{vorgehenSW}

Wie in Kapitel \ref{Zielstellung} beschrieben, benötigt dieses Projekt einen Desktop-Anwendung für die Nutzer:innen-Interaktion mit dem selbst-spielenden Klavier.
Außerdem wurde in Kapitel \ref{Hardware - Konzeption} bereits aufgezeigt, dass für die Ansteurung der Motoren ein \ac{MC} verwendet werden soll, der natürlich auch eine bestimmte Software ausführen muss.
Beide Komponenten, das \ac{UI} und der \ac{MC}, sollen in diesem Kapitel konzipiert werden.

Abbildung \ref*{fig:high-level-komponenten} zeigt diese unterschiedlichen Komponenten, sowie den Datenfluss zwischen diesen.
Nutzer:innen werden in der Abbildung auch dargestellt, um eindeutig zu machen, dass Nutzer-Eingaben nur an die \ac{UI} gehen.
Der \ac{MC} erhält Daten zum abzuspielenden Musikstück von der \ac{UI} und schickt darauf basierend in regelmäßigen Intervallen \ac{PWM}-Signale an die Hardware.

\begin{figure}[htbp]
	\centering
	\begin{tikzpicture}[node distance=2cm, line width=0.25mm]
		% Code for drawing a stick figure as taken from https://tex.stackexchange.com/a/441952
		% \begin{tikzpicture}
		% 	\thicklines
		% 	\put(5,18){\circle{5}}
		% 	\put(5,7){\line(0,1){8}}
		% 	\put(5,7){\line(1,-2){5}}
		% 	\put(5,7){\line(-1,-2){5}}
		% 	\put(1,12){\line(1,0){8}}
		% \end{tikzpicture}
		\node (user) [rect] {User}; % @TODO: draw Useer as stick figure
		\node (ui) [rect, below of=user] {UI};
		\node (mc) [rect, right of=ui, xshift=1.5cm] {\ac{MC}};
		\node (hw) [rect, right of=mc, xshift=1.5cm] {Hardware};

		\draw [arrow] (user) -- (ui);
		\draw [arrow] (ui) -- (mc);
		\draw [arrow] (mc) -- (hw);
	\end{tikzpicture}
	\caption{Komponenten-Diagramm mit Datenfluss}
	\label{fig:high-level-komponenten}
\end{figure}

Da die Hardware-Komponente in vorherigen Kapiteln bereits behandelt wurde, soll hier nur die Architektur der \ac{UI} (in Abschnitt \ref{vorgehenSW-UI}) und des \ac{MC} (in Abschnitt \ref{vorgehenSW-MC}) erarbeitet werden.
Dabei soll die Kommunikation zwischen diesen beiden Komponenten in Abschnitt \ref{vorgehenSW-SPPP} ebenfalls seperat betrachtet und spezifiziert werden.
Damit können die beiden Komponenten relativ aufwandsarm verändert oder ausgetauscht werden, solange sie sich weiterhin an das selbe Kommunikations-Protokoll halten.

Weiterhin soll auch das Format, in dem Musikstücke auf dem \ac{MC} gespeichert werden, in Abschnitt \ref{vorgehenSW-PIDI} seperat entworfen werden.
Da der \ac{MC} eine möglichst hohe Geschwindigkeit für die Ansteuerung der Motoren erbringen soll, muss das Format eine performante Nutzung ermöglichen.
Auch muss das Format möglichst wenig Speicherplatz aufbrauchen, damit möglichst viele Daten auf einmal auf dem \ac{MC} gespeichert werden können.
Je mehr Speicherplatz das Format nämlich benötigt, desto öfter muss die \ac{UI} Daten an den \ac{MC} schicken, was einen hohen Zeitaufwand mit sich bringen kann.

Zuletzt soll außerdem auch die Architektur der \ac{UI} in zwei Teile geteilt werden.
Die \ac{UI} hat an sich drei klar separierbare Aufgaben.
Zum einen muss die \ac{UI} natürlich Nutzer:innen-Eingaben verarbeiten und ein graphisches Interface rendern.
Zum anderen muss die \ac{UI} wie bereits erwähnt auch Daten an den \ac{MC} schicken.
Wie in Unterkapitel \ref{vorgehenSW-SPPP} aufgezeigt wird, muss die \ac{UI} auch Daten vom \ac{MC} empfangen können.
Die UI muss zusätzlich zum Rendering und der Kommunikation aber auch noch das Importieren von Musikstücken erlauben.
Wie in \ref{Zielstellung} spezifiziert, sollen Nutzer:innen Musikstücke in bestimmten Dateiformaten an die Anwendung geben können, um diese danach auf dem selbst-spielenden Klavier abspielen zu können.
Konzeptionelle Überlegungen dazu werden in \ref{vorgehenSW-MIDI} erläutert.


\section{PIDI} \label{vorgehenSW-PIDI}
\begin{enumerate}
	\item Existierendes vs. Custom Format
	\item Darstellung von Tönen (nicht an geg. Piano mit 88 Tasten gebunden)
	\item Speicher vs Performanz im Design des Formats (inkl. Alternativen)
\end{enumerate}

\section{Logik des Microcontrollers} \label{vorgehenSW-MC}
kurzer Überblick, vllt. mit SADT-Diagramm

\section{UI-Arduino Kommunikation} \label{vorgehenSW-SPPP}
\begin{enumerate}
	\item Wie wird Kommunikation mit Arduino i.d.R. gelöst
	\item Grundlagen der Kommunikation (Files, Polling, etc.)
	\item Prinzip der minimalen Arbeit → Custom Protocol → SPPP
	\item Asynchrone Umsetzung → Threading bei UI; Message-Buffer bei Arduino
\end{enumerate}

\section{UI} \label{vorgehenSW-UI}
\begin{enumerate}
	\item Grundlagen: Immediate vs. Retained Mode UI
	\item Cached Immediate Mode
	\item Generelle Diskussion bzgl. API-Design vllt
\end{enumerate}

\section{Parsing von MIDI} \label{vorgehenSW-MIDI}
\begin{enumerate}
	\item Überblick über Midi
	\item Vergleich MIDI vs PIDI
\end{enumerate}
(Kapitel vllt uninteressant?)