% !TEX root =  master.tex
\chapter{Vorgehen - Software} \label{vorgehenSW}

\nocite{*}
- Teilung des Problems (siehe Unterkapitel) \newline
- Grobe Architektur der Komponenten

\section{PIDI}
1. Existierendes vs. Custom Format\newline
2. Darstellung von Tönen (nicht an geg. Piano mit 88 Tasten gebunden)\newline
3. Speicher vs Performanz im Design des Formats (inkl. Alternativen)

\section{Arduino Logik}
kurzer Überblick, vllt. mit SADT-Diagramm

\section{UI-Arduino Kommunikation}
1. Wie wird Kommunikation mit Arduino i.d.R. gelöst\newline
2. Grundlagen der Kommunikation (Files, Polling, etc.)\newline
3. Prinzip der minimalen Arbeit → Custom Protocol → SPPP\newline
4. Asynchrone Umsetzung → Threading bei UI; Message-Buffer bei Arduino

\section{UI}
1. Grundlagen: Immediate vs. Retained Mode UI\newline
2. Cached Immediate Mode\newline
3. Generelle Diskussion bzgl. API-Design vllt

\section{Parsing von MIDI}
1. Überblick über Midi \newline
2. Vergleich MIDI vs PIDI \newline
(Kapitel vllt uninteressant?)